\subsection{Semantics of ForeC}
\label{sec:forec_semantics}
This section presents the semantics of ForeC as rewrite rules
in the style of structural operational semantics (SOS)~\cite{semantics_sos}. 
The semantics is inspired by that of other synchronous programming
languages (Esterel~\cite{timed_compiling_esterel} and \pretc{}~\cite{pret_pretc} in particular).
The semantics is defined on a set of primitive ForeC constructs
(the kernel of Table~\ref{table:forec_kernel}) from which the full ForeC 
constructs are derived. The kernel constructs are not used for compiling
and only consider a subset of the C language: the assignment operator (\verb$=$), 
the statement terminator (\texttt{;}) for sequencing,
and the \verb$if$ and \verb$while$ statements. 
Table~\ref{table:forec_structural_translations} shows how the
ForeC constructs (Table~\ref{table:forec:semantics}) are translated into the kernel 
constructs (Table~\ref{table:forec_kernel}). This is exemplified by the 
translation of the ForeC constructs in Figure~\ref{fig:transform_forec_refactored} 
into the kernel constructs in Figure~\ref{fig:transform_forec_kernel}. 
The translations for \verb$input$, \verb$output$, and \verb$pause$ are
straightforward. A \verb$shared$ variable is translated into
a global variable and a \verb$copy$ kernel statement that is placed at
the start of every thread body in the scope of the shared variable.
The \verb$copy$ kernel statement initiates the copying of the shared 
variables when the threads are forked and when the threads
start their local ticks.
The \verb$par$ statement is translated by prefixing each
thread body \body{} with a unique identifier \thread{} to 
allow the semantics to distinguish the body of one thread from another. 
The \verb$par$ kernel statement handles the resynchronization of the shared variables.
Traditionally, \verb$trap$s~\cite{timed_compiling_esterel} are used 
to translate \verb$abort$s and other complex preemption statements. 
In contrast, a simpler \verb$abort$ translation is 
possible in ForeC because \verb$abort$ is the only type of preemption statement.
Each \verb$abort$ is assigned a unique identifier $a$ and 
translated into the \verb$status$ and \verb$abort$ kernel
statements. The \verb$status$ kernel statement is needed to
define the immediate behavior of an \verb$abort$ and it takes the
unique identifier $a$ and an expression. The expression is
0 (zero) for a non-immediate \verb$abort$, but is \expression{}
(the preemption condition) for an immediate \verb$abort$. The
\verb$abort$ kernel statement takes the unique
identifier $a$ and the \verb$abort$ body~\body{}. The following section
describes the assumptions on ForeC kernel programs to 
simplify the presentation of the formal semantics. The notations, 
semantic functions, and rewrite rules are then presented.

\begin{table}
	\centering
	\def\arraystretch{1.3}
	
	\tbl{ForeC kernel constructs.\label{table:forec_kernel}}{
		\begin{tabular}{p{3cm} l}
			\bf{Kernel Construct}										& \bf{Short Description}			\\ 
			\hline
			\texttt{nop}												& Empty statement					\\
			\body{}\texttt{;}~\body{}									& Sequence operator					\\
			\var{}~\texttt{=}~\expression{}								& Assignment operator				\\
			\texttt{while}~(\expression{})~\body{}						& Loop								\\
			\texttt{if}~(\expression{})~\body{}~\texttt{else}~\body{}	& Conditional						\\
			\texttt{copy}												& Creates copies of shared variables\\
			\texttt{pause}												& Barrier synchronization			\\
			\texttt{par}(\thread{}:\body{},~\thread{}:\body{})			& Fork/join parallelism				\\
			\texttt{status}($a$, \expression{})							& Initial preemption status			\\
			\texttt{weak}?~\texttt{abort}($a$, \body{})					& Abort								\\
		\end{tabular}
	}
	
	\begin{tabnote}
		\Note{Note:}{
			\body{} is an arbitrary composition of kernel 
			constructs, \var{} is a variable, \expression{} is an expression, $t$ is a thread 
			identifier, and $a$ is an abort identifier. A question mark means that the 
			preceding symbol is optional.
		}
	\end{tabnote}
\end{table}
	
\begin{table}
	\centering
	\def\arraystretch{1.3}
	
	\tbl{Structural translations of the ForeC constructs (Table~\ref{table:forec:semantics}) to kernel constructs (Table~\ref{table:forec_kernel}).\label{table:forec_structural_translations}}{
		\begin{tabular}{| l | p{8cm} |}
			\hline
			\bf{ForeC Construct}															& \bf{ForeC Kernel Constructs}																\\ 
			\hline
			\texttt{input} and \texttt{output}												& Translated into global variables.															\\ \hline
			\texttt{shared}																	& Translated into global variables and \texttt{copy} kernel statements that are placed at 
																							  the start of every thread body.															\\ \hline
			\texttt{pause}																	& \texttt{pause}																			\\ \hline
			\texttt{par}(\body{}, \body{})													& \texttt{par}(\thread{}:\body{}, \thread{}:\body{})										\\ \hline
			\texttt{weak}?~\texttt{abort}~\body{}~\texttt{when}~(\expression{})				& \texttt{status}($a$, 0)\texttt{;} \texttt{weak}?~\texttt{abort}($a$, \body{})				\\ \hline
			\texttt{weak}?~\texttt{abort}~\body{}~\texttt{when immediate}~(\expression{})	& \texttt{status}($a$, \expression{})\texttt{;} \texttt{weak}?~\texttt{abort}($a$, \body{})	\\ \hline
		\end{tabular}
	}
\end{table}

\begin{figure}
	\centering 
	\subfloat[Original program.] {
		\begin{minipage}[b]{0.48\textwidth}
			\lstinputlisting[style=full]{./code/forec/formal/transform_forec_original.forec}
			\label{fig:transform_forec_original}
		\end{minipage}
	}
	\hfill
	\subfloat[Transformed program.] {
		\begin{minipage}[b]{0.43\textwidth}
			\lstinputlisting[style=full]{./code/forec/formal/transform_forec_refactored.forec}
			\label{fig:transform_forec_refactored}
		\end{minipage}
	}

	\subfloat[Translated kernel program.] {
		\begin{minipage}[b]{0.59\textwidth}
			\lstinputlisting[style=full]{./code/forec/formal/transform_forec_kernel.forec}
			\label{fig:transform_forec_kernel}
		\end{minipage}
	}
	
	\caption{Example of transforming and translating a ForeC program into the kernel constructs.}
	\label{fig:transform_forec}
\end{figure}


\subsubsection{Assumptions}
\label{sec:forec_semantics:assumptions}
We make the following assumptions about ForeC programs. 
(1) All programs follow safety-critical coding practices,
as discussed in Sections~\ref{sec:introduction:programming}
and \ref{sec:forec:overview}.
Dynamic memory allocation (e.g., \texttt{malloc}) and unstructured 
jumps (e.g., \texttt{goto}) cannot be used, and loops must be bounded. 
Moreover, C expressions may only be constants, variables, 
pointers, and arrays composed with the logical, bitwise, relational, 
and arithmetic operators of C. Arguments of functions and the 
right-hand side of assignment statements must not contain any assignment
operators. The sequencing operator ``\texttt{,}'' of C must not be used. These 
assumptions limit us to a deterministic subset of the C language. 
(2) All recursive function calls or forking of threads are bounded.
This assumption prevents the unbounded 
execution of functions and threads, leading to unbounded
memory use and execution time. 

To simplify the presentation of the semantics, we assume that the 
following transformations have been performed on ForeC programs. 
(1) Inlining of functions at their call sites, so that the semantics can
ignore function calls.
(2) Renaming variables uniquely and hoisting their declarations 
up to the program's global scope, so that the semantics can ignore 
(static) memory allocation and focus on the semantics of private variables 
(accessible to only one thread) and shared variables.
(3) Replacing pointers with the variables they reference, so that the 
semantics can ignore pointer analysis~\cite{BussBSE10,HardekopfL11}.
Consider the program of 
Figure~\ref{fig:transform_forec_original} that is 
transformed into the equivalent program of 
Figure~\ref{fig:transform_forec_refactored}.
The shared variable declaration for \verb$s$ 
(line~\ref{code:transform_forec_original_s} in Figure~\ref{fig:transform_forec_original}) 
is hoisted to the global scope 
(line~\ref{code:transform_forec_refactored_s} in Figure~\ref{fig:transform_forec_refactored}). 
The function \verb$f$ 
(line~\ref{code:transform_forec_original_f} in Figure~\ref{fig:transform_forec_original})
is inlined into the \verb$abort$ body 
(line~\ref{code:transform_forec_refactored_abort} in Figure~\ref{fig:transform_forec_refactored}) 
and the pointer inside \verb$f$ is replaced by the variable \verb$x$ it references.


%-----------------------------------------------------------------------------

\subsubsection{Notation}
\label{sec:forec_semantics:notation}
The rewrite rules have the following form in the style of 
structural operational semantics (SOS)~\cite{semantics_sos}:
\begin{equation*}
	\langle \StateP \rangle ~ \thread{}:\body{} 
		\xrightarrow[~~\Input~~]{k} 
	\langle \StateP' \rangle ~ \thread{}:\body{}'
\end{equation*}
This notation describes a program fragment \body{} belonging to thread \thread{},
in the program state $\StateP$ and with inputs \Input{}, which
reacts and modifies the program state to $\StateP'$, generates the 
completion code $k$, and becomes the new program fragment $\body{}'$. 
All the (globally declared) \verb$input$s are stored in \Input{}. 
Let \Thread{} be the set of all threads in the program. 
Let $\langle \StateP \rangle = \langle \Environment, \Abort \rangle$, where:
\begin{itemize}
	\item \Environment{} is an environment that maps the program's global scope to the program's global
		  variables and maps the threads' scopes to their local copies of shared variables. 
		  Specifically, \Environment{} is a partial function that maps the global 
		  scope (denoted by \Global{}) and threads ($\thread{} \in \Thread$) to a 
		  store (\Store{}) of variables. Let $\Id = \Thread \cup \{ \Global \}$, then 
		  $\Environment: \Id \hookrightarrow \Store$.
		  $\Environment[\Global]$ stores all the output, shared, and private variables
		  in the program, which are all globally declared thanks to the program 
		  transformations of Section~\ref{sec:forec_semantics:assumptions}. 
		  $\Environment[\thread{}]$ stores thread~\thread{}'s copies of shared variables. 
		  The store (\Store{}) is a partial function that maps variables ($\var \in \Var$) 
		  to values ($\val \in \Val$) and statuses ($\sts \in \Sts$), $\Store: \Var \hookrightarrow (\Val, \Sts)$. 
		  Statuses are used to define the behavior of the combine policies and
		  can be \verb$pre$ (previous resynchronized value), \verb$mod$ (modified value), 
		  \verb$cmb$ (combined value), or \verb$pvt$ (for a private variable). 
		  In $\Environment[\Global]$, the status of a private variable is always \verb$pvt$ 
		  and the status of a shared variable is always \verb$pre$. In $\Environment[\thread{}]$, 
		  a thread's copy of a shared variable always starts each local tick with
		  the status \verb$pre$.
		  		
		  For example, $\Environment = \left \lbrace \Global \to \lbrace s \to (1, \mathtt{pre}) \rbrace, t1 \to \lbrace s \to (3, \mathtt{mod}) \rbrace \right \rbrace$
		  for a program that has a shared variable \verb$s$ with value 1 in the global scope and modified
		  value $3$ in the scope of thread $t1$. We use the
		  notation $\Environment[t1][s]$ to look up the value and status $(3, \mathtt{mod})$ of \verb$s$ in $t1$'s store.
		  We use the notations $\Environment[t1][s].\val$ and $\Environment[t1][s].\sts$
		  to look up its value and status, respectively.
		  We use the notation $\StateP.\Environment$ to retrieve \Environment{} from the program state \StateP{}.
		  
	\item \Abort{} is a partial function that maps \verb$abort$ identifiers ($a \in \mathcal{A}$) 
		  to values ($\val \in \Val$) representing their preemption status, 
		  $\Abort: \mathcal{A} \to Val$. An \verb$abort$ with a non-zero value means that its
		  preemption condition is \emph{true} and that it has been triggered. 
		  
		  For example, $\Abort = \lbrace a1 \to 1, a2 \to 0 \rbrace$ 
		  for a program that has \verb$abort$s $a1$ and $a2$ with the statuses 1 and 0, respectively.
		  We use the notation $\Abort[a1]$ to look up the status of \verb$abort$~$a1$.
		  We use the notation $\StateP.\Abort$ to retrieve \Abort{} from the program state \StateP{}.
\end{itemize}
The transition of a program fragment from \body{} to $\body{}'$ 
is encoded by the completion code $k$, where:
\begin{equation*}
	k = \begin{cases}
		0		& \text{If the transition terminates.}			\\
		1		& \text{If the transition pauses.}				\\
		\bot	& \text{Otherwise (the transition continues).}
	\end{cases}
\end{equation*}


%-----------------------------------------------------------------------------
\subsubsection{Semantic Functions}
The following sections describe the semantic functions that are 
used by the rewrite rules to ensure semantic conciseness.

\subsubsection{Statically Known Information}
The following semantic functions return statically known 
information about the program:
\begin{itemize}
	\item $\GetParent(\thread{})$:
		Returns the parent of thread \thread{}.
		If $\thread{} =$ \verb$main$, then ``\verb$main$'' is returned.
	\item $\GetShared(\Global)$:
		Returns the set of all shared variables declared in the program.
	\item $\GetShared(\thread{})$:
		Returns the set of all shared variables that the body of thread \thread{} accesses (reads or writes).
	\item $\GetCombine(\var)$:
		Returns the combine function of shared variable \var{}.
	\item $\GetPolicy(\var)$:
		Returns the combine policy of shared variable \var{}.
	\item $\GetExp(a)$:
		Returns the preemption condition \expression{} of \verb$abort$ $a$.
\end{itemize}
Figure~\ref{fig:forec_predicates} exemplifies the use of these functions on
the program in Figure~\ref{fig:transform_forec_kernel}.

\begin{figure}
	\centering
	\def\arraystretch{1.3}
	
	\begin{tabular}{|l l l|}
		\hline
		$\GetParent($\verb$main$$) =$ \verb$main$	& $\GetParent($\verb$t1$$) =$ \verb$main$		& $\GetParent($\verb$t2$$) =$ \verb$main$	\\
		$\GetShared(\Global) = \{$\verb$s$$\}$		& $\GetShared($\verb$t1$$) = \{$\verb$s$$\}$	& $\GetShared($\verb$t2$$) = \{$\verb$s$$\}$\\
		$\GetCombine($\verb$s$$) =$ \verb$plus$		& $\GetPolicy($\verb$s$$) =$ \verb$all$			& $\GetExp($\verb$a1$$) =$ \verb$s>3$		\\
		\hline
	\end{tabular}
	
	\caption{Retrieving statically known information about Figure~\ref{fig:transform_forec_kernel}.}
	\label{fig:forec_predicates}
\end{figure}

\subsubsection{\sc{Eval}}
\label{sec:forec_Eval}
The semantic function $\Eval(\Environment, \Input, \id, \expression)$ 
follows the evaluation rules of C to evaluate the expression
\expression{} and return its value. 
The expression \expression{} has a classical tree structure: it can
be an atom (a constant, a variable, a string, ...), a unary
arithmetic or Boolean operation (\texttt{*}, \texttt{\&}, \texttt{!}, 
\texttt{-},~\texttt{\~}), a binary arithmetic or Boolean 
operation (\texttt{||}, \texttt{\&\&}, \texttt{\^}, 
\texttt{|}, \texttt{\&}, \texttt{<}\texttt{<}, \texttt{>}\texttt{>}, \texttt{==}, 
\texttt{!=}, \texttt{<}, \texttt{>}, \texttt{<=}, \texttt{>=}, \texttt{+},
\texttt{-}, \texttt{*}, \texttt{/},~\texttt{\%}), a function call with its arguments
passed by value and by reference, an array, and so on. For
the sake of simplicity, we do not give the details 
here~\cite{programming_languages_clight,programming_languages_cholera} and
will just write the string of the expression when calling
the $\Eval$ function. Finally, the $\Eval$ function returns the
value of \expression{}. Unlike in C, where expressions can have 
side-effects (which would be captured by the function $\Eval$
returning a \emph{pair} $(\Environment', \val)$ instead of just $\val$),
we have assumed that ForeC expressions and functions are
side-effect free. During the evaluation, a
variable's value is retrieved with the semantic function
$\GetVal(\Environment, \Input, \id, \var)$ described by
Algorithm~\ref{algo:GetVal}. The inputs to the algorithm
are: the program's environment \Environment{}, the inputs
\Input{}, the identifier \id{} of the
store to try and retrieve the value from, and the variable
\var{} of interest. The output is a value \val{}. If \var{}
is an input, then line~\ref{algo:GetVal_input} returns its 
value. Otherwise, if \var{} is in
\id{}'s store, then line~\ref{algo:GetVal_local} returns its 
value. Otherwise, line~\ref{algo:GetVal_global} returns the 
global value of \var{}.

\begin{algorithm}
	\KwIn{Program's environment \Environment{}, inputs \Input{}, identifier \id{} of the store to search, and variable \var{} of interest.}
	\KwOut{Value of \var{}.}
	
	\BlankLine
	
	\uIf(\tcp*[f]{If \var{} is an input.}){$\var \in \Input$} {
		\KwRet{$\Input[\var]$}						\tcp*{Return the input value of \var{}.}\label{algo:GetVal_input}
	} \uElseIf(\tcp*[f]{Otherwise, if a local copy of \var{} exists.}){$\var \in \Environment[\id]$} {
		\KwRet{$\Environment[\id][\var].\val$}		\tcp*{Return the value of \var{} from \id{}'s store.}\label{algo:GetVal_local}
	} \Else {
		\KwRet{$\Environment[\Global][\var].\val$}	\tcp*{Otherwise, return the global value of \var{}.}\label{algo:GetVal_global}
	}
	
	\caption{$\GetVal(\Environment, \Input, \id, \var)$: Gets the value of a given variable.}
	\label{algo:GetVal}
\end{algorithm}

\subsubsection{\sc{Copy}}
\label{sec:forec_Copy}
The semantic function $\Copy(\Environment, \thread{})$
creates in thread \thread{} the local copies of each
shared variable $\var \in \GetShared(\thread{})$ 
that it does not have. That is, if thread
\thread{} already has a copy of the shared variable
\var{}, then $\Copy$ skips the copying of \var{}.
This conditional behavior is needed because the semantic
function $\Copy$ may be invoked for a thread \thread{} that
already has a subset of its required local copies. For
example, when local copies are created for a parent thread
that is resuming from the termination of a \texttt{par}, 
the combined values from its child threads
must not be overwritten. 
The $\Copy$ function is described
by Algorithm~\ref{algo:Copy}. The inputs to the algorithm
are: the program's environment \Environment{} and a thread
\thread{}. The output is an updated environment
\Environment{}. Line~\ref{algo:Copy_retrieve} considers 
each shared variables that is accessed in the thread's 
body. For each shared
variable\footnotemark[1], line~\ref{algo:Copy_if} checks if
a copy already exists. If it does not exist, then 
lines~\ref{algo:Copy_parent_val}--\ref{algo:Copy_parent}
copy the parent thread's copy if available, 
otherwise from the shared variable 
(line~\ref{algo:Copy_global}). Line~\ref{algo:Copy_return}
returns the updated environment \Environment{}.

\footnotetext[1]{Recall from
Section~\ref{sec:forec_semantics:notation} that
\Environment{} maps the global and thread scopes to their
own store of variables, $\Environment: \Id \hookrightarrow
\Store$. Variables are mapped to a value and status,
$\Store: \Var \hookrightarrow (\Val, \Sts)$ where 
$\Sts = \{ \texttt{pre}, \texttt{mod}, \texttt{cmb}, \texttt{pvt} \}$.
A private variable has the status \texttt{pvt}, a shared
variable has the status \texttt{pre}, and a thread's copy of
a shared variable starts each local tick with the status
\texttt{pre}. The notation $\Environment[\thread{}][\var]$
looks up the value and status $(\val, \sts)$ of thread
\thread{}'s copy of \var{}.}

\begin{algorithm}
		\KwIn{Program's environment \Environment{}, and thread \thread{}.}
		\KwIn{Updated environment \Environment{}.}
		
		\BlankLine
		
		\ForAll(\tcp*[f]{For all shared variables needed by thread \thread{}.}){$\var \in \GetShared(\thread{})$} {	\label{algo:Copy_retrieve}
			\If(\tcp*[f]{If thread \thread{} does not have a copy.}){$\var \notin \Environment[\thread{}]$} {		\label{algo:Copy_if}
				\uIf(\tcp*[f]{If its parent has a copy.}){$\var \in \Environment[\GetParent(\thread{})]$} {
					$\val := \Environment[\GetParent(\thread{})][\var].\val$										\tcp*{Value of its parent's copy.}\label{algo:Copy_parent_val}
					$\Environment[\thread{}][\var \gets (\val, \texttt{pre})]$										\tcp*{Copy its parent's copy.}\label{algo:Copy_parent}
				} \Else(\tcp*[f]{Otherwise, its parent does not have a copy.}) {
					$\Environment[\thread{}][\var \gets \Environment[\Global][\var]]$								\tcp*{Copy the shared variable from the global scope.}\label{algo:Copy_global}
				}
			}
		}
		\KwRet{$\Environment$}																						\label{algo:Copy_return}
	
	\caption{$\Copy(\Environment, \thread{})$: Copies all the shared variables needed by a thread.}
	\label{algo:Copy}
\end{algorithm}

\subsubsection{\sc{Combine}}
\label{sec:forec_Combine}
The semantic function $\Combine(\Environment, \thread{1}, \thread{2}, \thread{0})$ 
combines all the copies of shared variables from two threads
and is described by Algorithm~\ref{algo:Combine}.
The inputs to the algorithm are: the program's environment
\Environment{}, two threads \thread{1} and \thread{2} to
combine, and thread \thread{0} to store the combined values.
The output is an updated environment \Environment{}. 
Line~\ref{algo:Combine_for} considers each shared variable\footnotemark[1]
\var{}.
Line~\ref{algo:Combine_pre} gets the shared variable's
\verb$pre$ value (\verb$preVal$). For the combine policy \verb$all$, the
copies from both threads are combined if they exist. Thus,
line~\ref{algo:Combine_default} gets the set of threads
\Thread{} that have a copy of the shared variable. If the
combine policy is \verb$new$, then line~\ref{algo:Combine_new} 
keeps only the copies with values that differ from the shared 
variable's \verb$pre$ value
($\Environment[\thread{}][\var].\val \neq \texttt{preVal}$) 
or copies that have been combined
($\Environment[\thread{}][\var].\sts = \texttt{cmb}$). 
If the combine policy is \verb$mod$, then line~\ref{algo:Combine_mod} 
keeps only the modified or combined copies 
($\Environment[\thread{}][\var].\sts \in \{ \texttt{mod}, \texttt{cmb} \}$). 
If two copies are
found, then line~\ref{algo:Combine_c} gets the shared 
variable's combine function (\emph{cf}) and line~\ref{algo:Combine_combine}
computes the combined value. Line~\ref{algo:Combine_assign2} 
assigns the combined value to thread \thread{0} with the
status \verb$cmb$ because it is now a combined value. 
If only one copy is found, then line~\ref{algo:Combine_assign1} 
assigns the value of that copy to thread \thread{0} with the
status \verb$cmb$. Line~\ref{algo:Combine_return} returns the 
updated environment \Environment{} restricted to \thread{0}
(i.e., without thread \thread{1} 
and \thread{2}'s store).

\begin{algorithm}
		\KwIn{Program's environment \Environment{}, threads \thread{1} and \thread{2} to combine, and thread \thread{0} to store the combined values.}
		\KwOut{Updated environment \Environment{}.}
		
		\BlankLine
		
		\ForAll(\tcp*[f]{For all shared variables.}){$\var \in \GetShared(\Global)$} {										\label{algo:Combine_for}
			$\texttt{preVal} := \Environment[\Global][\var].\val$															\tcp*{Get the \texttt{pre} of \var{}.}\label{algo:Combine_pre}
			$\Thread := \lbrace \thread{} \mid \thread{} \in \lbrace \thread{1}, \thread{2} \rbrace, \var \in \Environment[\thread{}] \rbrace$		\tcp*{Set of threads with a copy of \var{}.}\label{algo:Combine_default}
			\uIf{$\GetPolicy(\var) = \texttt{new}$} {
				\tcp{Keep only the copies that differ from \texttt{preVal} or have been combined.}
				$\Thread := \lbrace \thread{} \mid \thread{} \in \Thread, \Environment[\thread{}][\var].\val \neq \texttt{preVal} \lor \Environment[\thread{}][\var].\sts = \texttt{cmb} \rbrace$\;	\label{algo:Combine_new}
			} \ElseIf{$\GetPolicy(\var) = \texttt{mod}$} {
				\tcp{Keep only the modified or combined copies.}
				$\Thread := \lbrace \thread{} \mid \thread{} \in \Thread, \Environment[\thread{}][\var].\sts \in \{ \texttt{mod}, \texttt{cmb} \} \rbrace$\;	\label{algo:Combine_mod}
			}
			
			\BlankLine
			
			\uIf(\tcp*[f]{If there are two copies to combine.}){$\lvert \Thread \rvert = 2$} {			\label{algo:Combine_if}
				$\mathit{cf} := \GetCombine(\var)$														\tcp*{Get the combine function of \var{}.}\label{algo:Combine_c}
				$\val := cf(\Environment[\thread{1}][\var].\val, \Environment[\thread{2}][\var].\val)$	\tcp*{Combine the copies.}\label{algo:Combine_combine}
				$\Environment[\thread{0}][\var \gets (\val, \texttt{cmb}) ]$							\tcp*{Assign the combined value to \thread{0}.}\label{algo:Combine_assign2}
			} \ElseIf(\tcp*[f]{Otherwise, there is only one copy.}) {$\lvert \Thread \rvert = 1$} {
				$\Environment[\thread{0}][\var \gets (\Environment[\thread{} \in \Thread][\var].\val, \texttt{cmb}) ]$	\tcp*{Assign the only copy to \thread{0}.}\label{algo:Combine_assign1}
			}
		}
		$\Environment' = \{ (id, store) \mid (id, store) \in \Environment \land id \neq \thread{1} \land id \neq \thread{2} \}$\;
		\KwRet $\Environment'$																			\label{algo:Combine_return}
	
	\caption{$\Combine(\Environment, \thread{1}, \thread{2}, \thread{0})$: Combines the copies of shared variables from two threads.}
	\label{algo:Combine}
\end{algorithm}

\clearpage

%-----------------------------------------------------------------------------

\subsubsection{The Structural Operational Semantics}
\label{sec:forec_semantics:sos}
This section presents the operational semantics of the kernel
constructs presented in Table~\ref{table:forec_kernel}.

\subsubsection{The \texttt{nop} Statement}
The \verb$nop$ statement does nothing and terminates instantly:
\begin{equation*}
	\tag{nop}
	\label{forec:nop}
	\langle \Environment, \Abort \rangle ~ \thread{}:\mathtt{nop}
		\xrightarrow[~~\Input~~]{0} 
	\langle \Environment, \Abort \rangle ~ \thread{}:
\end{equation*}

\subsubsection{The \texttt{copy} Statement}
The \verb$copy$ statement copies the shared variables needed
by thread \thread{} and terminates instantly. The combining of 
the copies is handled by the \verb$par$ statement:
\begin{equation*}
	\tag{copy}
	\label{forec:copy}
	\langle \Environment, \Abort \rangle ~ \thread{}:\mathtt{copy}
		\xrightarrow[~~\Input~~]{0} 
	\langle \Copy(\Environment, \thread{}), \Abort \rangle ~ \thread{}:
\end{equation*}

\subsubsection{The \texttt{pause} Statement}
The \verb$pause$ statement rewrites into the \verb$copy$ statement
and pauses. The \verb$copy$ statement ensures that thread \thread{} 
starts its next local tick by copying the shared variables it needs
(the \verb$pre$ values are copied):
\begin{equation*}
	\tag{pause}
	\label{forec:pause}
	\langle \Environment, \Abort \rangle ~ \thread{}:\mathtt{pause}
		\xrightarrow[~~\Input~~]{1} 
	\langle \Environment, \Abort \rangle ~ \thread{}:\mathtt{copy}
\end{equation*}

\subsubsection{The \texttt{status} Statement}
Recall that the \texttt{abort} statement is mapped to a \texttt{status}
statement that evaluates the preemption status, followed by an invocation
of the \texttt{abort} kernel statement that accesses the result of the
evaluated preemption status.

The \verb$status$ statement sets \verb$abort$ $a$'s preemption status
to the value of the expression \expression{}, and then it terminates
instantly:
\begin{equation*}
	\tag{status}
	\label{forec:status}
	\langle \Environment, \Abort \rangle ~ \thread{}:\mathtt{status}(a, \expression)
		\xrightarrow[~~\Input~~]{0} 
	\langle \Environment, \Abort[a \gets \Eval(\Environment, \Input, \thread{}, \expression)] \rangle ~ \thread{}:
\end{equation*}

\subsubsection{The \texttt{abort} Statement}
The \verb$abort$ of $a$ executes its body \body{} if its preemption 
has not been triggered. The body may have paused ($k = 1$) or may have executed
some instantaneous statements ($k = \bot$):
\begin{equation*}
	\tag{abort-1}
	\label{forec:abort-1}
	\frac{
			\langle \Environment, \Abort \rangle ~ \thread{}:\body{}
				\xrightarrow[~~\Input~~]{k \in \lbrace 1, \bot \rbrace} 
			\langle \Environment', \Abort' \rangle ~ \thread{}:\body{}'			
		}{
			\langle \Environment, \Abort \rangle ~ \thread{}:\mathtt{weak}?~\mathtt{abort}(a, \body{})
				\xrightarrow[~~\Input~~]{k} 
			\langle \Environment', \Abort' \rangle ~ \thread{}:\mathtt{weak}?~\mathtt{abort}(a, \body{}')
		}
	(\Abort[a] = 0)
\end{equation*}
The \verb$abort$ terminates normally if its body terminates and its preemption has not
been triggered:
\begin{equation*}
	\tag{abort-2}
	\label{forec:abort-2}
	\frac{
			\langle \Environment, \Abort \rangle ~ \thread{}:\body{}
				\xrightarrow[~~\Input~~]{0} 
			\langle \Environment', \Abort' \rangle ~ \thread{}:
		}{
			\langle \Environment, \Abort \rangle ~ \thread{}:\mathtt{weak}?~\mathtt{abort}(a, \body{})
				\xrightarrow[~~\Input~~]{0} 
			\langle \Environment', \Abort' \rangle ~ \thread{}:
		}
	(\Abort[a] = 0)
\end{equation*}
The \verb$weak abort$ terminates normally if its 
body terminates, even if its preemption has been triggered:
\begin{equation*}
	\tag{abort-3}
	\label{forec:abort-3}
	\frac{
			\langle \Environment, \Abort \rangle ~ \thread{}:\body{}
				\xrightarrow[~~\Input~~]{0} 
			\langle \Environment', \Abort' \rangle ~ \thread{}:
		}{
			\langle \Environment, \Abort \rangle ~ \thread{}:\mathtt{weak}~\mathtt{abort}(a, \body{})
				\xrightarrow[~~\Input~~]{0} 
			\langle \Environment', \Abort' \rangle ~ \thread{}:
		}
	(\Abort[a] \neq 0)
\end{equation*}
The \verb$weak abort$ allows its body to execute instantaneous statements ($k = \bot$), 
even if its preemption has been triggered:
\begin{equation*}
	\tag{abort-4}
	\label{forec:abort-4}
	\frac{
			\langle \Environment, \Abort \rangle ~ \thread{}:\body{}
				\xrightarrow[~~\Input~~]{\bot} 
			\langle \Environment', \Abort' \rangle ~ \thread{}:\body{}'
		}{
			\langle \Environment, \Abort \rangle ~ \thread{}:\mathtt{weak}~\mathtt{abort}(a, \body{})
				\xrightarrow[~~\Input~~]{\bot} 
			\langle \Environment', \Abort' \rangle ~ \thread{}:\mathtt{weak}~\mathtt{abort}(a, \body{}')
		}
	(\Abort[a] \neq 0)
\end{equation*}
The \verb$weak abort$ terminates if its body pauses and its preemption has been triggered,
and then it rewrites into the \verb$copy$ statement because it may be the start of thread \thread{}'s 
local tick\footnotemark[2]:
\footnotetext[2]{\label{sec:forec_semantics:footnote2}The \texttt{abort} may have had a 
\texttt{par} statement that paused. In this case, when the \texttt{abort} kernel statement
preempts, thread \thread{} will start its local tick.}
\begin{equation*}
	\tag{abort-5}
	\label{forec:abort-5}
	\frac{
			\langle \Environment, \Abort \rangle ~ \thread{}:\body{}
				\xrightarrow[~~\Input~~]{1} 
			\langle \Environment', \Abort' \rangle ~ \thread{}:\body{}'
		}{
			\langle \Environment, \Abort \rangle ~ \thread{}:\mathtt{weak}~\mathtt{abort}(a, \body{})
				\xrightarrow[~~\Input~~]{\bot} 
			\langle \Environment', \Abort' \rangle ~ \thread{}:\mathtt{copy}
		}
	(\Abort[a] \neq 0)
\end{equation*}
The strong \verb$abort$ terminates without executing its body if its preemption has been triggered,
and then it rewrites into the \verb$copy$ statement because it may be the start of thread \thread{}'s 
local tick\footnotemark[3]:
\begin{equation*}
	\tag{abort-6}
	\label{forec:abort-6}
	\frac{
			\Abort[a] \neq 0
		}{
			\langle \Environment, \Abort \rangle ~ \thread{}:\mathtt{abort}(a, \body{})
				\xrightarrow[~~\Input~~]{\bot} 
			\langle \Environment, \Abort \rangle ~ \thread{}:\mathtt{copy}
		}
\end{equation*}

\footnotetext[3]{In addition to footnote~\ref{sec:forec_semantics:footnote2}, 
the strong preemption prevents the execution of a \texttt{copy} statement inside the abort body.}

\subsubsection{The Assignment Operator (\texttt{=})}
The assignment operator evaluates the expression
\expression{} into a value 
$\val = \Eval(\Environment, \Input, \thread{}, \expression)$.
If \var{} is a shared variable\footnotemark[4] (rule \ref{forec:assign-shared}), then the value \val{} and status \verb$mod$ 
is assigned to the thread's copy in $\Environment[\thread{}]$. 
Otherwise, if \var{} is a private variable (rule \ref{forec:assign-private}), then the value \val{}
and status \verb$pvt$ is assigned to the global variable in $\Environment[\Global]$:
\begin{equation*}
	\tag{assign-shared}
	\label{forec:assign-shared}
	\frac{
			\var \in \GetShared(\thread{})
		}{
			\langle \Environment, \Abort \rangle ~ \thread{}:\var \mathtt{=} \expression
				\xrightarrow[~~\Input~~]{0} 
			\langle \Environment[\thread{}][\var \gets (\val, \texttt{mod})], \Abort \rangle ~ \thread{}:
		}
\end{equation*}
\begin{equation*}
	\tag{assign-private}
	\label{forec:assign-private}
	\frac{
			\var \notin \GetShared(\thread{})
		}{
			\langle \Environment, \Abort \rangle ~ \thread{}:\var \mathtt{=} \expression
				\xrightarrow[~~\Input~~]{0} 
			\langle \Environment[\Global][\var \gets (\val, \texttt{pvt})], \Abort \rangle ~ \thread{}:
		}
\end{equation*}

\footnotetext[4]{Recall from
Section~\ref{sec:forec_semantics:notation} that
\Environment{} maps the global and thread scopes to their
own store of variables, $\Environment: \Id \hookrightarrow \Store$. 
Variables are mapped to a value and status,
$\Store: \Var \hookrightarrow (\Val, \Sts)$ where 
$\Sts = \{ \texttt{pre}, \texttt{mod}, \texttt{cmb}, \texttt{pvt} \}$.
A private variable has the status \texttt{pvt}, a shared
variable has the status \texttt{pre}, and a thread's copy of
a shared variable starts each local tick with the status
\texttt{pre}. The notation $\Environment[\thread{}][\var]$
looks up the value and status $(\val, \sts)$ of thread
\thread{}'s copy of \var{}.}

\subsubsection{The \texttt{if}--\texttt{else} Statement}
A conditional construct is rewritten into one of its branches, depending on 
the value of its condition \expression{}:
\begin{equation*}
	\tag{if-then}
	\label{forec:if-then}
	\frac{
			\Eval(\Environment, \Input, \thread{}, \expression) \neq 0
		}{
			\langle \Environment, \Abort \rangle ~ \thread{}:\mathtt{if}~(\expression)~\body{1}~\mathtt{else}~\body{2}
				\xrightarrow[~~\Input~~]{\bot} 
			\langle \Environment, \Abort \rangle ~ \thread{}:\body{1}
		}
\end{equation*}
\begin{equation*}
	\tag{if-else}
	\label{forec:if-else}
	\frac{
			\Eval(\Environment, \Input, \thread{}, \expression) = 0
		}{
			\langle \Environment, \Abort \rangle ~ \thread{}:\mathtt{if}~(\expression)~\body{1}~\mathtt{else}~\body{2}
				\xrightarrow[~~\Input~~]{\bot} 
			\langle \Environment, \Abort \rangle ~ \thread{}:\body{2}
		}
\end{equation*}

\subsubsection{The \texttt{while} Statement}
The body of a loop statement is either unrolled once
or it terminates, depending 
on the value of its condition \expression{}:
\begin{equation*}
	\tag{loop-then}
	\label{forec:loop-then}
	\frac{
			\Eval(\Environment, \Input, \thread{}, \expression) \neq 0
		}{
			\langle \Environment, \Abort \rangle ~ \thread{}:\mathtt{while}~(\expression)~\body{}
				\xrightarrow[~~\Input~~]{\bot} 
			\langle \Environment, \Abort \rangle ~ \thread{}:\body{};~\mathtt{while}~(\expression)~\body{}
		}
\end{equation*}
\begin{equation*}
	\tag{loop-else}
	\label{forec:loop-else}
	\frac{
			\Eval(\Environment, \Input, \thread{}, \expression) = 0
		}{
			\langle \Environment, \Abort \rangle ~ \thread{}:\mathtt{while}~(\expression)~\body{}
				\xrightarrow[~~\Input~~]{0} 
			\langle \Environment, \Abort \rangle ~ \thread{}:
		}
\end{equation*}

\subsubsection{The Sequence Operator (\texttt{;})}
For a sequence of program fragments, the first fragment \body{1}
must terminate before the second fragment \body{2} can be rewritten. 
In other words, the (\ref{forec:seq-left}) rule applies up to
the micro-step during which \body{1} emits the completion 
code $0$. At this point, the (\ref{forec:seq-right}) rule applies.
The (\ref{forec:seq-left}) rule emits the completion code of 
the first fragment:
\begin{equation*}
	\tag{seq-left}
	\label{forec:seq-left}
	\frac{
			\langle \Environment, \Abort \rangle ~ \thread{}:\body{1}
				\xrightarrow[~~\Input~~]{k \in \lbrace 1, \bot \rbrace}
			\langle \Environment', \Abort' \rangle ~ \thread{}:\body{1}'
		}{
			\langle \Environment, \Abort \rangle ~ \thread{}:\body{1}\mathtt{;}~\body{2}
				\xrightarrow[~~\Input~~]{k} 
			\langle \Environment', \Abort' \rangle ~ \thread{}:\body{1}'\mathtt{;}~\body{2}
		}
\end{equation*}
\begin{equation*}
	\tag{seq-right}
	\label{forec:seq-right}
	\frac{
			\langle \Environment, \Abort \rangle ~ \thread{}:\body{1}
				\xrightarrow[~~\Input~~]{0}
			\langle \Environment', \Abort' \rangle ~ \thread{}:
		}{
			\langle \Environment, \Abort \rangle ~ \thread{}:\body{1}\mathtt{;}~\body{2}
				\xrightarrow[~~\Input~~]{\bot}
			\langle \Environment', \Abort' \rangle ~ \thread{}:\body{2}
		}
\end{equation*}

\subsubsection{The \texttt{par} Statement}
The \verb$par$ statement allows both of its child threads, $\thread{1}$
and $\thread{2}$, to execute instantaneous statements in parallel. 
The parent thread is $\thread{0}$:
\begin{equation*}
	\tag{par-1}
	\label{forec:par-1}
	\frac{
		\langle \Environment, \Abort \rangle ~ \thread{1}:\body{1}
			\xrightarrow[~~\Input~~]{\bot} 
		\langle \Environment', \Abort' \rangle ~ \thread{1}:\body{1}'
		\qquad\qquad
		\langle \Environment, \Abort \rangle ~ \thread{2}:\body{2}
			\xrightarrow[~~\Input~~]{\bot} 
		\langle \Environment'', \Abort'' \rangle ~ \thread{2}:\body{2}'
	}{
		\langle \Environment, \Abort \rangle ~ \thread{0}:\mathtt{par}(\thread{1}:\body{1},~\thread{2}:\body{2})
			\xrightarrow[~~\Input~~]{\bot} 
		\langle \Environment^A, \Abort^A \rangle ~ \thread{0}:\mathtt{par}(\thread{1}:\body{1}',~\thread{2}:\body{2}')
	}
\end{equation*}
$\Environment^A$ and $\Abort^A$ are the \emph{aggregated}
environment and preemption statuses, respectively, and are required
for the following reason. Threads \thread{1} and \thread{2}
always modify the starting environment $\Environment$ in a mutually exclusive
manner. Indeed, the (\ref{forec:assign-shared}) rule only allows a thread 
to access its own copies of shared variables and the (\ref{forec:assign-private}) 
rule only allows a thread to access its own private variables. 
This means that thread \thread{1}'s new program environment $\Environment'$ 
contains the old variables of thread \thread{2} and 
\thread{2}'s nested child threads, and vice versa for $\Environment''$.
Thus, variables that changed in $\Environment'$ or $\Environment''$ 
are aggregated to form $\Environment^A$ by taking the union of the 
changes in $\Environment'$ 
(i.e., $\Environment' \setminus (\Environment' \cap \Environment)$) 
and in
$\Environment''$ (i.e., $\Environment'' \setminus (\Environment'' \cap \Environment)$) 
with the remaining unchanged variables
(i.e., $\Environment' \cap \Environment''$). Note that intersecting 
two environments, e.g., $\Environment' \cap \Environment''$, produces 
a new environment containing the variables that have the same values and 
statuses in $\Environment'$ and $\Environment''$. 
Thus, $\Environment^A = \left ( \Environment' \setminus (\Environment' \cap \Environment) \right ) \cup \left ( \Environment'' \setminus (\Environment'' \cap \Environment) \right ) \cup \left ( \Environment' \cap \Environment'' \right )$.
Similarly, the preemption statuses that changed in $\Abort'$
and $\Abort''$ are aggregated to form 
$\Abort^A = \left ( \Abort' \setminus (\Abort' \cap \Abort) \right ) \cup \left ( \Abort'' \setminus (\Abort'' \cap \Abort) \right ) \cup \left ( \Abort' \cap \Abort'' \right )$.
In Esterel, such aggregation is not required because signals 
are broadcasted instantaneously among all threads.

If a child thread can complete its local tick, by pausing or 
terminating, then it will wait for its sibling to complete its local tick.
The waiting is captured by stopping the child thread from taking its transition:
\begin{equation*}
	\tag{par-2}
	\label{forec:par-2}
	\frac{
		\langle \Environment, \Abort \rangle ~ \thread{1}:\body{1}
			\xrightarrow[~~\Input~~]{k \in \lbrace 0, 1 \rbrace} 
		\langle \Environment', \Abort' \rangle ~ \thread{1}:\body{1}'
		\qquad\qquad
		\langle \Environment, \Abort \rangle ~ \thread{2}:\body{2}
			\xrightarrow[~~\Input~~]{\bot} 
		\langle \Environment'', \Abort'' \rangle ~ \thread{2}:\body{2}'
	}{
		\langle \Environment, \Abort \rangle ~ \thread{0}:\mathtt{par}(\thread{1}:\body{1},~\thread{2}:\body{2})
			\xrightarrow[~~\Input~~]{\bot} 
		\langle \Environment'', \Abort'' \rangle ~ \thread{0}:\mathtt{par}(\thread{1}:\body{1},~\thread{2}:\body{2}')
	}
\end{equation*}
\begin{equation*}
	\tag{par-3}
	\label{forec:par-3}
	\frac{
		\langle \Environment, \Abort \rangle ~ \thread{1}:\body{1}
			\xrightarrow[~~\Input~~]{\bot} 
		\langle \Environment', \Abort' \rangle ~ \thread{1}:\body{1}'
		\qquad\qquad
		\langle \Environment, \Abort \rangle ~ \thread{2}:\body{2}
			\xrightarrow[~~\Input~~]{k \in \lbrace 0, 1 \rbrace} 
		\langle \Environment'', \Abort'' \rangle ~ \thread{2}:\body{2}'
	}{
		\langle \Environment, \Abort \rangle ~ \thread{0}:\mathtt{par}(\thread{1}:\body{1},~\thread{2}:\body{2})
			\xrightarrow[~~\Input~~]{\bot} 
		\langle \Environment', \Abort' \rangle ~ \thread{0}:\mathtt{par}(\thread{1}:\body{1}',~\thread{2}:\body{2})
	}
\end{equation*}
The \verb$par$ pauses if both of its child threads pause. 
The changes made to \Environment{} and \Abort{} are
aggregated into $\Environment^A$ and $\Abort^A$,
respectively, as defined for the (\ref{forec:par-1}) rule.
The copies of shared variables from the child threads are
combined and assigned to their parent thread, thanks to the
semantic function $\Combine$: 
\begin{equation*}
	\tag{par-4}
	\label{forec:par-4}
	\frac{
		\langle \Environment, \Abort \rangle ~ \thread{1}:\body{1}
			\xrightarrow[~~\Input~~]{1} 
		\langle \Environment', \Abort' \rangle ~ \thread{1}:\body{1}'
		\qquad\qquad
		\langle \Environment, \Abort \rangle ~ \thread{2}:\body{2}
			\xrightarrow[~~\Input~~]{1} 
		\langle \Environment'', \Abort'' \rangle ~ \thread{2}:\body{2}'
	}{
		\langle \Environment, \Abort \rangle ~ \thread{0}:\mathtt{par}(\thread{1}:\body{1},~\thread{2}:\body{2})
			\xrightarrow[~~\Input~~]{1} 
		\langle \Combine(\Environment^A, \thread{1}, \thread{2}, \thread{0}), \Abort^A \rangle ~ \thread{0}:\mathtt{par}(\thread{1}:\body{1}',~\thread{2}:\body{2}')
	}
\end{equation*}
Otherwise, the \verb$par$ terminates if both of its child threads terminate. 
The completion code is $\bot$ because the parent thread \thread{0} resumes its execution. The \verb$par$
rewrites into the \verb$copy$ statement because it may be the start of the parent
thread's local tick\footnotemark[5]: 
\footnotetext[5]{The \texttt{par} statement may have paused. In this case, when the \texttt{par} 
terminates, the parent thread \thread{0} will start its local tick.}
\begin{equation*}
	\tag{par-5}
	\label{forec:par-5}
	\frac{
		\langle \Environment, \Abort \rangle ~ \thread{1}:\body{1}
			\xrightarrow[~~\Input~~]{0} 
		\langle \Environment', \Abort' \rangle ~ \thread{1}:
		\qquad\qquad
		\langle \Environment, \Abort \rangle ~ \thread{2}:\body{2}
			\xrightarrow[~~\Input~~]{0} 
		\langle \Environment'', \Abort'' \rangle ~ \thread{2}:
	}{
		\langle \Environment, \Abort \rangle ~ \thread{0}:\mathtt{par}(\thread{1}:\body{1},~\thread{2}:\body{2})
			\xrightarrow[~~\Input~~]{\bot} 
		\langle \Combine(\Environment^A, \thread{1}, \thread{2}, \thread{0}), \Abort^A \rangle ~ \thread{0}:\mathtt{copy}
	}
\end{equation*}
If only one child thread terminates while the other pauses, then the terminated 
child thread rewrites into the \verb$nop$ statement and the \verb$par$ pauses:
% We do not rewrite the parent thread into the non-terminated child thread 
% because the shared variables that the remaining child thread needs may be
% different from those needed by its parent thread.
\begin{equation*}
	\tag{par-6}
	\label{forec:par-6}
	\frac{
		\langle \Environment, \Abort \rangle ~ \thread{1}:\body{1}
			\xrightarrow[~~\Input~~]{0} 
		\langle \Environment', \Abort' \rangle ~ \thread{1}:
		\qquad\qquad
		\langle \Environment, \Abort \rangle ~ \thread{2}:\body{2}
			\xrightarrow[~~\Input~~]{1} 
		\langle \Environment'', \Abort'' \rangle ~ \thread{2}:\body{2}'
	}{
		\langle \Environment, \Abort \rangle ~ \thread{0}:\mathtt{par}(\thread{1}:\body{1},~\thread{2}:\body{2})
			\xrightarrow[~~\Input~~]{1} 
		\langle \Combine(\Environment^A, \thread{1}, \thread{2}, \thread{0}), \Abort^A \rangle ~ \thread{0}:\mathtt{par}(\thread{1}:\mathtt{nop},~\thread{2}:\body{2}')
	}
\end{equation*}
\begin{equation*}
	\tag{par-7}
	\label{forec:par-7}
	\frac{
		\langle \Environment, \Abort \rangle ~ \thread{1}:\body{1}
			\xrightarrow[~~\Input~~]{1} 
		\langle \Environment', \Abort' \rangle ~ \thread{1}:\body{1}'
		\qquad\qquad
		\langle \Environment, \Abort \rangle ~ \thread{2}:\body{2}
			\xrightarrow[~~\Input~~]{0} 
		\langle \Environment'', \Abort'' \rangle ~ \thread{2}:
	}{
		\langle \Environment, \Abort \rangle ~ \thread{0}:\mathtt{par}(\thread{1}:\body{1},~\thread{2}:\body{2})
			\xrightarrow[~~\Input~~]{1} 
		\langle \Combine(\Environment^A, \thread{1}, \thread{2}, \thread{0}), \Abort^A \rangle ~ \thread{0}:\mathtt{par}(\thread{1}:\body{1}',~\thread{2}:\mathtt{nop})
	}
\end{equation*}


\subsubsection{Tick Completion}
A tick completes if the \verb$main$ thread pauses or terminates. 
If the \verb$main$ thread is executing a \verb$par$ statement, 
then a tick completes when all its child threads and nested 
child threads have paused or terminated. 
The shared variables are resynchronized (from $\Environment'$ to $\Environment''$), 
the preemption statuses are reevaluated (from $\Abort'$ to $\Abort''$), the outputs 
are emitted, and the inputs are resampled:
\begin{equation*}
	\tag{tick}
	\label{forec:tick}
	\frac{
			\langle \Environment, \Abort \rangle ~ main:\body{}
				\xrightarrow[~~\Input~~]{k \in \lbrace 0, 1 \rbrace} 
			\langle \Environment', \Abort' \rangle ~ main:\body{}'
		}{
			\langle \Environment, \Abort \rangle ~ main:\body{}
				\xrightarrow[~~\Input~~]{k} 
			\langle \Environment'', \Abort'' \rangle ~ main:\body{}'
		}
\end{equation*}
The rules for the \verb$par$ statement ensures that, 
when the tick completes,
\verb$main$'s store in $\Environment'$ has the combined 
values from all its child threads. The shared 
variables\footnotemark[6] are 
resynchronized by assigning the combined values from 
$\Environment'[main]$ to their corresponding 
shared variables in the global store $\Environment'[\Global]$. 
The \verb$main$'s store is then removed from 
$\Environment'$. Thus, for all \var{} in $\Environment'[main]$, we have 
$\Environment'' = \Environment'[\Global][\var \gets (\Environment'[main][\var].\val, \texttt{pre})] \setminus \{main\}$.
All the preemption statuses are updated by evaluating their preemption 
conditions with the resynchronized shared variables in 
$\Environment''[\Global]$. Thus, for all \abort{} in $\Abort'$, 
we have $\Abort'' = \Abort'[\abort \gets \Eval(\Environment'', \Input, \Global, \GetExp(a))]$.

\footnotetext[6]{Recall from
Section~\ref{sec:forec_semantics:notation} that
\Environment{} maps the global and thread scopes to their
own store of variables, $\Environment: \Id \hookrightarrow \Store$. 
Variables are mapped to a value and status,
$\Store: \Var \hookrightarrow (\Val, \Sts)$. In $\Environment[\Global]$,
shared variables have the status \texttt{pre}. The notation 
$\Environment[\thread{}][\var]$ looks up the value and 
status $(\val, \sts)$ of thread \thread{}'s copy of \var{}.}


%-----------------------------------------------------------------------------

%\subsection{Comparison with Concurrent Revisions}
%SPC-MoC subsumes the idea of concurrent revisions while offering extra flexibility.
%Similarities with concurrent revisions:
%\begin{itemize}
%	\item Assignment to variables: Classic data statements.
%	\item Classic control statements: Conditional statements.
%	\item Forking and joining of threads: Concurrent revisions is more expressive
%		  (Synchronous versus Asynchronous, Static versus Dynamic).
%\end{itemize}
%Dissimilarities with concurrent revisions:
%\begin{itemize}
%	\item Blocking semantics of fork and join.
%	\item Combining of revisions is only based on value.
%	\item Preemptions.
%	\item Synchronisation during thread executions (the butterfly communication pattern not 
%		  supported by concurrent revisions).
%\end{itemize}
%
%Creating equivalent Concurrent Revision programs in ForeC (ForeC is a superset of 
%concurrent revisions?):
%\begin{itemize}
%	\item Only use data, control, and fork/join statements.
%	\item How to extend the proofs of determinacy to preemptions and ticks?
%	\begin{itemize}
%		\item Preemptions: Preemption of executions are defined precisely by the control-flow.
%			  It is never the case that you can choose between taking a preemption and performing
%			  another action (deterministic execution semantics).
%		\item Ticks: All threads have paused and cannot continue executing. Therefore, 
%			  all copies of a shared variable can be safely combined.
%	\end{itemize}
%\end{itemize}
