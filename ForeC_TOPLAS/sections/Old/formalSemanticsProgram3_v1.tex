\subsubsection{Program 3}
\begin{lstlisting}[style=snippet]
int plus (int a, int b) {return a+b;}
shared int s=0 combine with plus;
void main(void) {
  s=s+1;
  par( {s=s+2; pause; s=s+3}, {s=s*2; pause;} );
}
\end{lstlisting}
This program is structurally translated to:
\begin{lstlisting}[style=snippet]
int plus (int a, int b) {return a+b;}
shared int s=0 combine with plus;
void main(void) {
  s=s+1;
  par( t1:{s=s+2; pause; s=s+3}, t2:{s=s*2; pause;} );
}
\end{lstlisting}
Initially, the set of \verb$abort$ identifiers \Abort{}
is empty. The program's environment \Environment{} and its
derivatives are defined in Table~\ref{figure:forec_program_3}.
\newline

\begin{figure}
	\centering
	$$\begin{array}{l l l l l l l}
		 \Environment	&=& \lbrace \Global \to \lbrace s \to 0 \rbrace \rbrace																		\\
		 \Environment^1	&=& \lbrace \Global \to \lbrace s \to 0 \rbrace, main \to \lbrace s \to 1 \rbrace \rbrace									\\
		 \Environment^2	&=& \lbrace \Global \to \lbrace s \to 0 \rbrace, main \to \lbrace s \to 1 \rbrace, t1 \to \lbrace s \to 3 \rbrace \rbrace	\\
		 \Environment^3	&=& \lbrace \Global \to \lbrace s \to 0 \rbrace, main \to \lbrace s \to 1 \rbrace, t2 \to \lbrace s \to 2 \rbrace \rbrace	\\
		 \Environment^4	&=& \lbrace \Global \to \lbrace s \to 0 \rbrace, main \to \lbrace s \to 1 \rbrace, t1 \to \lbrace s \to 3 \rbrace, t2 \to \lbrace s \to 2 \rbrace \rbrace	\\
		 \Environment^5	&=& \lbrace \Global \to \lbrace s \to 0 \rbrace, main \to \lbrace s \to 5 \rbrace \rbrace									\\
		 \Environment^6	&=& \lbrace \Global \to \lbrace s \to 5 \rbrace \rbrace																		\\
		 \Environment^7	&=& \lbrace \Global \to \lbrace s \to 5 \rbrace, t1 \to \lbrace s \to 8 \rbrace \rbrace										\\
		 \Environment^8	&=& \lbrace \Global \to \lbrace s \to 5 \rbrace, main \to \lbrace s \to 8 \rbrace \rbrace									\\
		 \Environment^9	&=& \lbrace \Global \to \lbrace s \to 8 \rbrace \rbrace																		\\
	\end{array}$$
	
	\caption{Definition of the initial program environment and its derivatives for Program 3.}
	\label{figure:forec_program_3}
\end{figure}

\noindent
Step 1: Apply the (\ref{forec:seq-left}) and (\ref{forec:assign-shared}) rules. 
\begin{equation*}
	\frac{
		\dfrac{
				x \in \Shared(\Environment[\Global])
			}{
				\langle \Environment, \Abort \rangle ~ \mathtt{main: s=s+1}
					\xrightarrow[~~\Input~~]{\bot} 
				\langle \Environment^1, \Abort \rangle ~ \mathtt{main: nop}
			}
		}{
			\begin{array}{l}
				\langle \Environment, \Abort \rangle ~ \mathtt{main: s=s+1;~par( t1:\{s=s+2;}	\\
				\mathtt{pause; s=s+3;\}, t2:\{s=s*2; pause;\} );}								\\
			\end{array}
				\xrightarrow[~~\Input~~]{\bot} 
			\begin{array}{l}
				\langle \Environment^1, \Abort \rangle ~ \mathtt{main: nop;~par( t1:\{s=s+2;}	\\
				\mathtt{pause; s=s+3;\}, t2:\{s=s*2; pause;\} );}								\\
			\end{array}
		}
\end{equation*}

\noindent
Step 2: Apply the (\ref{forec:seq-right}) and (\ref{forec:nop}) rules. 
\begin{equation*}
	\frac{
			\langle \Environment^1, \Abort \rangle ~ \mathtt{main: nop}
				\xrightarrow[~~\Input~~]{0} 
			\langle \Environment^1, \Abort \rangle ~ \mathtt{main: }
		}{
			\begin{array}{l}
				\langle \Environment^1, \Abort \rangle ~ \mathtt{main: nop;~par( t1:\{s=s+2;}	\\
				\mathtt{pause; s=s+3;\}, t2:\{s=s*2; pause;\} );}								\\
			\end{array}
				\xrightarrow[~~\Input~~]{\bot} 
			\begin{array}{l}
				\langle \Environment^1, \Abort \rangle ~ \mathtt{main: par( t1:\{s=s+2;}		\\
				\mathtt{pause; s=s+3;\}, t2:\{s=s*2; pause;\} );}								\\
			\end{array}
		}
\end{equation*}

\noindent
Step 3: Apply the (\ref{forec:par-1}) rule. Additionally, apply 
the (\ref{forec:seq-left}) and (\ref{forec:assign-shared})
rules to each thread.
\begin{equation*}
	\frac{
		\dfrac{
			\dfrac{
					s \in \Shared(\Environment[\Global])
				}{
					\langle \Environment^1, \Abort \rangle ~ \mathtt{t1: s=s+2}
						\xrightarrow[~~\Input~~]{\bot} 
					\langle \Environment^2, \Abort \rangle ~ \mathtt{t1: nop}
				}
			}{
				\begin{array}{l}
					\langle \Environment^1, \Abort \rangle ~ \mathtt{t1: s=s+2;}	\\
					\mathtt{pause; s=s+3}											\\
				\end{array}
					\xrightarrow[~~\Input~~]{\bot} 
				\begin{array}{l}
					\langle \Environment^2, \Abort \rangle ~ \mathtt{t1: nop;}		\\
					\mathtt{pause; s=s+3}											\\
				\end{array}
			}
			\quad
		\dfrac{
			\dfrac{
					s \in \Shared(\Environment[\Global])
				}{
					\langle \Environment^1, \Abort \rangle ~ \mathtt{t2: s=s*2}
						\xrightarrow[~~\Input~~]{\bot} 
					\langle \Environment^3, \Abort \rangle ~ \mathtt{t2: nop}
				}
			}{
				\begin{array}{l}
					\langle \Environment^1, \Abort \rangle ~ \mathtt{t2: }		\\
					\mathtt{s=s*2; pause}										\\
				\end{array}
					\xrightarrow[~~\Input~~]{\bot} 
				\begin{array}{l}
					\langle \Environment^3, \Abort \rangle ~ \mathtt{t2: }		\\
					\mathtt{nop; pause}											\\
				\end{array}
			}
		}{
			\begin{array}{l}
				\langle \Environment^1, \Abort \rangle ~ \mathtt{main: par( t1:\{s=s+2;}	\\
				\mathtt{pause; s=s+3;\}, t2:\{s=s*2; pause;\} );}							\\
			\end{array}
				\xrightarrow[~~\Input~~]{\bot} 
			\begin{array}{l}
				\langle \Environment^4, \Abort \rangle ~ \mathtt{main: par( t1:\{nop;}		\\
				\mathtt{pause; s=s+3;\}, t2:\{nop; pause;\} );}								\\
			\end{array}
		}
\end{equation*}

\noindent
Step 4: Apply the (\ref{forec:par-1}) rule. Additionally, apply 
the (\ref{forec:seq-right}) and (\ref{forec:nop})
rules to each thread.
\begin{equation*}
	\frac{
		\dfrac{
				\langle \Environment^4, \Abort \rangle ~ \mathtt{t1: nop}
					\xrightarrow[~~\Input~~]{0} 
				\langle \Environment^4, \Abort \rangle ~ \mathtt{t1: }
			}{
				\begin{array}{l}
					\langle \Environment^4, \Abort \rangle ~ \mathtt{t1: nop;}	\\
					\mathtt{pause; s=s+3}										\\
				\end{array}
					\xrightarrow[~~\Input~~]{\bot} 
				\begin{array}{l}
					\langle \Environment^4, \Abort \rangle ~ \mathtt{t1: }		\\
					\mathtt{pause; s=s+3}										\\
				\end{array}
			}
			\quad
		\dfrac{
				\langle \Environment^4, \Abort \rangle ~ \mathtt{t2: nop}
					\xrightarrow[~~\Input~~]{0} 
				\langle \Environment^4, \Abort \rangle ~ \mathtt{t2: }
			}{
				\begin{array}{l}
					\langle \Environment^4, \Abort \rangle ~ \mathtt{t2: }	\\
					\mathtt{nop; pause}										\\
				\end{array}
					\xrightarrow[~~\Input~~]{\bot} 
				\begin{array}{l}
					\langle \Environment^4, \Abort \rangle ~ \mathtt{t2: }	\\
					\mathtt{pause}											\\
				\end{array}
			}
		}{
			\begin{array}{l}
				\langle \Environment^4, \Abort \rangle ~ \mathtt{main: par( t1:\{nop;}	\\
				\mathtt{pause; s=s+3;\}, t2:\{nop; pause;\} );}							\\
			\end{array}
				\xrightarrow[~~\Input~~]{\bot} 
			\begin{array}{l}
				\langle \Environment^4, \Abort \rangle ~ \mathtt{main: par( t1:\{pause;}\\
				\mathtt{s=s+3;\}, t2:\{pause;\} );}										\\
			\end{array}
		}
\end{equation*}

\noindent
Step 5: Apply the (\ref{forec:global-tick}) and (\ref{forec:par-4}) rules.
Additionally, apply the (\ref{forec:seq-left}) and (\ref{forec:pause}) rules
to the first thread, and the (\ref{forec:pause}) rule to the second thread.
The program completes a global tick.
\begin{equation*}
	\frac{
		\dfrac{
			\dfrac{
					\langle \Environment^4, \Abort \rangle ~ \mathtt{t1: pause}
						\xrightarrow[~~\Input~~]{1} 
					\langle \Environment^4, \Abort \rangle ~ \mathtt{t1: nop}
				}{
					\begin{array}{l}
						\langle \Environment^4, \Abort \rangle ~ \mathtt{t1: }	\\
						\mathtt{pause; s=s+3}										\\
					\end{array}
						\xrightarrow[~~\Input~~]{1} 
					\begin{array}{l}
						\langle \Environment^4, \Abort \rangle ~ \mathtt{t1: }		\\
						\mathtt{nop; s=s+3}										\\
					\end{array}
				}
				\quad
				\twolines{
					\twolines{
							~							
						}{
							~
						}
					}{
						\langle \Environment^4, \Abort \rangle ~ \mathtt{t2: pause}
							\xrightarrow[~~\Input~~]{1} 
						\langle \Environment^4, \Abort \rangle ~ \mathtt{t2: nop}
					}
			}{
				\twolines{
					~
					}{
						\mathtt{main} = main
					}
				\quad
				\begin{array}{l}
					\langle \Environment^4, \Abort \rangle ~ \mathtt{main: par( t1:\{pause;}\\
					\mathtt{s=s+3\}, t2:\{pause;\} );}										\\
				\end{array}
					\xrightarrow[~~\Input~~]{1} 
				\begin{array}{l}
					\langle \Environment^5, \Abort \rangle ~ \mathtt{main: par( t1:\{nop;}	\\
					\mathtt{s=s+3;\}, t2:\{nop;\} );}										\\
				\end{array}
			}
		}{
			\begin{array}{l}
				\langle \Environment^4, \Abort \rangle ~ \mathtt{main: par( t1:\{pause;}\\
				\mathtt{s=s+3\}, t2:\{pause;\} );}										\\
			\end{array}
				\xrightarrow[~~\Input~~]{1} 
			\begin{array}{l}
				\langle \Environment^6, \Abort \rangle ~ \mathtt{main: par( t1:\{nop;}	\\
				\mathtt{s=s+3;\}, t2:\{nop;\} );}										\\
			\end{array}
		}
\end{equation*}

\noindent
Step 6: Apply the (\ref{forec:par-3}) rule. Additionally, apply the (\ref{forec:seq-right}) 
and (\ref{forec:nop}) rules to the first thread, and the (\ref{forec:nop}) rule to the second
thread.
\begin{equation*}
	\frac{
		\dfrac{
				\langle \Environment^6, \Abort \rangle ~ \mathtt{t1: nop}
					\xrightarrow[~~\Input~~]{0} 
				\langle \Environment^6, \Abort \rangle ~ \mathtt{t1: }
			}{
				\begin{array}{l}
					\langle \Environment^6, \Abort \rangle ~ \mathtt{t1: }	\\
					\mathtt{nop; s=s+3}										\\
				\end{array}
					\xrightarrow[~~\Input~~]{\bot} 
				\begin{array}{l}
					\langle \Environment^6, \Abort \rangle ~ \mathtt{t1: }	\\
					\mathtt{s=s+3}											\\
				\end{array}
			}
			\quad
			\twolines{
				\twolines{
						~
					}{
						~
					}
				}{
					\langle \Environment^6, \Abort \rangle ~ \mathtt{t2: nop}
						\xrightarrow[~~\Input~~]{0} 
					\langle \Environment^6, \Abort \rangle ~ \mathtt{t2: }
				}
		}{
			\begin{array}{l}
				\langle \Environment^6, \Abort \rangle ~ \mathtt{main: par( t1:\{nop;}	\\
				\mathtt{s=s+3\}, t2:\{nop;\} );}										\\
			\end{array}
				\xrightarrow[~~\Input~~]{\bot} 
			\begin{array}{l}
				\langle \Environment^6, \Abort \rangle ~ \mathtt{main: par( t1: }		\\
				\mathtt{\{s=s+3;\}, t2:\{nop;\} );}										\\
			\end{array}
		}
\end{equation*}

\noindent
Step 7: Apply the (\ref{forec:par-3}) rule. Additionally, apply the (\ref{forec:assign-shared}) 
rule to the first thread and the (\ref{forec:nop}) rule to the second
thread.
\begin{equation*}
	\frac{
		\dfrac{
				s \in \Shared(\Environment[\Global])
			}{
				\langle \Environment^6, \Abort \rangle ~ \mathtt{t1: s=s+3}
					\xrightarrow[~~\Input~~]{\bot} 
				\langle \Environment^7, \Abort \rangle ~ \mathtt{t1: nop}
			}
			\quad
			\twolines{
					~
				}{
					\langle \Environment^6, \Abort \rangle ~ \mathtt{t2: nop}
						\xrightarrow[~~\Input~~]{0} 
					\langle \Environment^6, \Abort \rangle ~ \mathtt{t2: }
				}
		}{
			\begin{array}{l}
				\langle \Environment^6, \Abort \rangle ~ \mathtt{main: par( t1: \{s=s+3;\},}	\\
				\mathtt{ t2:\{nop;\} );}														\\
			\end{array}
				\xrightarrow[~~\Input~~]{\bot} 
			\langle \Environment^7, \Abort \rangle ~ \mathtt{main: par( t1: \{nop;\}, t2:\{nop;\} );}
		}
\end{equation*}

\noindent
Step 8: Apply the (\ref{forec:par-5}) rule. Additionally, apply the (\ref{forec:nop}) 
rule to both threads.
\begin{equation*}
	\frac{
			\langle \Environment^7, \Abort \rangle ~ \mathtt{t1: nop}
				\xrightarrow[~~\Input~~]{0} 
			\langle \Environment^7, \Abort \rangle ~ \mathtt{t1: }
			\qquad
			\qquad
			\langle \Environment^7, \Abort \rangle ~ \mathtt{t2: nop}
				\xrightarrow[~~\Input~~]{0} 
			\langle \Environment^7, \Abort \rangle ~ \mathtt{t2: }
		}{
			\langle \Environment^7, \Abort \rangle ~ \mathtt{main: par( t1: \{nop;\}, t2:\{nop;\} );}
				\xrightarrow[~~\Input~~]{\bot} 
			\langle \Environment^8, \Abort \rangle ~ \mathtt{main: nop;}
		}
\end{equation*}

\noindent
Step 9: Apply the (\ref{forec:global-tick}) rule.
The program terminates.
\begin{equation*}
	\frac{
			\mathtt{main} = main
			\quad
			\langle \Environment^8, \Abort \rangle ~ \mathtt{main: nop}
				\xrightarrow[~~\Input~~]{0} 
			\langle \Environment^8, \Abort \rangle ~ \mathtt{main: }
		}{
			\langle \Environment^8, \Abort \rangle ~ \mathtt{main: nop;}
				\xrightarrow[~~\Input~~]{0} 
			\langle \Environment^9, \Abort \rangle ~ \mathtt{main: }
		}
\end{equation*}

