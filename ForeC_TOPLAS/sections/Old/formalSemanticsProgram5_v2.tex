\subsubsection{Program 5}
\begin{lstlisting}[style=snippet]
int x=1;
void main(void) {
  par(
    {pause;}, 
    {par({x++; pause;},{pause;});}
  );
}
\end{lstlisting}
This program is structurally translated to:
\begin{lstlisting}[style=snippet]
int x=1;
void main(void) {
  par(
    t1:{pause;}, 
    t2:{par(t3:{x++; pause;},t4:{pause;});}
  );
}
\end{lstlisting}
The predicates $\Shared(main)$, $\Shared(t1)$, $\Shared(t2)$, $\Shared(t3)$,
$\Shared(t4)$ and $\Shared(\Global)$ are all $\emptyset$.
Initially, the set of preemption statuses \Abort{} is $\emptyset$.
The program's environment \Environment{} 
and its derivatives are defined in Figure~\ref{figure:forec_program_5}.
\newline

\begin{figure}
	\centering
	$$\begin{array}{l l l}
		\Environment		&=& \left \lbrace
									\Global \to \lbrace x \to (1, \mathtt{pre}) \rbrace
								\right \rbrace	\\
		\Environment^1		&=& \left \lbrace
									\Global \to \lbrace x \to (2, \mathtt{pre}) \rbrace
								\right \rbrace	\\
	\end{array}$$
	
	\caption{Definition of the initial program state and its derivatives for Program 5.}
	\label{figure:forec_program_5}
\end{figure}

\noindent
Step 1: Apply the (\ref{forec:par-2}) rule. Additionally, apply the (\ref{forec:pause}) 
rule to the first thread and (\ref{forec:par-3}) rule to the second thread. Additionally
apply the (\ref{forec:seq-right}) and (\ref{forec:assign-private}) rules to the first 
nested thread and the (\ref{forec:pause}) rule to the second nested thread.
{\footnotesize
\begin{equation*}
	\frac{
		\twolines{
			\twolines{
					~
				}{
					~
				}
			}{
				\begin{array}{l}
					\langle \Environment, \Abort \rangle ~ \mathtt{t1:}							\\
					\mathtt{pause}																\\
				\end{array}
					\xrightarrow[~~\Input~~]{1} 
				\begin{array}{l}
					\langle \Environment, \Abort \rangle ~ \mathtt{t1:}							\\
					\mathtt{copy}																\\
				\end{array}
			}
			\qquad
		\dfrac{
			\dfrac{
				\dfrac{
						x \notin \lbrace ~ \rbrace
					}{
						\langle \Environment, \Abort \rangle ~ \mathtt{t3: x++}
							\xrightarrow[~~\Input~~]{0} 
						\langle \Environment^1, \Abort \rangle ~ \mathtt{t3: nop}
					}
				}{
					\begin{array}{l}
						\langle \Environment, \Abort \rangle ~ \mathtt{t3:}					\\
						\mathtt{x++; pause}													\\
					\end{array}
						\xrightarrow[~~\Input~~]{\bot} 
					\langle \Environment^1, \Abort \rangle ~ \mathtt{t3: pause}
				}
				\qquad
			\twolines{
				~
				}{
					\begin{array}{l}
						\langle \Environment, \Abort \rangle ~ \mathtt{t4:}						\\
						\mathtt{pause}															\\
					\end{array}
						\xrightarrow[~~\Input~~]{1} 
					\begin{array}{l}
						\langle \Environment, \Abort \rangle ~ \mathtt{t4:}						\\
						\mathtt{copy}															\\
					\end{array}
				}
			}{
				\begin{array}{l}
					\langle \Environment, \Abort \rangle ~ \mathtt{t2: par(t3:\{x++; pause;\},}	\\
					\mathtt{t4:\{pause;\})}														\\
				\end{array}
					\xrightarrow[~~\Input~~]{\bot} 
				\begin{array}{l}
					\langle \Environment^1, \Abort \rangle ~ \mathtt{t2: par(t3:\{pause;\},}	\\
					\mathtt{t4:\{pause;\})}														\\
				\end{array}
			}
		}{
			\begin{array}{l}
				\langle \Environment, \Abort \rangle ~ \mathtt{main: par(t1:\{pause;\},}		\\
				\mathtt{t2:\{par(t3:\{x++; pause;\},t4:\{pause;\});\})}							\\
			\end{array}
				\xrightarrow[~~\Input~~]{\bot} 
			\begin{array}{l}
				\langle \Environment^1, \Abort \rangle ~ \mathtt{main: par(t1:\{pause;\},}		\\
				\mathtt{t2:\{par(t3:\{pause;\},t4:\{pause;\});\})}								\\
			\end{array}
		}
\end{equation*}
}

\noindent
Step 2: Apply the (\ref{forec:par-4}) rule. Additionally, apply the (\ref{forec:pause}) 
rule to the first thread and (\ref{forec:par-4}) rule to the second thread. Additionally
apply the (\ref{forec:pause}) rule to both nested threads.
{\footnotesize
\begin{equation*}
	\frac{
		\twolines{
			\twolines{
					~
				}{
					~
				}
			}{
				\begin{array}{l}
					\langle \Environment^1, \Abort \rangle ~ \mathtt{t1:}							\\
					\mathtt{pause}																	\\
				\end{array}
					\xrightarrow[~~\Input~~]{1} 
				\begin{array}{l}
					\langle \Environment^1, \Abort \rangle ~ \mathtt{t1:}							\\
					\mathtt{copy}																	\\
				\end{array}
			}
			\qquad
		\dfrac{
				\begin{array}{l}
					\langle \Environment^1, \Abort \rangle ~ \mathtt{t3:}						\\
					\mathtt{pause}																\\
				\end{array}
					\xrightarrow[~~\Input~~]{1} 
				\begin{array}{l}
					\langle \Environment^1, \Abort \rangle ~ \mathtt{t3:}						\\
					\mathtt{copy}																\\
				\end{array}
				\qquad
				\begin{array}{l}
					\langle \Environment^1, \Abort \rangle ~ \mathtt{t4:}						\\
					\mathtt{pause}																\\
				\end{array}
					\xrightarrow[~~\Input~~]{1} 
				\begin{array}{l}
					\langle \Environment^1, \Abort \rangle ~ \mathtt{t4:}						\\
					\mathtt{copy}																\\
				\end{array}
			}{
				\begin{array}{l}
					\langle \Environment^1, \Abort \rangle ~ \mathtt{t2: par(t3:\{pause;\},}	\\
					\mathtt{t4:\{pause;\})}														\\
				\end{array}
					\xrightarrow[~~\Input~~]{1} 
				\begin{array}{l}
					\langle \Environment^1, \Abort \rangle ~ \mathtt{t2: par(t3:\{copy;\},}		\\
					\mathtt{t4:\{copy;\})}														\\
				\end{array}
			}
		}{
			\begin{array}{l}
				\langle \Environment^1, \Abort \rangle ~ \mathtt{main: par(t1:\{pause;\},}		\\
				\mathtt{t2:\{par(t3:\{pause;\},t4:\{pause;\});\})}								\\
			\end{array}
				\xrightarrow[~~\Input~~]{\bot} 
			\begin{array}{l}
				\langle \Environment^1, \Abort \rangle ~ \mathtt{main: par(t1:\{copy;\},}		\\
				\mathtt{t2:\{par(t3:\{copy;\},t4:\{copy;\});\})}								\\
			\end{array}
		}
\end{equation*}
}

\noindent
Step 3: Apply the (\ref{forec:par-2}) rule. Additionally, apply the (\ref{forec:copy}) 
rule to the first thread and (\ref{forec:par-5}) rule to the second thread. Additionally
apply the (\ref{forec:copy}) rule to both nested threads.
{\footnotesize
\begin{equation*}
	\frac{
	\twolines{
			~
		}{
			\langle \Environment^1, \Abort \rangle ~ \mathtt{t1: copy}
				\xrightarrow[~~\Input~~]{0} 
			\langle \Environment^1, \Abort \rangle ~ \mathtt{t1: nop}
			\qquad
		}
		\dfrac{
				\langle \Environment^1, \Abort \rangle ~ \mathtt{t3: copy}
					\xrightarrow[~~\Input~~]{0} 
				\langle \Environment^1, \Abort \rangle ~ \mathtt{t3: nop}
				\qquad
				\langle \Environment^1, \Abort \rangle ~ \mathtt{t4: copy}
					\xrightarrow[~~\Input~~]{0} 
				\langle \Environment^1, \Abort \rangle ~ \mathtt{t4: nop}
			}{
				\langle \Environment^1, \Abort \rangle ~ \mathtt{t2: par(t3:\{copy;\},t4:\{copy;\})}
					\xrightarrow[~~\Input~~]{\bot} 
				\langle \Environment^1, \Abort \rangle ~ \mathtt{t2: copy}
			}
		}{
			\langle \Environment^1, \Abort \rangle ~ \mathtt{main: par(t1:\{copy;\},t2:\{par(t3:\{copy;\},t4:\{copy;\});\})}
				\xrightarrow[~~\Input~~]{\bot} 
			\langle \Environment^1, \Abort \rangle ~ \mathtt{main: par(t1:\{copy;\},t2:\{copy;\})}
		}
\end{equation*}
}

\noindent
Step 4: Apply the (\ref{forec:par-5}) rule. Additionally, 
apply the (\ref{forec:copy}) rule to both threads.
{\footnotesize
\begin{equation*}
	\frac{
			\langle \Environment^1, \Abort \rangle ~ \mathtt{t1: copy}
				\xrightarrow[~~\Input~~]{0} 
			\langle \Environment^1, \Abort \rangle ~ \mathtt{t1: nop}
			\qquad
			\langle \Environment^1, \Abort \rangle ~ \mathtt{t2: copy}
				\xrightarrow[~~\Input~~]{0} 
			\langle \Environment^1, \Abort \rangle ~ \mathtt{t2: nop}
		}{
			\langle \Environment^1, \Abort \rangle ~ \mathtt{main: par(t1:\{copy;\},t2:\{copy;\})}
				\xrightarrow[~~\Input~~]{\bot} 
			\langle \Environment^1, \Abort \rangle ~ \mathtt{main: copy}
		}
\end{equation*}
}

\noindent
Step 5: Apply the (\ref{forec:tick}) and (\ref{forec:copy}) rules. The program
terminates.
{\footnotesize
\begin{equation*}
	\frac{
			\langle \Environment^1, \Abort \rangle ~ \mathtt{main: copy}
				\xrightarrow[~~\Input~~]{0} 
			\langle \Environment^1, \Abort \rangle ~ \mathtt{main: nop}
		}{
			\langle \Environment^1, \Abort \rangle ~ \mathtt{main: copy}
				\xrightarrow[~~\Input~~]{0} 
			\langle \Environment^1, \Abort \rangle ~ \mathtt{main: nop}
		}
\end{equation*}
}
