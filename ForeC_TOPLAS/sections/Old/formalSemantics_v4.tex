\section{Semantics}
\label{sec:formalSemantics}

This section presents the semantics of ForeC as rewrite rules,
in the style of Structural Operation Semantics~\cite{semantics_sos}. 
The semantics is inspired by that of other synchronous programming
languages (Esterel~\cite{timed_compiling_esterel} and \pretc{}~\cite{pret_pretc} in particular).
The semantics is defined on a set of primitive (kernel) ForeC constructs
(see Table~\ref{table:forec_kernel}) from which all the ForeC 
constructs are derived from. The kernel constructs are not used for compiling
and only considers a subset of the C language: the assignment (\verb$=$)
and sequence (\verb$;$) operators and the \verb$if$ and \verb$while$ statements. 
Table~\ref{table:forec_structural_translations} shows how the
ForeC constructs (Table~\ref{table:forec_semantics}) are translated into the kernel 
constructs (Table~\ref{table:forec_kernel}). This is exemplified by the 
translation of the ForeC constructs in Figure~\ref{fig:transform_forec_refactored} 
into the kernel constructs in Figure~\ref{fig:transform_forec_kernel}. 
The translations for \verb$input$, \verb$output$, and \verb$pause$ are
straightforward. The \verb$shared$ variable is translated into
a global variable and a \verb$copy$ kernel statement is placed at
the start of every thread body. The
\verb$copy$ kernel statement is needed to define how threads
create local copies of shared variables.
The \verb$par$ statement is translated by prefixing each
thread body \body{} with a unique identifier thread \thread{} to 
distinguish the body of one thread from another. 
Traditionally, \verb$trap$s~\cite{timed_compiling_esterel} are used 
to describe \verb$abort$s and other complex preemption statements. 
In ForeC, a straightforward \verb$abort$ translation is 
possible because it is the only type of preemption statement.
Each \verb$abort$ is assigned a unique identifier $a$ and then 
translated into the \verb$status$ and \verb$abort$ kernel
statements. The \verb$status$ kernel statement is needed to
define the immediate behavior of an \verb$abort$ and it takes the
unique identifier $a$ and an expression. The expression is
0 (zero) for a non-immediate \verb$abort$, but is \expression{}
(preemption condition) for an immediate \verb$abort$. The
\verb$abort$ kernel statement takes the unique
identifier $a$ and the \verb$abort$ body~\body{}. This section
describes the assumptions on ForeC kernel programs to 
simplify the presentation of the formal semantics. The notation
of the rewrite rules, useful predicates, and semantic functions 
are described before the rewrite rules are presented.

\begin{table}[h]
	\centering
	\renewcommand{\arraystretch}{1.25}
	
	\begin{tabular}{l l}
		\bf{Kernel Construct}									& \bf{Short Description}	\\ 
		\hline
		\verb$nop$												& Empty statement			\\
		\body{}\verb$;$~\body{}									& Sequence					\\
		\var{}~\verb$=$~\expression{}							& Variable assignment		\\
		\verb$while$~(\expression{})~\body{}					& Loop						\\
		\verb$if$~(\expression{})~\body{}~\verb$else$~\body{}	& Conditional				\\
		\verb$copy$												& Copy shared variables		\\
		\verb$pause$											& Barrier synchronization	\\
		\verb$par$(\thread{}:\body{},~\thread{}:\body{})		& Fork/join parallelism		\\
		\verb$status$($a$, \expression{})						& Initial preemption status	\\
		\verb$weak$?~\verb$abort$($a$, \body{})					& Abort						\\
	\end{tabular}
	
	\caption{The ForeC kernel constructs. Here, \body{} is an arbitrary composition of kernel 
			 constructs, \var{} is a variable, \expression{} is an expression, $t$ is a thread 
			 identifier, and $a$ is an abort identifier. A question mark denotes that the 
			 preceding symbol is optional.}
	\label{table:forec_kernel}
\end{table}

\begin{table}
	\centering
	\renewcommand{\arraystretch}{1.25}
	
	\begin{tabular}{| l | p{8cm} |}
		\hline
		\bf{ForeC Construct}													& \bf{Translation to ForeC Kernel Constructs}											\\ 
		\hline
		\verb$input$ and \verb$output$											& Translated into global variables.														\\ \hline
		\verb$shared$															& Translated into global variables. The \verb$copy$ kernel statement is placed at 
																				  start of every thread body.															\\ \hline
		\verb$pause$															& \verb$pause$																			\\ \hline
		\verb$par$(\body{}, \body{})											& \verb$par$(\thread{}:\body{}, \thread{}:\body{})										\\ \hline
		\verb$weak$?~\verb$abort$~\body{}~\verb$when$~(\expression{})			& \verb$status$($a$, 0)\verb$;$ \verb$weak$?~\verb$abort$($a$, \body{})					\\ \hline
		\verb$weak$?~\verb$abort$~\body{}~\verb$when immediate$~(\expression{})	& \verb$status$($a$, \expression{})\verb$;$ \verb$weak$?~\verb$abort$($a$, \body{})		\\ \hline
	\end{tabular}
	
	\caption{Structural translations of the ForeC constructs  
			 (Table~\ref{table:forec_semantics}) to kernel constructs 
			 (Table~\ref{table:forec_kernel}).}
	\label{table:forec_structural_translations}
\end{table}

\begin{figure}
	\centering 
	\subfloat[Original program.] {
		\begin{minipage}[b]{0.48\textwidth}
			\lstinputlisting[style=full]{./code/transform_forec_original.forec}
			\label{fig:transform_forec_original}
		\end{minipage}
	}
	\hfill
	\subfloat[Transformed program.] {
		\begin{minipage}[b]{0.43\textwidth}
			\lstinputlisting[style=full]{./code/transform_forec_refactored.forec}
			\label{fig:transform_forec_refactored}
		\end{minipage}
	}
	
	\subfloat[Translated kernel program.] {
		\begin{minipage}[b]{0.63\textwidth}
			\lstinputlisting[style=full]{./code/transform_forec_kernel.forec}
			\label{fig:transform_forec_kernel}
		\end{minipage}
	}
	
	\caption{Example of transforming and translating a ForeC program into the kernel constructs.}
	\label{fig:transform_forec}
\end{figure}


\subsection{Assumptions}
\label{sec:formalSemantics_assumptions}
We make the following assumptions about ForeC programs. 
(1) All programs follow safety-critical coding practices,
as discussed in Section~\ref{sec:forecLanguage_programming}.
For example, all expressions are side-effect free, 
dynamic memory allocation is not used, and loops that can 
iterate an unbounded number of times without pausing are not used.
This assumption limits us to a deterministic subset of the 
C language.
(2) All recursive function calls or forking of threads are bounded.
This assumption prevents the unbounded 
execution of functions and threads, leading to unbounded
memory use and execution time. 
To simplify the presentation of the semantics, we assume that the 
following transformations have been performed on ForeC programs. 
(1) Inlining of functions at their call sites. The semantics can
ignore function calls.
(2) Renaming variables uniquely and hoisting their declarations 
up to the program's global scope. The semantics can ignore 
memory allocation and focus on the semantics of private and 
shared variables.
(3) Replacing pointers with the variables they reference. The 
semantics can ignore pointer analysis.
For example, the program of 
Figure~\ref{fig:transform_forec_original} can be
transformed into the equivalent program of 
Figure~\ref{fig:transform_forec_refactored}.


%-----------------------------------------------------------------------------

\subsection{Notation}
The rewrite rules have the following form in the style of 
Structural Operational Semantics~\cite{semantics_sos}:
\begin{equation*}
	\langle \StateP \rangle ~ \thread{}:\body{} 
		\xrightarrow[~~\Input~~]{k} 
	\langle \StateP' \rangle ~ \thread{}:\body{}'
\end{equation*}
This notation describes a program fragment \body{} belonging to thread \thread{},
in the program state $\langle \StateP \rangle$ and with inputs \Input{}, which
reacts and modifies the program state to $\langle \StateP' \rangle$, emits the 
completion code $k$, and becomes the new program fragment $\body{}'$. 
All the input variables are stored in \Input{}. Recall 
from Section~\ref{sec:forecLanguage_io} that all \verb$input$ variables 
must be globally declared.
Let $\thread{} \in \Thread$ be the set of all threads in the program. 
Let $\langle \StateP \rangle$ be defined as 
$\langle \Environment, \Abort \rangle$, where:
\begin{itemize}
	\item \Environment{} is an environment that maintains the valuations of all the variables and 
		  copies of shared variables in the program. Specifically, \Environment{} is a partial
		  function that maps the global scope \Global{} and threads ($\thread{} \in \Thread$) to a 
		  store (\Store{}) of variables. Let $\Id = \Thread \cup \{\Global\}$, then 
		  $\Environment: \Id \to \Store$.
		  Since all variables are globally declared (Section~\ref{sec:formalSemantics_assumptions}), 
		  $\Environment[\Global]$ stores all the output, shared, and ordinary C variables. 
		  $\Environment[\thread{}]$ stores thread \thread{}'s copies of shared variables. 
		  The store (\Store{}) is a partial function that maps variables ($\var \in \Var$) to values ($\val \in \Val$) 
		  and statuses ($\sts \in \Sts$), $\Store: \Var \to (\Val, \Sts)$. Statuses are
		  needed only to define how the combine policies affect which copies get combined.
		  The status of a variable in $\Environment[\thread{}]$ is either \verb$pre$ (previous 
		  combined value), \verb$mod$ (modified value), or \verb$cmb$ (partially combined value). 
		  The status of each global variable $\var \in \Environment[\Global]$ is always \verb$pre$.
		
		  For example, $\Environment = \left \lbrace \Global \to \lbrace s \to (1, \mathtt{pre}) \rbrace, t1 \to \lbrace s \to (3, \mathtt{mod}) \right \rbrace$
		  for a program with shared variable \verb$s$ with value 1 and thread $t1$'s modified copy of
		  \verb$s$ with value 3. We use the
		  notation, e.g., $\Environment[t1][s]$ to look up the value and status $(3, \mathtt{mod})$ of \verb$s$ in $t1$'s store.
		  We use the notations, e.g.,  $\Environment[t1][s].\val$ and $\Environment[t1][s].\sts$
		  to look up its value and status separately.
		  
	\item \Abort{} is a partial function that maps the \verb$abort$ identifiers in the program 
		  ($a \in \mathcal{A}$) to values ($\val \in \Val$) representing their preemption status, 
		  $\Abort: \mathcal{A} \to Val$. An \verb$abort$ with a non-zero value means that its
		  preemption has been triggered. 
		  
		  For example, $\Abort = \lbrace a1 \to 1, a2 \to 0 \rbrace$ 
		  for a program with \verb$abort$s $a1$ and $a2$ with the statuses 1 and 0, respectively.
		  We use the notation, e.g., $\Abort[a1]$ to look up the status of \verb$abort$ $a1$.
\end{itemize}
The completion code of a transition is encoded by $k$, where:
\begin{equation*}
	k = \begin{cases}
		0		& \text{If the transition terminates.}								\\
		1		& \text{If the transition pauses until a tick occurs.}		\\
		\bot	& \text{If the transition can continue.}
	\end{cases}
\end{equation*}


%-----------------------------------------------------------------------------

\subsection{Predicates}
To keep the ForeC semantics concise, we define predicates for returning 
statically known information about the program. 
\begin{itemize}
	\item $\Parent(\thread{})$:
		Returns the parent of thread \thread{}.
		If $\thread{} =$ \verb$main$, then ``\verb$main$'' is returned.
	\item $\Shared(\Global)$:
		Returns the set of all shared variables declared in the program.
	\item $\Shared(\thread{})$:
		Returns the set of all shared variables that the body of thread \thread{} accesses (reads or writes).
	\item $\GetCombine(\var)$:
		Returns the combine function of shared variable \var{}.
	\item $\GetPolicy(\var)$:
		Returns the combine policy of shared variable \var{}.
	\item $\GetExp(a)$:
		Returns the preemption condition \expression{} of \verb$abort$ $a$.
\end{itemize}
As an example, Figure~\ref{fig:forec_predicates} shows the use of predicates on
the program in Figure~\ref{fig:transform_forec_kernel}:
\begin{figure}[ht]
	\centering
	\renewcommand{\arraystretch}{1.25}
	\begin{tabular}{|l l l|}
		\hline
		$\Parent($\verb$main$$) =$ \verb$main$			& $\Parent($\verb$t1$$) =$ \verb$main$					& $\Parent($\verb$t2$$) =$ \verb$main$		\\
		$\Shared(\Global) = \{$\verb$x$$,$\verb$s$$\}$	& $\Shared($\verb$t1$$) = \{$\verb$x$$,$\verb$s$$\}$	& $\Shared($\verb$t2$$) = \{$\verb$x$$\}$	\\
		$\GetCombine($\verb$s$$) =$ \verb$plus$			& $\GetPolicy($\verb$s$$) =$ \verb$all$					& $\GetExp($\verb$a1$$) =$ \verb$s>10$		\\
		\hline
	\end{tabular}
	
	\caption{Example of applying predicates to the program of Figure~\ref{fig:transform_forec_kernel}.}
	\label{fig:forec_predicates}
\end{figure}


%-----------------------------------------------------------------------------

\subsection{Semantic Functions}
To keep the ForeC semantics concise, we define semantic functions 
used by the rewrite rules.

\subsubsection{\textsc{Eval}}
\label{sec:forec_Eval}
The semantic function $\Eval(\Environment, \Input, \id, \expression)$ 
follows the evaluation rules of the C-language to 
evaluate the expression \expression{} and return 
its value. During the evaluation, the value of a 
variable is retrieved using the semantic function $\GetVal(\Environment, \Input, \id, \var)$ 
described by Algorithm~\ref{algo:GetVal}. 
The inputs to the algorithm are: the program's environment 
\Environment{}, the inputs \Input{}, the identifier \id{} of the
store to try and retrieve the value from, and the variable \var{} of interest. The 
output is a value \val{}. If \var{} is an input, 
then its value is returned (line~\ref{algo:GetVal_input}). 
Otherwise, if \var{} is in \id{}'s store, then its value is returned 
(line~\ref{algo:GetVal_local}). 
Otherwise, the global value of \var{} is returned
(line~\ref{algo:GetVal_global}).

\begin{algorithm}[t]
	\begin{algorithmic}[1]
		\Function{$\GetVal$}{\Environment{}, \Input{}, \id{}, \var{}}
			\If {$\var \in \Input$}
				\State \Return $\Input[\var]$								\Comment{Return the input value for \var{}.}\label{algo:GetVal_input}
			\ElsIf {$\var \in \Environment[\id{}]$}	
					\State \Return $\Environment[\id{}][\var].\val$			\Comment{Return the value for \var{} from \id{}'s store.}\label{algo:GetVal_local}
			\Else
				\State \Return $\Environment[\Global][\var].\val$			\Comment{Return the global value for \var{}.}\label{algo:GetVal_global}
			\EndIf
		\EndFunction
	\end{algorithmic}
	
	\caption{Gets the value of a given variable.}
	\label{algo:GetVal}
\end{algorithm}

\subsubsection{\textsc{Copy}}
\label{sec:forec_Copy}
The semantic function $\Copy(\Environment, \thread{})$ creates 
copies of shared variables that thread \thread{} needs but 
does not already have. This copying behavior ensures that, 
when a \verb$par$ terminates, the combined values assigned to the 
parent thread are not overwritten. The behavior also ensures
that, when a \verb$par$ forks and joins in the same tick, 
the parent thread's existing copies are not overwritten.
This function is described by Algorithm~\ref{algo:Copy}. 
The inputs to the algorithm are: the program's environment 
\Environment{} and a thread \thread{}. The output is an updated
environment \Environment{}. The algorithm retrieves all the shared 
variables that are accessed in the thread's body (line~\ref{algo:Copy_retrieve}).
For each shared variable, line~\ref{algo:Copy_if} checks if a
copy already exists. If it does not exist, then a copy is created 
from the parent thread's copy if available (line~\ref{algo:Copy_parent}), 
otherwise from the shared variable (line~\ref{algo:Copy_global}).  
Finally, the updated environment \Environment{} is returned (line~\ref{algo:Copy_return}).

\begin{algorithm}[t]
	\begin{algorithmic}[1]
		\Function{$\Copy$}{\Environment{}, \thread{}}
			\ForAll {$\var \in \Shared(\thread{})$}						\Comment{For all shared variables needed by thread \thread{}.}\label{algo:Copy_retrieve}
				\If {$\var \notin \Environment[\thread{}]$}				\Comment{If thread \thread{} does not have a copy.}\label{algo:Copy_if}
					\If {$\var \in \Environment[\Parent(\thread{})]$}
						\State $\Environment[\thread{}][\var \gets (\Environment[\Parent(\thread{})][\var].val, \texttt{pre})]$	\Comment{Copy its parent's copy.}\label{algo:Copy_parent}
					\Else
						\State $\Environment[\thread{}][\var \gets (\Environment[\Global][\var].val, \texttt{pre})]$			\Comment{Copy the shared variable.}\label{algo:Copy_global}
					\EndIf
				\EndIf
			\EndFor
			\State \Return $\Environment$		\label{algo:Copy_return}
		\EndFunction
	\end{algorithmic}
	
	\caption{Copy all the shared variables needed by a thread.}
	\label{algo:Copy}
\end{algorithm}

\subsubsection{\textsc{Combine}}
\label{sec:forec_Combine}
The semantic function $\Combine(\Environment, \thread{1}, \thread{2}, \thread{0})$ combines 
the copies of shared variables of two threads. This 
function is described by Algorithm~\ref{algo:Combine}. The inputs to the 
algorithm are: the program's environment \Environment{}, two threads 
\thread{1} and \thread{2}, and a thread \thread{0} to assign the 
combined values to. The output is an updated environment \Environment{}.
The algorithm combines the shared variables one at a time (line~\ref{algo:Combine_for}).
By default, we combine the copies from both threads if they exist (line~\ref{algo:Combine_default}).
If the combine policy is \verb$new$ (line~\ref{algo:Combine_new}), then it finds the copies with 
values different to $\Environment[\Global][\var]$ (i.e., the \verb$pre$ value) 
or are partially combined values (i.e., status is \verb$cmb$).
If the combine policy is \verb$mod$ (line~\ref{algo:Combine_mod}), then it finds the copies 
that are modified or are partially combined values (i.e., status is \verb$mod$ or \verb$cmb$).
If two copies are found, then the shared variable's combine 
function is retrieved (line~\ref{algo:Combine_c}) to compute the combined value (line~\ref{algo:Combine_combine}).
The combined value is assigned to thread \thread{0} with the status 
\verb$cmb$ to indicate that it is a partially combined result (line~\ref{algo:Combine_assign2}).
If one copy is found, then the value of that copy is assigned to thread \thread{0}
with the status \verb$cmb$ (line~\ref{algo:Combine_assign1}). 
Finally, the updated environment \Environment{} without thread \thread{1} 
and \thread{2}'s store is returned (line~\ref{algo:Combine_return}).

\begin{algorithm}[t]
	\begin{algorithmic}[1]
		\Function{$\Combine$}{\Environment{}, \thread{1}, \thread{2}, \thread{0}}
			\ForAll {$\var \in \Shared(\Global)$}										\Comment{All the shared variables.}\label{algo:Combine_for}
				\State $\Thread = \lbrace \thread{} \mid \thread{} \in \lbrace \thread{1}, \thread{2} \rbrace, \var \in \Environment[\thread{}] \rbrace$		\label{algo:Combine_default}
				\If {$\GetPolicy(\var) = \texttt{new}$}
					\State $\Thread = \lbrace \thread{} \mid \thread{} \in \Thread, \Environment[\thread{}][\var].\val \neq \Environment[\Global][\var].\val \lor \Environment[\thread{}][\var].\sts = \texttt{cmb} \rbrace$	\label{algo:Combine_new}
				\ElsIf {$\GetPolicy(\var) = \texttt{mod}$}
					\State $\Thread = \lbrace \thread{} \mid \thread{} \in \Thread, \Environment[\thread{}][\var].\sts \in \{ \texttt{mod}, \texttt{cmb} \} \rbrace$						\label{algo:Combine_mod}
				\EndIf	\label{algo:Combine_endif1}
				\If {$\lvert \Thread \rvert = 2$}	\label{algo:Combine_if}
					\State $cf = \GetCombine(\var)$													\Comment{Combine function for \var{}.}\label{algo:Combine_c}
					\State $\val = cf(\Environment[\thread{1}][\var].\val, \Environment[\thread{2}][\var].\val, \Environment[\Global][\var].\val)$	\Comment{Combine the two copies.}\label{algo:Combine_combine}
					\State $\Environment[\thread{0}][\var \gets (\val, \texttt{cmb}) ]$	\Comment{Assign the combined result to \thread{0}.}\label{algo:Combine_assign2}
				\ElsIf {$\lvert \Thread \rvert = 1$}
					\State $\Environment[\thread{0}][\var \gets (\Environment[\thread{} \in \Thread][\var].\val, \texttt{cmb}) ]$	\Comment{Assign \thread{}'s value directly to \thread{0}.}\label{algo:Combine_assign1}
				\EndIf	\label{algo:Combine_endif2}
			\EndFor
			\State \Return $\Environment \setminus \{ \thread{1}, \thread{2} \}$															\label{algo:Combine_return}
		\EndFunction
	\end{algorithmic}
	
	\caption{Combines the copies of shared variables of two threads.}
	\label{algo:Combine}
\end{algorithm}


%-----------------------------------------------------------------------------

\subsection{The Structural Operational Semantics}

\subsubsection{The \texttt{nop} Statement}
The \verb$nop$ statement does nothing and terminates:
\begin{equation*}
	\tag{nop}
	\label{forec:nop}
	\langle \Environment, \Abort \rangle ~ \thread{}:\mathtt{nop}
		\xrightarrow[~~\Input~~]{0} 
	\langle \Environment, \Abort \rangle ~ \thread{}:
\end{equation*}

\subsubsection{The \texttt{copy} Statement}
The \verb$copy$ statement copies the shared variables needed
by thread \thread{}:
\begin{equation*}
	\tag{copy}
	\label{forec:copy}
	\langle \Environment, \Abort \rangle ~ \thread{}:\mathtt{copy}
		\xrightarrow[~~\Input~~]{0} 
	\langle \Copy(\Environment, \thread{}), \Abort \rangle ~ \thread{}:\mathtt{nop}
\end{equation*}

\subsubsection{The \texttt{pause} Statement}
The \verb$pause$ statement rewrites into the \verb$copy$ statement
and pauses. The \verb$copy$ statement ensures that thread \thread{} 
starts its next local tick by copying the shared variables it needs:
\begin{equation*}
	\tag{pause}
	\label{forec:pause}
	\langle \Environment, \Abort \rangle ~ \thread{}:\mathtt{pause}
		\xrightarrow[~~\Input~~]{1} 
	\langle \Environment, \Abort \rangle ~ \thread{}:\mathtt{copy}
\end{equation*}

\subsubsection{The \texttt{status} Statement}
The \verb$status$ statement sets \verb$abort$ $a$'s preemption status
to the value of the expression \expression{}:
\begin{equation*}
	\tag{status}
	\label{forec:status}
	\langle \Environment, \Abort \rangle ~ \thread{}:\mathtt{status}(a, \expression)
		\xrightarrow[~~\Input~~]{0} 
	\langle \Environment, \Abort[a \gets \Eval(\Environment, \Input, \thread{}, \expression)] \rangle ~ \thread{}:\mathtt{nop}
\end{equation*}

\subsubsection{The \texttt{abort} Statement}
The \verb$abort$ executes its body \body{} if its preemption has not been triggered:
\begin{equation*}
	\tag{abort-1}
	\label{forec:abort-1}
	\frac{
			\langle \Environment, \Abort \rangle ~ \thread{}:\body{}
				\xrightarrow[~~\Input~~]{k \in \lbrace 1, \bot \rbrace} 
			\langle \Environment', \Abort' \rangle ~ \thread{}:\body{}'
			\quad
			\Abort[a] = 0
		}{
			\langle \Environment, \Abort \rangle ~ \thread{}:\mathtt{weak}?~\mathtt{abort}(a, \body{})
				\xrightarrow[~~\Input~~]{k} 
			\langle \Environment', \Abort' \rangle ~ \thread{}:\mathtt{weak}?~\mathtt{abort}(a, \body{}')
		}
\end{equation*}
The \verb$abort$ terminates normally if its body terminates and its preemption has not
been triggered:
\begin{equation*}
	\tag{abort-2}
	\label{forec:abort-2}
	\frac{
			\langle \Environment, \Abort \rangle ~ \thread{}:\body{}
				\xrightarrow[~~\Input~~]{0} 
			\langle \Environment', \Abort' \rangle ~ \thread{}:\body{}'
			\quad
			\Abort[a] = 0
		}{
			\langle \Environment, \Abort \rangle ~ \thread{}:\mathtt{weak}?~\mathtt{abort}(a, \body{})
				\xrightarrow[~~\Input~~]{0} 
			\langle \Environment', \Abort' \rangle ~ \thread{}:\mathtt{nop}
		}
\end{equation*}
Additionally, the \verb$weak abort$ terminates normally even
if its preemption has been triggered:
\begin{equation*}
	\tag{abort-3}
	\label{forec:abort-3}
	\frac{
			\langle \Environment, \Abort \rangle ~ \thread{}:\body{}
				\xrightarrow[~~\Input~~]{0} 
			\langle \Environment', \Abort' \rangle ~ \thread{}:\body{}'
			\quad
			\Abort[a] \neq 0
		}{
			\langle \Environment, \Abort \rangle ~ \thread{}:\mathtt{weak}~\mathtt{abort}(a, \body{})
				\xrightarrow[~~\Input~~]{0} 
			\langle \Environment', \Abort' \rangle ~ \thread{}:\mathtt{nop}
		}
\end{equation*}
The \verb$weak abort$ terminates if its body pauses and its preemption has been triggered. 
It rewrites into the \verb$copy$ statement because it could be the start of thread \thread{}'s 
local tick:
\begin{equation*}
	\tag{abort-4}
	\label{forec:abort-4}
	\frac{
			\langle \Environment, \Abort \rangle ~ \thread{}:\body{}
				\xrightarrow[~~\Input~~]{1} 
			\langle \Environment', \Abort' \rangle ~ \thread{}:\body{}'
			\quad
			\Abort[a] \neq 0
		}{
			\langle \Environment, \Abort \rangle ~ \thread{}:\mathtt{weak}~\mathtt{abort}(a, \body{})
				\xrightarrow[~~\Input~~]{\bot} 
			\langle \Environment', \Abort' \rangle ~ \thread{}:\mathtt{copy}
		}
\end{equation*}
The \verb$weak abort$ allows its body to execute until it pauses if its 
preemption has been triggered:
\begin{equation*}
	\tag{abort-5}
	\label{forec:abort-5}
	\frac{
			\langle \Environment, \Abort \rangle ~ \thread{}:\body{}
				\xrightarrow[~~\Input~~]{\bot} 
			\langle \Environment', \Abort' \rangle ~ \thread{}:\body{}'
			\quad
			\Abort[a] \neq 0
		}{
			\langle \Environment, \Abort \rangle ~ \thread{}:\mathtt{weak}~\mathtt{abort}(a, \body{})
				\xrightarrow[~~\Input~~]{\bot} 
			\langle \Environment', \Abort' \rangle ~ \thread{}:\mathtt{weak}~\mathtt{abort}(a, \body{}')
		}
\end{equation*}
The strong \verb$abort$ terminates without executing its body if its preemption has been triggered. 
It rewrites into the \verb$copy$ statement because it could be the start of thread \thread{}'s 
local tick:
\begin{equation*}
	\tag{abort-6}
	\label{forec:abort-6}
	\frac{
			\langle \Environment, \Abort \rangle ~ \thread{}:\body{}
				\xrightarrow[~~\Input~~]{k \in \lbrace 0, 1, \bot \rbrace} 
			\langle \Environment', \Abort' \rangle ~ \thread{}:\body{}'
			\quad
			\Abort[a] \neq 0
		}{
			\langle \Environment, \Abort \rangle ~ \thread{}:\mathtt{abort}(a, \body{})
				\xrightarrow[~~\Input~~]{\bot} 
			\langle \Environment, \Abort \rangle ~ \thread{}:\mathtt{copy}
		}
\end{equation*}

\subsubsection{The Assignment Operator (\texttt{=})}
The assignment operator evaluates the expression
\expression{} into a value $\val = \Eval(\Environment, \Input, \thread{}, \expression)$.
If \var{} is a shared variable (\ref{forec:assign-shared}), then the value and status \verb$mod$ 
is assigned to the thread's copy in $\Environment[\thread{}]$. 
Otherwise, if the variable \var{} is a private variable (\ref{forec:assign-private}), then the value 
and status \verb$pre$ is assigned to the global variable in $\Environment[\Global]$:
\begin{equation*}
	\tag{assign-shared}
	\label{forec:assign-shared}
	\frac{
			\var \in \Shared(\thread{})
		}{
			\langle \Environment, \Abort \rangle ~ \thread{}:\var \mathtt{=} \expression
				\xrightarrow[~~\Input~~]{0} 
			\langle \Environment[\thread{}][\var \gets (\val, \texttt{mod})], \Abort \rangle ~ \thread{}:\mathtt{nop}
		}
\end{equation*}

\begin{equation*}
	\tag{assign-private}
	\label{forec:assign-private}
	\frac{
			\var \notin \Shared(\thread{})
		}{
			\langle \Environment, \Abort \rangle ~ \thread{}:\var \mathtt{=} \expression
				\xrightarrow[~~\Input~~]{0} 
			\langle \Environment[\Global][\var \gets (\val, \texttt{pre})], \Abort \rangle ~ \thread{}:\mathtt{nop}
		}
\end{equation*}

\subsubsection{The \texttt{if-else} Statement}
A conditional construct is rewritten into one of its branches, depending on 
the value of its condition \expression{}:
\begin{equation*}
	\tag{if-then}
	\label{forec:if-then}
	\frac{
			\Eval(\Environment, \Input, \thread{}, \expression) \neq 0
		}{
			\langle \Environment, \Abort \rangle ~ \thread{}:\mathtt{if}~(\expression)~\body{1}~\mathtt{else}~\body{2}
				\xrightarrow[~~\Input~~]{\bot} 
			\langle \Environment, \Abort \rangle ~ \thread{}:\body{1}
		}
\end{equation*}

\begin{equation*}
	\tag{if-else}
	\label{forec:if-else}
	\frac{
			\Eval(\Environment, \Input, \thread{}, \expression) = 0
		}{
			\langle \Environment, \Abort \rangle ~ \thread{}:\mathtt{if}~(\expression)~\body{1}~\mathtt{else}~\body{2}
				\xrightarrow[~~\Input~~]{\bot} 
			\langle \Environment, \Abort \rangle ~ \thread{}:\body{2}
		}
\end{equation*}

\subsubsection{The \texttt{while} Statement}
The body of a loop statement is unrolled once, depending 
on the value of its condition \expression{}:
\begin{equation*}
	\tag{loop-then}
	\label{forec:loop-then}
	\frac{
			\Eval(\Environment, \Input, \thread{}, \expression) \neq 0
		}{
			\langle \Environment, \Abort \rangle ~ \thread{}:\mathtt{while}~(\expression)~\body{}
				\xrightarrow[~~\Input~~]{\bot} 
			\langle \Environment, \Abort \rangle ~ \thread{}:\body{};~\mathtt{while}~(\expression)~\body{}
		}
\end{equation*}

\begin{equation*}
	\tag{loop-else}
	\label{forec:loop-else}
	\frac{
			\Eval(\Environment, \Input, \thread{}, \expression) = 0
		}{
			\langle \Environment, \Abort \rangle ~ \thread{}:\mathtt{while}~(\expression)~\body{}
				\xrightarrow[~~\Input~~]{0} 
			\langle \Environment, \Abort \rangle ~ \thread{}:\mathtt{nop}
		}
\end{equation*}

\subsubsection{The Sequence Operator (\texttt{;})}
For a sequence of program fragments, the first fragment \body{1}
must terminate before the second fragment \body{2} can be rewritten. 
The (\ref{forec:seq-left}) rule emits the completion code of 
the first fragment:
\begin{equation*}
	\tag{seq-left}
	\label{forec:seq-left}
	\frac{
			\langle \Environment, \Abort \rangle ~ \thread{}:\body{1}
				\xrightarrow[~~\Input~~]{k \in \lbrace 1, \bot \rbrace}
			\langle \Environment', \Abort' \rangle ~ \thread{}:\body{1}'
		}{
			\langle \Environment, \Abort \rangle ~ \thread{}:\body{1}\mathtt{;}~\body{2}
				\xrightarrow[~~\Input~~]{k} 
			\langle \Environment', \Abort' \rangle ~ \thread{}:\body{1}'\mathtt{;}~\body{2}
		}
\end{equation*}
The (\ref{forec:seq-right}) rule is applied if the 
first fragment terminates:
\begin{equation*}
	\tag{seq-right}
	\label{forec:seq-right}
	\frac{
			\langle \Environment, \Abort \rangle ~ \thread{}:\body{1}
				\xrightarrow[~~\Input~~]{0}
			\langle \Environment', \Abort' \rangle ~ \thread{}:\body{1}'
		}{
			\langle \Environment, \Abort \rangle ~ \thread{}:\body{1}\mathtt{;}~\body{2}
				\xrightarrow[~~\Input~~]{\bot}
			\langle \Environment', \Abort' \rangle ~ \thread{}:\body{2}
		}
\end{equation*}

\subsubsection{The \texttt{par} Statement}
The \verb$par$ statement allows both its child threads, $\thread{1}$
and $\thread{2}$, to execute in parallel. The parent thread is $\thread{0}$:
\begin{equation*}
	\tag{par-1}
	\label{forec:par-1}
	\frac{
		\langle \Environment, \Abort \rangle ~ \thread{1}:\body{1}
			\xrightarrow[~~\Input~~]{\bot} 
		\langle \Environment', \Abort' \rangle ~ \thread{1}:\body{1}'
		\quad
		\langle \Environment, \Abort \rangle ~ \thread{2}:\body{2}
			\xrightarrow[~~\Input~~]{\bot} 
		\langle \Environment'', \Abort'' \rangle ~ \thread{2}:\body{2}'
	}{
		\langle \Environment, \Abort \rangle ~ \thread{0}:\mathtt{par}(\thread{1}:\body{1},~\thread{2}:\body{2})
			\xrightarrow[~~\Input~~]{\bot} 
		\langle \Environment^M, \Abort^M \rangle ~ \thread{0}:\mathtt{par}(\thread{1}:\body{1}',~\thread{2}:\body{2}')
	}
\end{equation*}
The variables that changed in $\Environment'$ or $\Environment''$ are merged 
together to form $\Environment^M$ by taking the union of the changes 
in $\Environment'$ (i.e., $\Environment' \setminus (\Environment' \cap \Environment)$) and in
$\Environment''$ (i.e., $\Environment'' \setminus (\Environment'' \cap \Environment)$) 
with the remaining unchanged variables
(i.e., $\Environment' \cap \Environment''$). Note that intersecting two environments, 
e.g., $\Environment' \cap \Environment''$,
produces a new environment containing variables that have the same values and statuses in $\Environment'$
and $\Environment''$.
Also, parallel threads always
modify the same environment $\Environment$ in a mutually exclusive
manner. The (\ref{forec:assign-shared}) rule only 
allows a thread to access its own copies of shared variables and the 
(\ref{forec:assign-private}) rule only allows a thread to access its 
own private variables.
Thus, $\Environment^M = \left ( \Environment' \setminus (\Environment' \cap \Environment) \right ) \cup \left ( \Environment'' \setminus (\Environment'' \cap \Environment) \right ) \cup \left ( \Environment' \cap \Environment'' \right )$.
Similarly, the \verb$abort$ statuses that changed in $\Abort'$
and $\Abort''$ are merged together to form 
$\Abort^M = \left ( \Abort' \setminus (\Abort' \cap \Abort) \right ) \cup \left ( \Abort'' \setminus (\Abort'' \cap \Abort) \right ) \cup \left ( \Abort' \cap \Abort'' \right )$.

If a child thread can complete its local tick, by pausing or 
terminating, then it will wait for its sibling to complete their local tick:
\begin{equation*}
	\tag{par-2}
	\label{forec:par-2}
	\frac{
		\langle \Environment, \Abort \rangle ~ \thread{1}:\body{1}
			\xrightarrow[~~\Input~~]{k \in \lbrace 0, 1 \rbrace} 
		\langle \Environment', \Abort' \rangle ~ \thread{1}:\body{1}'
		\quad
		\langle \Environment, \Abort \rangle ~ \thread{2}:\body{2}
			\xrightarrow[~~\Input~~]{\bot} 
		\langle \Environment'', \Abort'' \rangle ~ \thread{2}:\body{2}'
	}{
		\langle \Environment, \Abort \rangle ~ \thread{0}:\mathtt{par}(\thread{1}:\body{1},~\thread{2}:\body{2})
			\xrightarrow[~~\Input~~]{\bot} 
		\langle \Environment'', \Abort'' \rangle ~ \thread{0}:\mathtt{par}(\thread{1}:\body{1},~\thread{2}:\body{2}')
	}
\end{equation*}
\begin{equation*}
	\tag{par-3}
	\label{forec:par-3}
	\frac{
		\langle \Environment, \Abort \rangle ~ \thread{1}:\body{1}
			\xrightarrow[~~\Input~~]{\bot} 
		\langle \Environment', \Abort' \rangle ~ \thread{1}:\body{1}'
		\quad
		\langle \Environment, \Abort \rangle ~ \thread{2}:\body{2}
			\xrightarrow[~~\Input~~]{k \in \lbrace 0, 1 \rbrace} 
		\langle \Environment'', \Abort'' \rangle ~ \thread{2}:\body{2}'
	}{
		\langle \Environment, \Abort \rangle ~ \thread{0}:\mathtt{par}(\thread{1}:\body{1},~\thread{2}:\body{2})
			\xrightarrow[~~\Input~~]{\bot} 
		\langle \Environment', \Abort' \rangle ~ \thread{0}:\mathtt{par}(\thread{1}:\body{1}',~\thread{2}:\body{2})
	}
\end{equation*}
The \verb$par$ pauses if both its child threads pause. The changes made to \Environment{} and \Abort{} 
are merged into $\Environment^M$ and $\Abort^M$, respectively, as defined
for the (\ref{forec:par-1}) rule. The copies
of shared variables from the child threads are combined and assigned to their parent thread: 
\begin{equation*}
	\tag{par-4}
	\label{forec:par-4}
	\frac{
		\langle \Environment, \Abort \rangle ~ \thread{1}:\body{1}
			\xrightarrow[~~\Input~~]{1} 
		\langle \Environment', \Abort' \rangle ~ \thread{1}:\body{1}'
		\quad
		\langle \Environment, \Abort \rangle ~ \thread{2}:\body{2}
			\xrightarrow[~~\Input~~]{1} 
		\langle \Environment'', \Abort'' \rangle ~ \thread{2}:\body{2}'
	}{
		\langle \Environment, \Abort \rangle ~ \thread{0}:\mathtt{par}(\thread{1}:\body{1},~\thread{2}:\body{2})
			\xrightarrow[~~\Input~~]{1} 
		\langle \Combine(\Environment^M, \thread{1}, \thread{2}, \thread{0}), \Abort^M \rangle ~ \thread{0}:\mathtt{par}(\thread{1}:\body{1}',~\thread{2}:\body{2}')
	}
\end{equation*}
Otherwise, the \verb$par$ terminates if both its child threads terminate. The \verb$par$
rewrites into the \verb$copy$ statement because it could be the start of the parent
thread's local tick: 
\begin{equation*}
	\tag{par-5}
	\label{forec:par-5}
	\frac{
		\langle \Environment, \Abort \rangle ~ \thread{1}:\body{1}
			\xrightarrow[~~\Input~~]{0} 
		\langle \Environment', \Abort' \rangle ~ \thread{1}:\body{1}'
		\quad
		\langle \Environment, \Abort \rangle ~ \thread{2}:\body{2}
			\xrightarrow[~~\Input~~]{0} 
		\langle \Environment'', \Abort'' \rangle ~ \thread{2}:\body{2}'
	}{
		\langle \Environment, \Abort \rangle ~ \thread{0}:\mathtt{par}(\thread{1}:\body{1},~\thread{2}:\body{2})
			\xrightarrow[~~\Input~~]{\bot} 
		\langle \Combine(\Environment^M, \thread{1}, \thread{2}, \thread{0}), \Abort^M \rangle ~ \thread{0}:\mathtt{copy}
	}
\end{equation*}
If only one child thread terminates, then the terminated thread rewrites into the \verb$nop$ statement:
% We do not rewrite the parent thread into the non-terminated child thread 
% because the shared variables that the remaining child thread needs may be
% different from those needed by its parent thread.
\begin{equation*}
	\tag{par-6}
	\label{forec:par-6}
	\frac{
		\langle \Environment, \Abort \rangle ~ \thread{1}:\body{1}
			\xrightarrow[~~\Input~~]{0} 
		\langle \Environment', \Abort' \rangle ~ \thread{1}:\body{1}'
		\quad
		\langle \Environment, \Abort \rangle ~ \thread{2}:\body{2}
			\xrightarrow[~~\Input~~]{1} 
		\langle \Environment'', \Abort'' \rangle ~ \thread{2}:\body{2}'
	}{
		\langle \Environment, \Abort \rangle ~ \thread{0}:\mathtt{par}(\thread{1}:\body{1},~\thread{2}:\body{2})
			\xrightarrow[~~\Input~~]{1} 
		\langle \Combine(\Environment^M, \thread{1}, \thread{2}, \thread{0}), \Abort^M \rangle ~ \thread{0}:\mathtt{par}(\thread{1}:\mathtt{nop},~\thread{2}:\body{2}')
	}
\end{equation*}
\begin{equation*}
	\tag{par-7}
	\label{forec:par-7}
	\frac{
		\langle \Environment, \Abort \rangle ~ \thread{1}:\body{1}
			\xrightarrow[~~\Input~~]{1} 
		\langle \Environment', \Abort' \rangle ~ \thread{1}:\body{1}'
		\quad
		\langle \Environment, \Abort \rangle ~ \thread{2}:\body{2}
			\xrightarrow[~~\Input~~]{0} 
		\langle \Environment'', \Abort'' \rangle ~ \thread{2}:\body{2}'
	}{
		\langle \Environment, \Abort \rangle ~ \thread{0}:\mathtt{par}(\thread{1}:\body{1},~\thread{2}:\body{2})
			\xrightarrow[~~\Input~~]{1} 
		\langle \Combine(\Environment^M, \thread{1}, \thread{2}, \thread{0}), \Abort^M \rangle ~ \thread{0}:\mathtt{par}(\thread{1}:\body{1}',~\thread{2}:\mathtt{nop})
	}
\end{equation*}

\subsubsection{Tick Completion}
A tick completes if the \verb$main$ thread pauses or
terminates. The shared variables are resynchronized, preemption statuses
are updated, outputs are emitted, and inputs are sampled:
\begin{equation*}
	\tag{tick}
	\label{forec:tick}
	\frac{
			\langle \Environment, \Abort \rangle ~ main:\body{}
				\xrightarrow[~~\Input~~]{k \in \lbrace 0, 1 \rbrace} 
			\langle \Environment', \Abort' \rangle ~ main:\body{}'
		}{
			\langle \Environment, \Abort \rangle ~ main:\body{}
				\xrightarrow[~~\Input'~~]{k} 
			\langle \Environment'', \Abort'' \rangle ~ main:\body{}'
		}
\end{equation*}
The shared variables are resynchronized by assigning the 
combined values in $\Environment'[main]$ (i.e., \verb$main$'s store) to
their corresponding variables in $\Environment'[\Global]$ and then removing \verb$main$'s store. Thus, 
$\forall \var \in \Environment'[main]$, we have 
$\Environment'' = \Environment'[\Global][\var \gets (\Environment'[main][\var].\val, \texttt{pre})] \setminus \{main\}$.
All the preemption statuses are updated by evaluating their preemption 
conditions with the resynchronized global variables. Thus, 
$\forall \abort \in \Abort'$, we have $\Abort'' = \Abort'[\abort \gets \Eval(\Environment'', \Input, \Global, \GetExp(a))]$.


%-----------------------------------------------------------------------------

%\subsection{Comparison with Concurrent Revisions}
%SPC-MoC subsumes the idea of Concurrent Revisions while offering extra flexibility.
%Similarities with Concurrent Revisions:
%\begin{itemize}
%	\item Assignment to variables: Classic data statements.
%	\item Classic control statements: Conditional statements.
%	\item Forking and joining of threads: Concurrent Revisions is more expressive
%		  (Synchronous versus Asynchronous, Static versus Dynamic).
%\end{itemize}
%Dissimilarities with Concurrent Revisions:
%\begin{itemize}
%	\item Blocking semantics of fork and join.
%	\item Combining of revisions is only based on value.
%	\item Preemptions.
%	\item Synchronisation during thread executions (the butterfly communication pattern not 
%		  supported by Concurrent Revisions).
%\end{itemize}
%
%Creating equivalent Concurrent Revision programs in ForeC (ForeC is a superset of 
%Concurrent Revisions?):
%\begin{itemize}
%	\item Only use data, control, and fork/join statements.
%	\item How to extend the proofs of determinacy to preemptions and ticks?
%	\begin{itemize}
%		\item Preemptions: Preemption of executions are defined precisely by the control-flow.
%			  It is never the case that you can choose between taking a preemption and performing
%			  another action (deterministic execution semantics).
%		\item Ticks: All threads have paused and cannot continue executing. Therefore, 
%			  all copies of a shared variable can be safely combined.
%	\end{itemize}
%\end{itemize}
