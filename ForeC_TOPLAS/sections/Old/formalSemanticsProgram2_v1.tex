\subsubsection{Program 2}
\begin{lstlisting}[style=snippet]
int x=0;
void main(void) {
  weak abort {
    x=1; pause; x=2;
  } when immediate(x==0);
}
\end{lstlisting}
This program is structurally translated to:
\begin{lstlisting}[style=snippet]
int x=0;
void main(void) {
  addAbort(a_0,1,x==0);
  par( t1:{x=1; pause; x=2; preempt();}, t2:{preempt(a_0);} );
}
\end{lstlisting}
The initial program state and its derivatives are defined 
in Table~\ref{figure:forec_program_2}.
\newline

\begin{figure}
	\centering
	$$\begin{array}{l l l l l l l}
		 \Environment		&=& \lbrace \Global \to \lbrace x \to 0	\rbrace \rbrace	& \qquad \Abort		&=& \lbrace ~ \rbrace			\\
		 \Environment^1		&=& \lbrace \Global \to \lbrace x \to 1	\rbrace \rbrace & \qquad \Abort^1	&=& \lbrace a\_0 \to 1 \rbrace	\\
		 					& &														& \qquad \Abort^2	&=& \lbrace a\_0 \to 0 \rbrace	\\
	\end{array}$$
	
	\caption{Definition of the initial program state and its derivatives for Program 2.}
	\label{figure:forec_program_2}
\end{figure}

\noindent
Step 1: Apply the (\ref{forec:seq-left}) and (\ref{forec:imm-abort}) rules. 
\begin{equation*}
	\frac{
		\dfrac{
				1 = 1
			}{
				\langle \Environment, \Abort \rangle ~ \mathtt{main: addAbort(a\_0, 1, x==0)}
					\xrightarrow[~~\Input~~]{\bot} 
				\langle \Environment, \Abort^1 \rangle ~ \mathtt{main: nop}
			}
		}{
			\begin{array}{l}
				\langle \Environment, \Abort \rangle ~ \mathtt{main: addAbort(a\_0, 1, x==0);}		\\
				\mathtt{par( t1:\{x=1; pause; x=2; preempt();\},}									\\
				\mathtt{t2:\{preempt(a\_0);\} );}													\\
			\end{array}
				\xrightarrow[~~\Input~~]{\bot} 
			\begin{array}{l}
				\langle \Environment, \Abort^1 \rangle ~ \mathtt{main: nop;~par( t1:\{x=1; pause;}	\\
				\mathtt{x=2; preempt();\},}															\\
				\mathtt{t2:\{preempt(a\_0);\} );}	
			\end{array}
		}
\end{equation*}

\noindent
Step 2: Apply the (\ref{forec:seq-right}) and (\ref{forec:nop}) rules. 
\begin{equation*}
	\frac{
			\langle \Environment, \Abort^1 \rangle ~ \mathtt{main: nop}
				\xrightarrow[~~\Input~~]{0} 
			\langle \Environment, \Abort^1 \rangle ~ \mathtt{main: }
		}{
			\begin{array}{l}
				\langle \Environment, \Abort^1 \rangle ~ \mathtt{main: nop;~par( t1:\{x=1; pause;}	\\
				\mathtt{x=2; preempt();\}, t2:\{preempt(a\_0);\} );}								\\
			\end{array}
				\xrightarrow[~~\Input~~]{\bot} 
			\begin{array}{l}
				\langle \Environment, \Abort^1 \rangle ~ \mathtt{main: par( t1:\{x=1; pause;}		\\
				\mathtt{x=2; preempt();\}, t2:\{preempt(a\_0);\} );}								\\
			\end{array}
		}
\end{equation*}

\noindent
Step 3: Apply the (\ref{forec:par-3}) rule. Additionally, apply the 
(\ref{forec:seq-left}) and (\ref{forec:assign-private}) rules to the 
first thread, and the (\ref{forec:preempt}) rule to the second thread.
\begin{equation*}
	\frac{
			\twolines{
				~
			}{
				term1
			}
			\qquad
			\qquad
		\dfrac{
				1 \neq 0
			}{
				\langle \Environment, \Abort^1 \rangle ~ \mathtt{main: preempt(a\_0)}
					\xrightarrow[~~\Input~~]{2} 
				\langle \Environment, \Abort^1 \rangle ~ \mathtt{main: nop}
			}
		}{
			\begin{array}{l}
				\langle \Environment, \Abort^1 \rangle ~ \mathtt{main: par( t1:\{x=1; pause; x=2;}	\\
				\mathtt{preempt();\}, t2:\{preempt(a\_0);\} );}										\\
			\end{array}
				\xrightarrow[~~\Input~~]{\bot} 
			\begin{array}{l}
				\langle \Environment^1, \Abort^1 \rangle ~ \mathtt{main: par( t1:\{nop; pause; x=2;}\\
				\mathtt{preempt();\}, t2:\{preempt(a\_0);\} );}										\\
			\end{array}
		}
\end{equation*}
where $term1$ represents:
\begin{equation*}
	\frac{
			\dfrac{
				x \notin \Shared(\Environment[\Global])
			}{
				\langle \Environment, \Abort^1 \rangle ~ \mathtt{t1: x=1}
					\xrightarrow[~~\Input~~]{\bot} 
				\langle \Environment^1, \Abort^1 \rangle ~ \mathtt{t1: nop}
			}
		}{
			\langle \Environment, \Abort^1 \rangle ~ \mathtt{t1: x=1; pause;x=2; preempt()}
				\xrightarrow[~~\Input~~]{\bot} 
			\langle \Environment^1, \Abort^1 \rangle ~ \mathtt{t1: nop; pause;x=2; preempt()}
		}
\end{equation*}

\noindent
Step 4: Apply the (\ref{forec:par-3}) rule. Additionally, apply the 
(\ref{forec:seq-right}) and (\ref{forec:nop}) rules to the first
thread, and the (\ref{forec:preempt}) rule to the second thread.
\begin{equation*}
	\frac{
			\twolines{
				~
			}{
				term1
			}
			\qquad
			\qquad
		\dfrac{
				1 \neq 0
			}{
				\langle \Environment^1, \Abort^1 \rangle ~ \mathtt{main: preempt(a\_0)}
					\xrightarrow[~~\Input~~]{2} 
				\langle \Environment^1, \Abort^1 \rangle ~ \mathtt{main: nop}
			}
		}{
			\begin{array}{l}
				\langle \Environment^1, \Abort^1 \rangle ~ \mathtt{main: par( t1:\{nop; pause;x=2;}	\\
				\mathtt{preempt();\}, t2:\{preempt(a\_0);\} );}										\\
			\end{array}
				\xrightarrow[~~\Input~~]{\bot} 
			\begin{array}{l}
				\langle \Environment^1, \Abort^1 \rangle ~ \mathtt{main: par( t1:\{pause;x=2;}		\\
				\mathtt{preempt();\}, t2:\{preempt(a\_0);\} );}										\\
			\end{array}
		}
\end{equation*}
where $term2$ represents:
\begin{equation*}
	\frac{
			\langle \Environment^1, \Abort^1 \rangle ~ \mathtt{t1: nop}
				\xrightarrow[~~\Input~~]{0} 
			\langle \Environment^1, \Abort^1 \rangle ~ \mathtt{t1: }
		}{
			\langle \Environment^1, \Abort^1 \rangle ~ \mathtt{t1: nop; pause;x=2; preempt()}
				\xrightarrow[~~\Input~~]{\bot} 
			\langle \Environment^1, \Abort^1 \rangle ~ \mathtt{t1: pause;x=2; preempt()}
		}
\end{equation*}

\noindent
Step 5: Apply the (\ref{forec:par-9}) rule. Additionally, apply the 
(\ref{forec:seq-left}) and (\ref{forec:pause}) rules to the first thread,
and the (\ref{forec:preempt}) rule to the second thread.
\begin{equation*}
	\frac{
			\twolines{
				~
			}{
				term3
			}
			\qquad
			\qquad
		\dfrac{
				1 \neq 0
			}{
				\langle \Environment^1, \Abort^1 \rangle ~ \mathtt{main: preempt(a\_0)}
					\xrightarrow[~~\Input~~]{2} 
				\langle \Environment^1, \Abort^1 \rangle ~ \mathtt{main: nop}
			}
		}{
			\begin{array}{l}
				\langle \Environment^1, \Abort^1 \rangle ~ \mathtt{main: par( t1:\{pause;x=2;}		\\
				\mathtt{preempt();\}, t2:\{preempt(a\_0);\} );}										\\
			\end{array}
				\xrightarrow[~~\Input~~]{\bot} 
			\begin{array}{l}
				\langle \Environment^1, \Abort^1 \rangle ~ \mathtt{main: nop;}						\\
			\end{array}
		}
\end{equation*}
where $term3$ represents:
\begin{equation*}
	\frac{
			\langle \Environment^1, \Abort^1 \rangle ~ \mathtt{t1: pause}
				\xrightarrow[~~\Input~~]{1} 
			\langle \Environment^1, \Abort^1 \rangle ~ \mathtt{t1: nop}
		}{
			\langle \Environment^1, \Abort^1 \rangle ~ \mathtt{t1: pause;x=2; preempt()}
				\xrightarrow[~~\Input~~]{1} 
			\langle \Environment^1, \Abort^1 \rangle ~ \mathtt{t1: nop;x=2; preempt()}
		}
\end{equation*}

\noindent
Step 6: Apply the (\ref{forec:global-tick}) rule.
The program terminates.
\begin{equation*}
	\frac{
			\mathtt{main} = main
			\quad
			\langle \Environment^1, \Abort^1 \rangle ~ \mathtt{main: nop}
				\xrightarrow[~~\Input~~]{0} 
			\langle \Environment^1, \Abort^1 \rangle ~ \mathtt{main: }
		}{
			\langle \Environment^1, \Abort^1 \rangle ~ \mathtt{main: nop;}
				\xrightarrow[~~\Input~~]{0} 
			\langle \Environment^1, \Abort^2 \rangle ~ \mathtt{main: }
		}
\end{equation*}
