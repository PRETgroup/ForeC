\subsubsection{Program 6}
\begin{lstlisting}[style=snippet]
int plus(int a, int b) {return a + b;}
shared:plus int x=2;
void main(void) {
  x=x+1;
  par( 
    t0:{
      x=x+2; 
      par( t1:{x=x+3;} );
    } 
  );
}
\end{lstlisting}
The initial program state, $\langle State \rangle$,
and its derivatives are defined in Table~\ref{table:forec_program_6}.
\newline

\begin{table}[t]
	\centering
	\renewcommand{\arraystretch}{1.25}
	
	\begin{tabular}{| l | c | c | c | c | c |}
		\hline
		\textbf{State}						& \textbf{\textit{A}}			& \textbf{\textit{I}} 			& \textbf{\textit{P}} 	& \textbf{\textit{C}} 	& \textbf{\textit{E}}											\\ \hline
		$\langle State \rangle$				& $\lbrace main \rbrace$		& $\lbrace ~ \rbrace$			& $\lbrace ~ \rbrace$	& $\lbrace ~ \rbrace$	& $\lbrace \Global \to \lbrace x \to 2 \rbrace \rbrace$			\\ \cline{1-1} \cline{6-6}
		$\langle State^1 \rangle$			&								&								&						&						& $\begin{Bmatrix} \begin{array}{r l} 
																																								\Global &\to \lbrace x \to 2 \rbrace,	\\ 
																																								main &\to \lbrace x \to 3 \rbrace 
																																							\end{array} \end{Bmatrix}$										\\ \cline{1-3}
		$\langle State^2 \rangle$			& $\lbrace t0 \rbrace$			& $\lbrace main \rbrace$		&						&						& 																\\ \cline{1-1} \cline{6-6}
		$\langle State^3 \rangle$			& 								& 								&						&						& $\begin{Bmatrix} \begin{array}{r l} 
																																								\Global &\to \lbrace x \to 2 \rbrace,	\\ 
																																								main &\to \lbrace x \to 3 \rbrace,		\\
																																								t0 &\to \lbrace x \to 5 \rbrace 
																																							\end{array} \end{Bmatrix}$										\\ \cline{1-3}
		$\langle State^4 \rangle$			& $\lbrace t1 \rbrace$			& $\lbrace main, t0 \rbrace$	&						&						& 																\\ \cline{1-1} \cline{6-6}
		$\langle State^5 \rangle$			& 								& 								& 						&						& $\begin{Bmatrix} \begin{array}{r l} 
																																								\Global &\to \lbrace x \to 2 \rbrace,	\\ 
																																								main &\to \lbrace x \to 3 \rbrace,		\\
																																								t0 &\to \lbrace x \to 5 \rbrace, 		\\
																																								t1 &\to \lbrace x \to 8 \rbrace
																																							\end{array} \end{Bmatrix}$										\\ \cline{1-3} \cline{6-6}
		$\langle State^6 \rangle$			& $\lbrace t0 \rbrace$			& $\lbrace main \rbrace$		&						&						& $\begin{Bmatrix} \begin{array}{r l} 
																																								\Global &\to \lbrace x \to 2 \rbrace,	\\ 
																																								main &\to \lbrace x \to 3 \rbrace,		\\
																																								t0 &\to \lbrace x \to 8 \rbrace
																																							\end{array} \end{Bmatrix}$										\\ \cline{1-3} \cline{6-6}
		$\langle State^7 \rangle$			& $\lbrace main \rbrace$		& $\lbrace ~ \rbrace$			& 						&						& $\begin{Bmatrix} \begin{array}{r l} 
																																								\Global &\to \lbrace x \to 2 \rbrace,	\\ 
																																								main &\to \lbrace x \to 8 \rbrace
																																							\end{array} \end{Bmatrix}$										\\ \cline{1-2} \cline{6-6}
		$\langle State^8 \rangle$			& $\lbrace ~ \rbrace$			&								&						&						& $\lbrace \Global \to \lbrace x \to 8 \rbrace \rbrace$			\\ \hline
	\end{tabular}
	
	\caption{Definition of the initial program state and its derivatives for Program 6.}
	\label{table:forec_program_6}
\end{table}

\noindent
Step 1: Apply the (\ref{forec:seq-right}) and (\ref{forec:assign-shared}) rules 
on thread \verb$main$.
\begin{equation*}
	\frac{
			\langle State \rangle ~ \text{main: x=x+1}
				\xrightarrow[~~\Input~~]{0} 
			\langle State^1 \rangle ~ \text{main: nop}
		}{
			\begin{array}{l}
				\langle State \rangle ~ \text{main: x=x+1;}	\\
				\text{par( t0:\{}							\\
				\text{~~~x=x+2; par( t1:\{x=x+3;\} );}		\\
				\text{\} );}								
			\end{array}
				\xrightarrow[~~\Input~~]{\bot} 
			\begin{array}{l}
				\langle State^1 \rangle ~ \text{main: par( t0:\{}	\\
				\text{~~~x=x+2; par( t1:\{x=x+3;\} );}				\\
				\text{\} );}								
			\end{array}
		}
\end{equation*}

\noindent
Step 2: Apply the (\ref{forec:par}) rule on thread \verb$main$.
Thread \verb$t0$ is forked.
\begin{equation*}
	\begin{array}{l}
		\langle State^1 \rangle ~ \text{main: par( t0:\{}	\\
		\text{~~~x=x+2; par( t1:\{x=x+3;\} );}				\\
		\text{\} );}
	\end{array}
		\xrightarrow[~~\Input~~]{\bot} 
	\langle State^2 \rangle ~ \text{main: nop}
\end{equation*}

\noindent
Step 3: Apply the (\ref{forec:seq-right}) and (\ref{forec:assign-shared})
rules on thread \verb$t0$.
\begin{equation*}
	\frac{
			\langle State^2 \rangle ~ \text{t0: x=x+2}
				\xrightarrow[~~\Input~~]{0} 
			\langle State^3 \rangle ~ \text{t0: nop}
		}{
			\langle State^2 \rangle ~ \text{t0:	x=x+2; par( t1:\{x=x+3;\} );}
				\xrightarrow[~~\Input~~]{\bot} 
			\langle State^3 \rangle ~ \text{t0:	par( t1:\{x=x+3;\} );}
		}
\end{equation*}

\noindent
Step 4: Apply the (\ref{forec:par}) rule on thread \verb$t0$.
Thread \verb$t1$ is forked.
\begin{equation*}
	\langle State^3 \rangle ~ \text{t0: par( t1:\{x=x+3;\} );}
		\xrightarrow[~~\Input~~]{\bot} 
	\langle State^4 \rangle ~ \text{t0: nop;}
\end{equation*}

\noindent
Step 5: Apply the (\ref{forec:assign-shared}) rule on thread \verb$t1$.
\begin{equation*}
	\langle State^4 \rangle ~ \text{t1: x=x+3;}
		\xrightarrow[~~\Input~~]{0} 
	\langle State^5 \rangle ~ \text{t1: nop;}
\end{equation*}

\noindent
Step 6: Apply the (\ref{forec:term2}) rule on thread \verb$t1$.
This terminates \verb$t1$ and a join occurs.
\begin{equation*}
	\frac{
			\text{t1} \xrightarrow{~~0~~} \text{t1}
		}{
			\langle State^5 \rangle \xrightarrow{~~~~~} \langle State^6 \rangle
		}
\end{equation*}

\noindent
Step 7: Apply the (\ref{forec:nop}) rule on thread \verb$t0$.
\begin{equation*}
	\langle State^6 \rangle ~ \text{t0: nop}
		\xrightarrow[~~\Input~~]{0} 
	\langle State^6 \rangle ~ \text{t0: }
\end{equation*}

\noindent
Step 8: Apply the (\ref{forec:term2}) rule on thread \verb$t0$.
This terminates \verb$t0$ and a join occurs.
\begin{equation*}
	\frac{
			\text{t0} \xrightarrow{~~0~~} \text{t0}
		}{
			\langle State^6 \rangle \xrightarrow{~~~~~} \langle State^7 \rangle
		}
\end{equation*}

\noindent
Step 9: Apply the (\ref{forec:nop}) rule on thread \verb$main$.
\begin{equation*}
	\langle State^7 \rangle ~ \text{main: nop}
		\xrightarrow[~~\Input~~]{0} 
	\langle State^7 \rangle ~ \text{main: }
\end{equation*}

Step 10: Apply the (\ref{forec:term3}) rule on thread \verb$main$.
The program terminates.
\begin{equation*}
	\frac{
			\text{main} \xrightarrow{~~0~~} \text{main}
		}{
			\langle State^7 \rangle \xrightarrow{~~~~~} \langle State^8 \rangle
		}
\end{equation*}


