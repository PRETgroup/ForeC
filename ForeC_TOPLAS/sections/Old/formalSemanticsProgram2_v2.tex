\subsubsection{Program 2}
\begin{lstlisting}[style=snippet]
shared int s=0 combine all with plus;
void main(void) {
	par({pause; s=3;},{s=4;});
}
\end{lstlisting}
This program is structurally translated to:
\begin{lstlisting}[style=snippet]
int s=0;
void main(void) {
  copy;
  par(t1:{copy; pause; s=3;},t2:{copy; s=4;});
}
\end{lstlisting}
The predicates $\Shared(t1)$, $\Shared(t2)$ and $\Shared(\Global)$ 
are all $\lbrace s \rbrace$.
The predicate $\Shared(main)$ is $\emptyset$.
Initially, the set of preemption statuses \Abort{} is $\emptyset$.
The program's environment \Environment{} and its derivatives 
are defined in Figure~\ref{figure:forec_program_2}.
\newline

\begin{figure}
	\centering
	$$\begin{array}{l l l}
		\Environment		&=& \left \lbrace
									\Global \to \lbrace s \to (0, \mathtt{pre}) \rbrace
								\right \rbrace		\\
		\Environment^1		&=& \left \lbrace
									\Global \to \lbrace s \to (0, \mathtt{pre}) \rbrace,
									t1 \to \lbrace s \to (0, \mathtt{pre}) \rbrace
								\right \rbrace		\\
		\Environment^2		&=& \left \lbrace
									\Global \to \lbrace s \to (0, \mathtt{pre}) \rbrace,
									t2 \to \lbrace s \to (0, \mathtt{pre}) \rbrace
								\right \rbrace		\\
		\Environment^3		&=& \left \lbrace
									\Global \to \lbrace s \to (0, \mathtt{pre}) \rbrace,
									t1 \to \lbrace s \to (0, \mathtt{pre}) \rbrace,
									t2 \to \lbrace s \to (0, \mathtt{pre}) \rbrace
								\right \rbrace		\\
		\Environment^4		&=& \left \lbrace
									\Global \to \lbrace s \to (0, \mathtt{pre}) \rbrace,
									t1 \to \lbrace s \to (0, \mathtt{pre}) \rbrace,
									t2 \to \lbrace s \to (4, \mathtt{mod}) \rbrace
								\right \rbrace		\\
		\Environment^5		&=& \left \lbrace
									\Global \to \lbrace s \to (0, \mathtt{pre}) \rbrace,
									main \to \lbrace s \to (4, \mathtt{cmb}) \rbrace
								\right \rbrace		\\
		\Environment^6		&=& \left \lbrace
									\Global \to \lbrace s \to (4, \mathtt{pre}) \rbrace
								\right \rbrace		\\
		\Environment^7		&=& \left \lbrace
									\Global \to \lbrace s \to (4, \mathtt{pre}) \rbrace,
									t1 \to \lbrace s \to (4, \mathtt{pre}) \rbrace
								\right \rbrace		\\
		\Environment^8		&=& \left \lbrace
									\Global \to \lbrace s \to (4, \mathtt{pre}) \rbrace,
									t1 \to \lbrace s \to (3, \mathtt{mod}) \rbrace
								\right \rbrace		\\
		\Environment^9		&=& \left \lbrace
									\Global \to \lbrace s \to (4, \mathtt{pre}) \rbrace,
									main \to \lbrace s \to (3, \mathtt{cmb}) \rbrace
								\right \rbrace		\\
		\Environment^{10}	&=& \left \lbrace
									\Global \to \lbrace s \to (3, \mathtt{pre}) \rbrace
								\right \rbrace		\\
	\end{array}$$
	
	\caption{Definition of the initial program environment and its derivatives for Program 2.}
	\label{figure:forec_program_2}
\end{figure}

\noindent
Step 1: Apply the (\ref{forec:seq-right}) and (\ref{forec:copy}) rules. The \verb$copy$ statement
has no effect because thread \verb$main$ does not access any shared variables.
\begin{equation*}
	\frac{
			\langle \Environment, \Abort \rangle ~ \mathtt{main: copy}
				\xrightarrow[~~\Input~~]{0} 
			\langle \Environment, \Abort \rangle ~ \mathtt{main: nop}
		}{
			\begin{array}{l}
				\langle \Environment, \Abort \rangle ~ \mathtt{main: copy; par(t1:\{copy; pause;}	\\
				\mathtt{s=3;\},t2:\{copy; s=4;\})}
			\end{array}
				\xrightarrow[~~\Input~~]{\bot} 
			\begin{array}{l}
				\langle \Environment, \Abort \rangle ~ \mathtt{main: par(t1:\{copy; pause;}			\\
				\mathtt{s=3;\},t2:\{copy; s=4;\})}
			\end{array}
		}
\end{equation*}

\noindent
Step 2: Apply the (\ref{forec:par-1}) rule. Additionally, apply
the (\ref{forec:seq-right}) and (\ref{forec:copy}) rules to both threads.
\begin{equation*}
	\frac{
		\dfrac{
				\langle \Environment, \Abort \rangle ~ \mathtt{t1: copy}
					\xrightarrow[~~\Input~~]{0} 
				\langle \Environment^1, \Abort \rangle ~ \mathtt{t1: nop}
			}{
				\begin{array}{l}
					\langle \Environment, \Abort \rangle ~ \mathtt{t1: copy;}				\\
					\mathtt{pause; s=3}
				\end{array}
					\xrightarrow[~~\Input~~]{\bot} 
				\begin{array}{l}
					\langle \Environment^1, \Abort \rangle ~ \mathtt{t1:}					\\
					\mathtt{pause; s=3}
				\end{array}
			}
			\qquad
		\dfrac{
				\langle \Environment, \Abort \rangle ~ \mathtt{t2: copy}
					\xrightarrow[~~\Input~~]{0} 
				\langle \Environment^2, \Abort \rangle ~ \mathtt{t2: nop}
			}{
				\begin{array}{l}
					\langle \Environment, \Abort \rangle ~ \mathtt{t2:}						\\
					\mathtt{copy; s=4;}
				\end{array}
					\xrightarrow[~~\Input~~]{\bot} 
				\begin{array}{l}
					\langle \Environment^2, \Abort \rangle ~ \mathtt{t2:}					\\
					\mathtt{s=4}
				\end{array}
			}
		}{
			\begin{array}{l}
				\langle \Environment, \Abort \rangle ~ \mathtt{main: par(t1:\{copy; pause;}	\\
				\mathtt{s=3;\},t2:\{copy; s=4;\})}
			\end{array}
				\xrightarrow[~~\Input~~]{\bot} 
			\begin{array}{l}
				\langle \Environment^3, \Abort \rangle ~ \mathtt{main: par(t1:\{pause;}		\\
				\mathtt{s=3;\},t2:\{s=4;\})}
			\end{array}
		}
\end{equation*}

\noindent
Step 3: Apply the (\ref{forec:tick}) and (\ref{forec:par-7}) 
rules. Additionally, apply the (\ref{forec:seq-right}) and (\ref{forec:pause}) rules to the 
first thread and the (\ref{forec:assign-shared}) rule to the second thread. The program completes a 
tick.
\begin{equation*}
	\frac{
		\dfrac{
			\dfrac{
					\langle \Environment^3, \Abort \rangle ~ \mathtt{t1: pause}
						\xrightarrow[~~\Input~~]{1} 
					\langle \Environment^3, \Abort \rangle ~ \mathtt{t1: copy}
				}{
					\begin{array}{l}
						\langle \Environment^3, \Abort \rangle ~ \mathtt{t1:}			\\
						\mathtt{pause; s=3}
					\end{array}
							\xrightarrow[~~\Input~~]{1}
					\begin{array}{l}
						\langle \Environment^3, \Abort \rangle ~ \mathtt{t1:}			\\
						\mathtt{copy; s=3}
					\end{array}
				}
				\qquad
			\dfrac{
					s \in \lbrace s \rbrace
				}{
					\langle \Environment^3, \Abort \rangle ~ \mathtt{t2: s=4;}
						\xrightarrow[~~\Input~~]{0} 
					\langle \Environment^4, \Abort \rangle ~ \mathtt{t2: nop}
				}
			}{
				\begin{array}{l}
					\langle \Environment^3, \Abort \rangle ~ \mathtt{main: par(t1:}		\\
					\mathtt{\{pause; s=3;\},t2:\{s=4;\})}
				\end{array}
					\xrightarrow[~~\Input~~]{1} 
				\langle \Environment^5, \Abort \rangle ~ \mathtt{main: par(t1:\{copy; s=3;\}, t2:\{nop;\})}
			}
		}{
			\begin{array}{l}
				\langle \Environment^3, \Abort \rangle ~ \mathtt{main: par(t1:}			\\
				\mathtt{\{pause; s=3;\},t2:\{s=4;\})}
			\end{array}
				\xrightarrow[~~\Input~~]{1} 
			\langle \Environment^6, \Abort \rangle ~ \mathtt{main: par(t1:\{copy; s=3;\}, t2:\{nop;\})}
		}
\end{equation*}

\noindent
Step 4: Apply the (\ref{forec:par-3}) rule. Additionally, apply the (\ref{forec:seq-right}) and 
(\ref{forec:copy}) rules to the first thread and the (\ref{forec:nop}) rule to the second thread.
\begin{equation*}
	\frac{
		\dfrac{
				\langle \Environment^6, \Abort \rangle ~ \mathtt{t1: copy}
					\xrightarrow[~~\Input~~]{0} 
				\langle \Environment^7, \Abort \rangle ~ \mathtt{t1: nop}
			}{
				\langle \Environment^6, \Abort \rangle ~ \mathtt{t1: copy; s=3}
						\xrightarrow[~~\Input~~]{\bot}
				\langle \Environment^7, \Abort \rangle ~ \mathtt{t1: s=3}
			}
			\qquad
		\twolines{
			~
			}{
				\langle \Environment^6, \Abort \rangle ~ \mathtt{t2: nop}
					\xrightarrow[~~\Input~~]{0} 
				\langle \Environment^6, \Abort \rangle ~ \mathtt{t2: }
			}
		}{
			\langle \Environment^6, \Abort \rangle ~ \mathtt{main: par(t1:\{copy; s=3;\}, t2:\{nop;\})}
				\xrightarrow[~~\Input~~]{\bot} 
			\langle \Environment^7, \Abort \rangle ~ \mathtt{main: par(t1:\{s=3;\}, t2:\{nop;\})}
		}
\end{equation*}

\noindent
Step 5: Apply the (\ref{forec:par-5}) rule. Additionally, apply the (\ref{forec:assign-shared}) 
rule to the first thread and the (\ref{forec:nop}) rule to the second thread.
\begin{equation*}
	\frac{
		\dfrac{
				s \in \lbrace s \rbrace
			}{
				\langle \Environment^7, \Abort \rangle ~ \mathtt{t1: s=3}
					\xrightarrow[~~\Input~~]{0} 
				\langle \Environment^8, \Abort \rangle ~ \mathtt{t1: nop}
			}
			\qquad
		\twolines{
			~
			}{
				\langle \Environment^7, \Abort \rangle ~ \mathtt{t2: nop}
					\xrightarrow[~~\Input~~]{0} 
				\langle \Environment^7, \Abort \rangle ~ \mathtt{t2: }
			}
		}{
			\langle \Environment^7, \Abort \rangle ~ \mathtt{main: par(t1:\{s=3;\}, t2:\{nop;\})}
				\xrightarrow[~~\Input~~]{\bot} 
			\langle \Environment^9, \Abort \rangle ~ \mathtt{main: copy}
		}
\end{equation*}

\noindent
Step 6: Apply the (\ref{forec:tick}) and (\ref{forec:copy}) rules.
The program terminates.
\begin{equation*}
	\frac{
			\langle \Environment^9, \Abort \rangle ~ \mathtt{main: copy}
				\xrightarrow[~~\Input~~]{0} 
			\langle \Environment^9, \Abort \rangle ~ \mathtt{main: nop}
		}{
			\langle \Environment^9, \Abort \rangle ~ \mathtt{main: copy}
				\xrightarrow[~~\Input~~]{0} 
			\langle \Environment^{10}, \Abort \rangle ~ \mathtt{main: nop}
		}
\end{equation*}
