\subsubsection{Program 5}
\begin{lstlisting}[style=snippet]
int x=1;
void main(void) {
  abort {
    weak abort {
      pause;
    } when immediate (x == 0);
  } when (x == 1);
}
\end{lstlisting}
This program is structurally translated to:
\begin{lstlisting}[style=snippet]
int x=1;
void main(void) {
  abort a_0 { // Strong and delayed abort
    addAbort(a_0,0); 
    preempt(a_0);
    abort a_1 {  // Weak and immediate abort
      addAbort(a_1,1); 
      preempt(a_1); 
      pause; 
      preempt(a_0);
    } when (x == 0);
  } when (x == 1);
}
\end{lstlisting}
The initial program state, $\langle State \rangle$,
and its derivatives are defined in Table~\ref{table:forec_program_5}.
\newline

\begin{table}[t]
	\centering
	\renewcommand{\arraystretch}{1.25}
	
	\begin{tabular}{| l | c | c | c | c | c |}
		\hline
		\textbf{State}						& \textbf{\textit{A}}			& \textbf{\textit{I}} 		& \textbf{\textit{P}} 	& \textbf{\textit{C}}	 					& \textbf{\textit{E}}										\\ \hline
		$\langle State \rangle$				& $\lbrace main \rbrace$		& $\lbrace ~ \rbrace$		& $\lbrace ~ \rbrace$	& $\lbrace ~ \rbrace$						& $\lbrace \Global \to \lbrace x \to 0 \rbrace \rbrace$		\\ \cline{1-1} \cline{5-5}
		$\langle State^1 \rangle$			&								&							&						& $\lbrace a\_0 \to 0 \rbrace$				& 															\\ \cline{1-1} \cline{5-5}
		$\langle State^2 \rangle$			&								&							&						& $\lbrace a\_0 \to 0, a\_1 \to 0 \rbrace$	& 															\\ \cline{1-1} \cline{5-5}
		$\langle State^3 \rangle$			& 								& 							&						& $\lbrace a\_0 \to 1, a\_1 \to 0 \rbrace$	& 															\\ \cline{1-2}
		$\langle State^4 \rangle$			& $\lbrace ~ \rbrace$			& 							&						&											& 															\\ \hline
	\end{tabular}
	
	\caption{Definition of the initial program state and its derivatives for Program 5.}
	\label{table:forec_program_5}
\end{table}

\noindent
Step 1: Apply the (\ref{forec:not-aborted}), (\ref{forec:seq-right}) and 
(\ref{forec:del-abort}) rules on thread \verb$main$.
\begin{equation*}
	\frac{
		\dfrac{
				\langle State \rangle ~ \text{main: addAbort(a\_0,0)}
					\xrightarrow[~~\Input~~]{0}
				\langle State^1 \rangle ~ \text{main: nop}
			}{
				\begin{array}{l}
					\langle State \rangle ~ \text{main:}		\\
					\text{addAbort(a\_0,0); preempt(a\_0);}		\\
					\text{abort a\_1 \{}						\\
					\text{~~~addAbort(a\_1,1); preempt(a\_1);}	\\
					\text{~~~pause; preempt(a\_0);}				\\
					\text{\} when (x == 0);}					
				\end{array}
					\xrightarrow[~~\Input~~]{\bot} 
				\begin{array}{l}
					\langle State^1 \rangle ~ \text{main:}		\\
					\text{preempt(a\_0);}						\\
					\text{abort a\_1 \{}						\\
					\text{~~~addAbort(a\_1,1); preempt(a\_1);}	\\
					\text{~~~pause; preempt(a\_0);}				\\
					\text{\} when (x == 0);}					
				\end{array}
			}
		}{
			\begin{array}{l}
				\langle State \rangle ~ \text{main: abort a\_0 \{}	\\
				\text{~~~addAbort(a\_0,0); preempt(a\_0);}			\\
				\text{~~~abort a\_1 \{}								\\
				\text{~~~~~~addAbort(a\_1,1); preempt(a\_1);}		\\
				\text{~~~~~~pause; preempt(a\_0);}					\\
				\text{~~~\} when (x == 0);}							\\
				\text{\} when (x == 1);}
			\end{array}
				\xrightarrow[~~\Input~~]{\bot} 
			\begin{array}{l}
				\langle State^1 \rangle ~ \text{main: abort a\_0 \{}\\
				\text{~~~preempt(a\_0);}							\\
				\text{~~~abort a\_1 \{}								\\
				\text{~~~~~~addAbort(a\_1,1); preempt(a\_1);}		\\
				\text{~~~~~~pause; preempt(a\_0);}					\\
				\text{~~~\} when (x == 0);}							\\
				\text{\} when (x == 1);}
			\end{array}
		}
\end{equation*}

\noindent
Step 2: Apply the (\ref{forec:not-aborted}), (\ref{forec:seq-right}) and 
(\ref{forec:not-preempt}) rules on thread \verb$main$.
\begin{equation*}
	\frac{
		\dfrac{
				\langle State^1 \rangle ~ \text{main: preempt(a\_0)}
					\xrightarrow[~~\Input~~]{0}
				\langle State^1 \rangle ~ \text{main: nop}
			}{
				\begin{array}{l}
					\langle State^1 \rangle ~ \text{main: preempt(a\_0);}	\\
					\text{abort a\_1 \{}									\\
					\text{~~~addAbort(a\_1,1); preempt(a\_1);}				\\
					\text{~~~pause; preempt(a\_0);}							\\
					\text{\} when (x == 0);}					
				\end{array}
					\xrightarrow[~~\Input~~]{\bot} 
				\begin{array}{l}
					\langle State^1 \rangle ~ \text{main:}		\\
					\text{abort a\_1 \{}						\\
					\text{~~~addAbort(a\_1,1); preempt(a\_1);}	\\
					\text{~~~pause; preempt(a\_0);}				\\
					\text{\} when (x == 0);}					
				\end{array}
			}
		}{
			\begin{array}{l}
				\langle State^1 \rangle ~ \text{main: abort a\_0 \{}	\\
				\text{~~~preempt(a\_0);}								\\
				\text{~~~abort a\_1 \{}									\\
				\text{~~~~~~addAbort(a\_1,1); preempt(a\_1);}			\\
				\text{~~~~~~pause; preempt(a\_0);}						\\
				\text{~~~\} when (x == 0);}								\\
				\text{\} when (x == 1);}
			\end{array}
				\xrightarrow[~~\Input~~]{\bot} 
			\begin{array}{l}
				\langle State^1 \rangle ~ \text{main: abort a\_0 \{}\\
				\text{~~~abort a\_1 \{}								\\
				\text{~~~~~~addAbort(a\_1,1); preempt(a\_1);}		\\
				\text{~~~~~~pause; preempt(a\_0);}					\\
				\text{~~~\} when (x == 0);}							\\
				\text{\} when (x == 1);}
			\end{array}
		}
\end{equation*}
\newpage 

\noindent
Step 3: Apply the (\ref{forec:not-aborted}), (\ref{forec:not-aborted}), (\ref{forec:seq-right}) and 
(\ref{forec:imm-abort}) rules on thread \verb$main$.
\begin{equation*}
	\frac{
		\dfrac{
			\dfrac{
					\langle State^1 \rangle ~ \text{main: addAbort(a\_1,1)}
						\xrightarrow[~~\Input~~]{0}
					\langle State^2 \rangle ~ \text{main: nop}
				}{
					\begin{array}{l}
						\langle State^1 \rangle ~ \text{main:}		\\
						\text{~~~addAbort(a\_1,1); preempt(a\_1);}	\\
						\text{~~~pause; preempt(a\_0);}				
					\end{array}
						\xrightarrow[~~\Input~~]{\bot} 
					\begin{array}{l}
						\langle State^2 \rangle ~ \text{main:}		\\
						\text{~~~preempt(a\_1);}					\\
						\text{~~~pause; preempt(a\_0);}				
					\end{array}
				}
			}{
				\begin{array}{l}
					\langle State^1 \rangle ~ \text{main:}		\\
					\text{abort a\_1 \{}						\\
					\text{~~~addAbort(a\_1,1); preempt(a\_1);}	\\
					\text{~~~pause; preempt(a\_0);}				\\
					\text{\} when (x == 0);}					
				\end{array}
					\xrightarrow[~~\Input~~]{\bot} 
				\begin{array}{l}
					\langle State^2 \rangle ~ \text{main:}		\\
					\text{abort a\_1 \{}						\\
					\text{~~~preempt(a\_1);}					\\
					\text{~~~pause; preempt(a\_0);}				\\
					\text{\} when (x == 0);}					
				\end{array}
			}
		}{
			\begin{array}{l}
				\langle State^1 \rangle ~ \text{main: abort a\_0 \{}	\\
				\text{~~~abort a\_1 \{}									\\
				\text{~~~~~~addAbort(a\_1,1); preempt(a\_1);}			\\
				\text{~~~~~~pause; preempt(a\_0);}						\\
				\text{~~~\} when (x == 0);}								\\
				\text{\} when (x == 1);}
			\end{array}
				\xrightarrow[~~\Input~~]{\bot} 
			\begin{array}{l}
				\langle State^2 \rangle ~ \text{main: abort a\_0 \{}\\
				\text{~~~abort a\_1 \{}								\\
				\text{~~~~~~preempt(a\_1);}							\\
				\text{~~~~~~pause; preempt(a\_0);}					\\
				\text{~~~\} when (x == 0);}							\\
				\text{\} when (x == 1);}
			\end{array}
		}
\end{equation*}

\noindent
Step 4: Apply the (\ref{forec:not-aborted}), (\ref{forec:not-aborted}), (\ref{forec:seq-right}) and 
(\ref{forec:not-preempt}) rules on thread \verb$main$.
\begin{equation*}
	\frac{
		\dfrac{
			\dfrac{
					\langle State^2 \rangle ~ \text{main: preempt(a\_1)}
						\xrightarrow[~~\Input~~]{0}
					\langle State^2 \rangle ~ \text{main: nop}
				}{
					\begin{array}{l}
						\langle State^2 \rangle ~ \text{main:}		\\
						\text{~~~preempt(a\_1);}					\\
						\text{~~~pause; preempt(a\_0);}				
					\end{array}
						\xrightarrow[~~\Input~~]{\bot} 
					\begin{array}{l}
						\langle State^2 \rangle ~ \text{main:}		\\
						\text{~~~pause; preempt(a\_0);}				
					\end{array}
				}
			}{
				\begin{array}{l}
					\langle State^2 \rangle ~ \text{main:}		\\
					\text{abort a\_1 \{}						\\
					\text{~~~preempt(a\_1);}					\\
					\text{~~~pause; preempt(a\_0);}				\\
					\text{\} when (x == 0);}					
				\end{array}
					\xrightarrow[~~\Input~~]{\bot} 
				\begin{array}{l}
					\langle State^2 \rangle ~ \text{main:}		\\
					\text{abort a\_1 \{}						\\
					\text{~~~pause; preempt(a\_0);}				\\
					\text{\} when (x == 0);}					
				\end{array}
			}
		}{
			\begin{array}{l}
				\langle State^2 \rangle ~ \text{main: abort a\_0 \{}	\\
				\text{~~~abort a\_1 \{}									\\
				\text{~~~~~~preempt(a\_1);}								\\
				\text{~~~~~~pause; preempt(a\_0);}						\\
				\text{~~~\} when (x == 0);}								\\
				\text{\} when (x == 1);}
			\end{array}
				\xrightarrow[~~\Input~~]{\bot} 
			\begin{array}{l}
				\langle State^2 \rangle ~ \text{main: abort a\_0 \{}\\
				\text{~~~abort a\_1 \{}								\\
				\text{~~~~~~pause; preempt(a\_0);}					\\
				\text{~~~\} when (x == 0);}							\\
				\text{\} when (x == 1);}
			\end{array}
		}
\end{equation*}
\newpage

\noindent
Step 5: Apply the (\ref{forec:not-aborted}), (\ref{forec:not-aborted}), (\ref{forec:seq-left}) and 
(\ref{forec:pause}) rules on thread \verb$main$.
\begin{equation*}
	\frac{
		\dfrac{
			\dfrac{
					\langle State^2 \rangle ~ \text{main: pause}
						\xrightarrow[~~\Input~~]{1}
					\langle State^2 \rangle ~ \text{main: nop}
				}{
					\begin{array}{l}
						\langle State^2 \rangle ~ \text{main: pause; preempt(a\_0);}				
					\end{array}
						\xrightarrow[~~\Input~~]{1} 
					\begin{array}{l}
						\langle State^2 \rangle ~ \text{main: nop; preempt(a\_0);}				
					\end{array}
				}
			}{
				\begin{array}{l}
					\langle State^2 \rangle ~ \text{main:}		\\
					\text{abort a\_1 \{}						\\
					\text{~~~pause; preempt(a\_0);}				\\
					\text{\} when (x == 0);}					
				\end{array}
					\xrightarrow[~~\Input~~]{1} 
				\begin{array}{l}
					\langle State^2 \rangle ~ \text{main:}		\\
					\text{abort a\_1 \{}						\\
					\text{~~~nop; preempt(a\_0);}				\\
					\text{\} when (x == 0);}					
				\end{array}
			}
		}{
			\begin{array}{l}
				\langle State^2 \rangle ~ \text{main: abort a\_0 \{}	\\
				\text{~~~abort a\_1 \{}									\\
				\text{~~~~~~pause; preempt(a\_0);}						\\
				\text{~~~\} when (x == 0);}								\\
				\text{\} when (x == 1);}
			\end{array}
				\xrightarrow[~~\Input~~]{1} 
			\begin{array}{l}
				\langle State^2 \rangle ~ \text{main: abort a\_0 \{}\\
				\text{~~~abort a\_1 \{}								\\
				\text{~~~~~~nop; preempt(a\_0);}					\\
				\text{~~~\} when (x == 0);}							\\
				\text{\} when (x == 1);}
			\end{array}
		}
\end{equation*}

\noindent
Step 6: Apply the (\ref{forec:global-tick}) rule on thread
\verb$main$. A global tick occurs.
\begin{equation*}
	\frac{
			\text{main} \xrightarrow{~~1~~} \text{main}
		}{
			\langle State^2 \rangle \xrightarrow{~~~~~} \langle State^3 \rangle
		}
\end{equation*}

\noindent
Step 7: Apply the (\ref{forec:not-aborted}), (\ref{forec:not-aborted}), (\ref{forec:seq-right}) and 
(\ref{forec:nop}) rules on thread \verb$main$.
\begin{equation*}
	\frac{
		\dfrac{
			\dfrac{
					\langle State^3 \rangle ~ \text{main: nop}
						\xrightarrow[~~\Input~~]{0}
					\langle State^3 \rangle ~ \text{main: }
				}{
					\begin{array}{l}
						\langle State^3 \rangle ~ \text{main: nop; preempt(a\_0);}				
					\end{array}
						\xrightarrow[~~\Input~~]{\bot} 
					\begin{array}{l}
						\langle State^3 \rangle ~ \text{main: preempt(a\_0);}				
					\end{array}
				}
			}{
				\begin{array}{l}
					\langle State^3 \rangle ~ \text{main:}		\\
					\text{abort a\_1 \{}						\\
					\text{~~~nop; preempt(a\_0);}				\\
					\text{\} when (x == 0);}					
				\end{array}
					\xrightarrow[~~\Input~~]{1} 
				\begin{array}{l}
					\langle State^3 \rangle ~ \text{main:}		\\
					\text{abort a\_1 \{}						\\
					\text{~~~preempt(a\_0);}					\\
					\text{\} when (x == 0);}					
				\end{array}
			}
		}{
			\begin{array}{l}
				\langle State^3 \rangle ~ \text{main: abort a\_0 \{}	\\
				\text{~~~abort a\_1 \{}									\\
				\text{~~~~~~nop; preempt(a\_0);}						\\
				\text{~~~\} when (x == 0);}								\\
				\text{\} when (x == 1);}
			\end{array}
				\xrightarrow[~~\Input~~]{1} 
			\begin{array}{l}
				\langle State^3 \rangle ~ \text{main: abort a\_0 \{}\\
				\text{~~~abort a\_1 \{}								\\
				\text{~~~~~~preempt(a\_0);}							\\
				\text{~~~\} when (x == 0);}							\\
				\text{\} when (x == 1);}
			\end{array}
		}
\end{equation*}

\noindent
Step 8: Apply the (\ref{forec:aborted}), (\ref{forec:not-aborted}) and (\ref{forec:preempt})  
rules on thread \verb$main$.
\begin{equation*}
	\frac{
		\dfrac{
				\langle State^3 \rangle ~ \text{main: preempt(a\_0)}
					\xrightarrow[~~\Input~~]{a\_0}
				\langle State^3 \rangle ~ \text{main: nop}
			}{
				\begin{array}{l}
					\langle State^3 \rangle ~ \text{main:}		\\
					\text{abort a\_1 \{}						\\
					\text{~~~preempt(a\_0);}				\\
					\text{\} when (x == 0);}					
				\end{array}
					\xrightarrow[~~\Input~~]{a\_0} 
				\begin{array}{l}
					\langle State^3 \rangle ~ \text{main:}		\\
					\text{abort a\_1 \{}						\\
					\text{~~~nop;}					\\
					\text{\} when (x == 0);}					
				\end{array}
			}
		}{
			\begin{array}{l}
				\langle State^3 \rangle ~ \text{main: abort a\_0 \{}	\\
				\text{~~~abort a\_1 \{}									\\
				\text{~~~~~~preempt(a\_0);}								\\
				\text{~~~\} when (x == 0);}								\\
				\text{\} when (x == 1);}
			\end{array}
				\xrightarrow[~~\Input~~]{0} 
			\begin{array}{l}
				\langle State^3 \rangle ~ \text{main: nop}
			\end{array}
		}
\end{equation*}

\noindent
Step 9: Apply the (\ref{forec:term3}) rule on thread \verb$main$.
The program terminates.
\begin{equation*}
	\frac{
			\text{main} \xrightarrow{~~0~~} \text{main}
		}{
			\langle State^3 \rangle \xrightarrow{~~~~~} \langle State^4 \rangle
		}
\end{equation*}
