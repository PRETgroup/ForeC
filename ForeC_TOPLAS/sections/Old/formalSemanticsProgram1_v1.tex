\subsection{Example Executions}
This section provides examples of how ForeC programs
execute with the rewrite rules. A possible execution 
for each example program is given as a sequence of 
rewrites. 

\subsubsection{Program 1}
\begin{lstlisting}[style=snippet]
void main(void) {
  int x=1;
  par( {int u=2;}, {int v=3;}, {int w=4;} );
}
\end{lstlisting}
This program is structurally translated to:
\begin{lstlisting}[style=snippet]
int u, v, w, x;
void main(void) {
  x=1;
  par( t1:{u=2;}, t2:{par( t3:{v=3;}, t4:{w=4;} );} );
}
\end{lstlisting}
Initially, the set of \verb$abort$ identifiers \Abort{}
is empty. The program's environment \Environment{} and its
derivatives are defined in Table~\ref{figure:forec_program_1}.
\newline

\begin{figure}
	\centering
	$$\begin{array}{l l l l l l l}
		 \Environment		&=& \lbrace \Global \to \lbrace u		&, v		&, w		&, x~~~~~~ \rbrace \rbrace	\\
		 \Environment^1		&=& \lbrace \Global \to \lbrace u		&, v		&, w		&, x \to 1 \rbrace \rbrace	\\
		 \Environment^2		&=& \lbrace \Global \to \lbrace u \to 2	&, v		&, w		&, x \to 1 \rbrace \rbrace	\\
		 \Environment^3		&=& \lbrace \Global \to \lbrace u 		&, v \to 3	&, w		&, x \to 1 \rbrace \rbrace	\\
		 \Environment^4		&=& \lbrace \Global \to \lbrace u 		&, v		&, w \to 4	&, x \to 1 \rbrace \rbrace	\\
		 \Environment^5		&=& \lbrace \Global \to \lbrace u 		&, v \to 3	&, w \to 4	&, x \to 1 \rbrace \rbrace	\\
		 \Environment^6		&=& \lbrace \Global \to \lbrace u \to 2 &, v \to 3	&, w \to 4	&, x \to 1 \rbrace \rbrace
	\end{array}$$
	
	\caption{Definition of the initial program environment and its derivatives for Program 1.}
	\label{figure:forec_program_1}
\end{figure}

\noindent
Step 1: Apply the (\ref{forec:seq-left}) and (\ref{forec:assign-private}) rules. 
\begin{equation*}
	\frac{
		\dfrac{
				x \notin \Shared(\Environment[\Global])
			}{
				\langle \Environment, \Abort \rangle ~ \mathtt{main: x=1}
					\xrightarrow[~~\Input~~]{\bot} 
				\langle \Environment^1, \Abort \rangle ~ \mathtt{main: nop}
			}
		}{
			\begin{array}{l}
				\langle \Environment, \Abort \rangle ~ \mathtt{main: x=1;~par(t1:\{u=2;\},}	\\
				\mathtt{t2:\{par(t3:\{v=3;\}, t4:\{w=4;\} );\});}				\\
			\end{array}
				\xrightarrow[~~\Input~~]{\bot} 
			\begin{array}{l}
				\langle \Environment^1, \Abort \rangle ~ \mathtt{main: nop;~par(t1:\{u=2;\},}	\\
				\mathtt{t2:\{par( t3:\{v=3;\},t4:\{w=4;\} );\} );}				\\
			\end{array}
		}
\end{equation*}

\noindent
Step 2: Apply the (\ref{forec:seq-right}) and (\ref{forec:nop}) rules. 
\begin{equation*}
	\frac{
			\langle \Environment^1, \Abort \rangle ~ \text{main: nop}
				\xrightarrow[~~\Input~~]{0} 
			\langle \Environment^1, \Abort \rangle ~ \text{main:}
		}{
			\begin{array}{l}
				\langle \Environment^1, \Abort \rangle ~ \mathtt{main: nop;~par(t1:\{u=2;\},}	\\
				\mathtt{t2:\{par(t3:\{v=3;\}, t4:\{w=4;\} );\});}				\\
			\end{array}
				\xrightarrow[~~\Input~~]{\bot} 
			\begin{array}{l}
				\langle \Environment^1, \Abort \rangle ~ \mathtt{main: par(t1:\{u=2;\},}		\\
				\mathtt{t2:\{par( t3:\{v=3;\},t4:\{w=4;\} );\} );}				\\
			\end{array}
		}
\end{equation*}

\noindent
Step 3: Apply the (\ref{forec:par-1}) rule. Additionally, apply the 
(\ref{forec:assign-private}) rule to the first thread, and the 
(\ref{forec:par-1}) and (\ref{forec:assign-private})
rules to the second thread.
\begin{equation*}
	\frac{
		\dfrac{
				u \notin \Shared(\Environment[\Global])
			}{
				\langle \Environment^1, \Abort \rangle ~ \mathtt{t1: u=2}
					\xrightarrow[~~\Input~~]{\bot} 
				\langle \Environment^2, \Abort \rangle ~ \mathtt{t1: nop}
			}
			\qquad
			\qquad
			\twolines{
				~
			}{
				term1
			}
		}{
			\begin{array}{l}
				\langle \Environment^1, \Abort \rangle ~ \mathtt{main: par(t1:\{u=2;\},}	\\
				\mathtt{t2:\{par(t3:\{v=3;\}, t4:\{w=4;\} );\});}			\\
			\end{array}
				\xrightarrow[~~\Input~~]{\bot} 
			\begin{array}{l}
				\langle \Environment^6, \Abort \rangle ~ \mathtt{main: par(t1:\{nop;\},}	\\
				\mathtt{t2:\{par( t3:\{nop;\},t4:\{nop;\} );\} );}			\\
			\end{array}
		}
\end{equation*}
where $term1$ represents:
\begin{equation*}
	\frac{
		\dfrac{
				v \notin \Shared(\Environment[\Global])
			}{
				\langle \Environment^1, \Abort \rangle ~ \mathtt{t3:v=3}
					\xrightarrow[~~\Input~~]{\bot} 
				\langle \Environment^3, \Abort \rangle ~ \mathtt{t3:nop}
			}
			\quad
		\dfrac{
				w \notin \Shared(\Environment[\Global])
			}{
				\langle \Environment^1, \Abort \rangle ~ \mathtt{t4:w=4}
					\xrightarrow[~~\Input~~]{\bot} 
				\langle \Environment^4, \Abort \rangle ~ \mathtt{t4:nop}
			}
		}{
			\langle \Environment^1, \Abort \rangle ~ \mathtt{t2:par(t3:\{v=3;\},t4:\{w=4;\} )}
				\xrightarrow[~~\Input~~]{\bot} 
			\langle \Environment^5, \Abort \rangle ~ \mathtt{t2:par( t3:\{nop;\},t4:\{nop;\} )}
		}
\end{equation*}

\noindent
Step 4: Apply the (\ref{forec:par-2}) rule. Additionally, apply the 
(\ref{forec:nop}) rule to the first thread, and the (\ref{forec:par-5})
and (\ref{forec:nop}) rules to the second thread.
\begin{equation*}
	\frac{
			\langle \Environment^6, \Abort \rangle ~ \mathtt{t1: nop}
				\xrightarrow[~~\Input~~]{0} 
			\langle \Environment^6, \Abort \rangle ~ \mathtt{t1:}
			\qquad
			\qquad
			term2
		}{
			\begin{array}{l}
				\langle \Environment^6, \Abort \rangle ~ \mathtt{main: par(t1:\{nop;\},}	\\
				\mathtt{t2:\{par( t3:\{nop;\},t4:\{nop;\} );\} );}							\\
			\end{array}
				\xrightarrow[~~\Input~~]{\bot} 
			\langle \Environment^6, \Abort \rangle ~ \mathtt{main: par(t1:\{nop;\}, t2:\{nop;\} );}
		}
\end{equation*}
where $term2$ represents:
\begin{equation*}
	\frac{
			\langle \Environment^6, \Abort \rangle ~ \mathtt{t3:nop}
				\xrightarrow[~~\Input~~]{0} 
			\langle \Environment^6, \Abort \rangle ~ \mathtt{t3:}
			\qquad
			\langle \Environment^6, \Abort \rangle ~ \mathtt{t4:nop}
				\xrightarrow[~~\Input~~]{0} 
			\langle \Environment^6, \Abort \rangle ~ \mathtt{t4:}
		}{
			\langle \Environment^6, \Abort \rangle ~ \mathtt{t2:par(t3:\{nop;\},t4:\{nop;\} )}
				\xrightarrow[~~\Input~~]{\bot} 
			\langle \Environment^6, \Abort \rangle ~ \mathtt{t2:nop}
		}
\end{equation*}

\noindent
Step 5: Apply the (\ref{forec:par-5}) rule and the \verb$par$
instruction terminates.
\begin{equation*}
	\frac{
			\langle \Environment^6, \Abort \rangle ~ \mathtt{t1: nop}
				\xrightarrow[~~\Input~~]{0} 
			\langle \Environment^6, \Abort \rangle ~ \mathtt{t1:}
			\qquad
			\langle \Environment^6, \Abort \rangle ~ \mathtt{t2: nop}
				\xrightarrow[~~\Input~~]{0} 
			\langle \Environment^6, \Abort \rangle ~ \mathtt{t2:}
		}{
			\langle \Environment^6, \Abort \rangle ~ \mathtt{main: par(t1:\{nop;\}, t2:\{nop;\} );}
				\xrightarrow[~~\Input~~]{\bot} 
			\langle \Environment^6, \Abort \rangle ~ \mathtt{main: nop;}
		}
\end{equation*}

\noindent
Step 6: Apply the (\ref{forec:global-tick}) rule.
The program terminates.
\begin{equation*}
	\frac{
			\mathtt{main} = main
			\quad
			\langle \Environment^6, \Abort \rangle ~ \mathtt{main: nop}
				\xrightarrow[~~\Input~~]{0} 
			\langle \Environment^6, \Abort \rangle ~ \mathtt{main: }
		}{
			\langle \Environment^6, \Abort \rangle ~ \mathtt{main: nop;}
				\xrightarrow[~~\Input~~]{0} 
			\langle \Environment^6, \Abort \rangle ~ \mathtt{main: }
		}
\end{equation*}
