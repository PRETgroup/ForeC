\subsubsection{Program 4}
\begin{lstlisting}[style=snippet]
int x=1;
void main(void) {
  weak abort {
    par( t0:{pause;} , t1:{pause;} );
    x=x+1;
  } when immediate (x == 1);
}
\end{lstlisting}
This program is structurally translated to:
\begin{lstlisting}[style=snippet]
int x=1;
void main(void) {
  abort a {  // Weak and immediate abort
    addAbort(a,1);
    par( t0:{preempt(a); pause;} ,
         t1:{preempt(a); pause;} );
    preempted(a);
    x=x+1;
  } when (x == 1);
}
\end{lstlisting}
The initial program state, $\langle State \rangle$,
and its derivatives are defined in Table~\ref{table:forec_program_4}.
\newline

\begin{table}[t]
	\centering
	\renewcommand{\arraystretch}{1.25}
	
	\begin{tabular}{| l | c | c | c | c | c |}
		\hline
		\textbf{State}						& \textbf{\textit{A}}			& \textbf{\textit{I}} 		& \textbf{\textit{P}} 	& \textbf{\textit{C}} 		& \textbf{\textit{E}}										\\ \hline
		$\langle State \rangle$				& $\lbrace main \rbrace$		& $\lbrace ~ \rbrace$		& $\lbrace ~ \rbrace$	& $\lbrace ~ \rbrace$		& $\lbrace \Global \to \lbrace x \to 1 \rbrace \rbrace$		\\ \cline{1-1} \cline{5-5}
		$\langle State^1 \rangle$			&								&							&						& $\lbrace a \to 1 \rbrace$	& 															\\ \cline{1-3}
		$\langle State^2 \rangle$			& $\lbrace t0, t1 \rbrace$		& $\lbrace main \rbrace$	&						&							& 															\\ \cline{1-2}
		$\langle State^3 \rangle$			& $\lbrace t1 \rbrace$			& 							&						&							& 															\\ \cline{1-3}
		$\langle State^4 \rangle$			& $\lbrace main \rbrace$		& $\lbrace ~ \rbrace$		&						&							& 															\\ \cline{1-2}
		$\langle State^5 \rangle$			& $\lbrace ~ \rbrace$			& 							& 						&							& 															\\ \hline
	\end{tabular}
	
	\caption{Definition of the initial program state and its derivatives for Program 4.}
	\label{table:forec_program_4}
\end{table}

\noindent
Step 1: Apply the (\ref{forec:not-aborted}), (\ref{forec:seq-right}) and (\ref{forec:imm-abort})
rules on thread \verb$main$.
\begin{equation*}
	\frac{
		\dfrac{
				\langle State \rangle ~ \text{main: addAbort(a,1)}
					\xrightarrow[~~\Input~~]{0}
				 \langle State^1 \rangle ~ \text{main: nop}
			}{
				\begin{array}{l}
					\langle State \rangle ~ \text{main: addAbort(a,1);}	\\
					\text{par( t0:\{preempt(a); pause;\} ,}				\\
					\text{~~~~~~t1:\{preempt(a); pause;\} );}			\\
					\text{preempted(a); x=x+1;}							
				\end{array}
					\xrightarrow[~~\Input~~]{\bot} 
				\begin{array}{l}
					\langle State^1 \rangle ~ \text{main:}		\\
					\text{par( t0:\{preempt(a); pause;\} ,}		\\
					\text{~~~~~~t1:\{preempt(a); pause;\} );}	\\
					\text{preempted(a); x=x+1;}							
				\end{array}
			}
		}{
			\begin{array}{l}
				\langle State \rangle ~ \text{main: abort a \{}	\\
				\text{~~~addAbort(a,1);}						\\
				\text{~~~par( t0:\{preempt(a); pause;\} ,}		\\
				\text{~~~~~~~~~t1:\{preempt(a); pause;\} );}	\\
				\text{~~~preempted(a); x=x+1;}					\\
				\text{\} when (x == 1);}						
			\end{array}
				\xrightarrow[~~\Input~~]{\bot} 
			\begin{array}{l}
				\langle State^1 \rangle ~ \text{main: abort a \{}	\\
				\text{~~~par( t0:\{preempt(a); pause;\} ,}			\\
				\text{~~~~~~~~~t1:\{preempt(a); pause;\} );}		\\
				\text{~~~preempted(a); x=x+1;}						\\
				\text{\} when (x == 1);}							
			\end{array}
		}
\end{equation*}

\noindent
Step 2: Apply the (\ref{forec:not-aborted}), (\ref{forec:seq-right}) and 
(\ref{forec:par}) rules on thread \verb$main$. Threads \verb$t0$ and 
\verb$t1$ are forked.
\begin{equation*}
	\frac{
		\dfrac{
				\begin{array}{l}
					\langle State^1 \rangle ~ \text{main: par( t0:\{preempt(a); pause;\} ,}	 \\
					\text{~~~~~~~~~~~~~~~~~~~~~~~~~~~t1:\{preempt(a); pause;\} );}
				\end{array}
					\xrightarrow[~~\Input~~]{\bot} 
				\langle State^2 \rangle ~ \text{main: nop}
			}{
				\begin{array}{l}
					\langle State^1 \rangle ~ \text{main:}		\\
					\text{par( t0:\{preempt(a); pause;\} ,}		\\
					\text{~~~~~~t1:\{preempt(a); pause;\} );}	\\
					\text{preempted(a); x=x+1;}					
				\end{array}
					\xrightarrow[~~\Input~~]{\bot} 
				\begin{array}{l}
					\langle State^2 \rangle ~ \text{main:}			\\
					\text{nop; preempted(a); x=x+1;}				
				\end{array}
			}
		}{
			\begin{array}{l}
				\langle State^1 \rangle ~ \text{main: abort a \{}	\\
				\text{~~~par( t0:\{preempt(a); pause;\} ,}			\\
				\text{~~~~~~~~~t1:\{preempt(a); pause;\} );}		\\
				\text{~~~preempted(a); x=x+1;}						\\
				\text{\} when (x == 1);}							
			\end{array}
				\xrightarrow[~~\Input~~]{\bot} 
			\begin{array}{l}
				\langle State^2 \rangle ~ \text{main: abort a \{}	\\
				\text{~~~nop; preempted(a); x=x+1;}					\\
				\text{\} when (x == 1);}							
			\end{array}
		}
\end{equation*}

\noindent
Step 3: Apply the (\ref{forec:seq-left}) and (\ref{forec:preempt}) rules 
on thread \verb$t0$. Preemption is taken.
\begin{equation*}
	\frac{
		\langle State^2 \rangle ~ \text{t0: preempt(a);}
			\xrightarrow[~~\Input~~]{~~a~~} 
		\langle State^2 \rangle ~ \text{t0: nop;}
	}{
		\langle State^2 \rangle ~ \text{t0: preempt(a); pause;}
			\xrightarrow[~~\Input~~]{~~a~~} 
		\langle State^2 \rangle ~ \text{t0: nop; pause;}
	}
\end{equation*}

\noindent
Step 4: Apply the (\ref{forec:term1}) rule on thread \verb$t0$. 
This terminates \verb$t0$.
\begin{equation*}
	\frac{
			\text{t0} \xrightarrow{~~a~~} \text{t0}
		}{
			\langle State^2 \rangle \xrightarrow{~~~~~} \langle State^3 \rangle
		}
\end{equation*}

\noindent
Step 5: Apply the (\ref{forec:seq-left}) and (\ref{forec:preempt}) rules
on thread \verb$t1$. Preemption is taken.
\begin{equation*}
	\frac{
		\langle State^3 \rangle ~ \text{t1: preempt(a);}
			\xrightarrow[~~\Input~~]{~~a~~} 
		\langle State^3 \rangle ~ \text{t1: nop;}
	}{
		\langle State^3 \rangle ~ \text{t1: preempt(a); pause;}
			\xrightarrow[~~\Input~~]{~~a~~} 
		\langle State^3 \rangle ~ \text{t1: nop; pause;}
	}
\end{equation*}

\noindent
Step 6: Apply the (\ref{forec:term2}) rule on thread \verb$t1$. 
This terminates \verb$t0$ and a join occurs.
\begin{equation*}
	\frac{
			\text{t1} \xrightarrow{~~a~~} \text{t1}
		}{
			\langle State^3 \rangle \xrightarrow{~~~~~} \langle State^4 \rangle
		}
\end{equation*}

\noindent
Step 7: Apply the (\ref{forec:not-aborted}), (\ref{forec:seq-right}) and 
(\ref{forec:nop}) rules on thread \verb$main$.
\begin{equation*}
	\frac{
		\dfrac{
				\langle State^4 \rangle ~ \text{main: nop}
					\xrightarrow[~~\Input~~]{0} 
				\langle State^4 \rangle ~ \text{main: }
			}{
				\langle State^4 \rangle ~ \text{main: nop; preempted(a); x=x+1;}					
					\xrightarrow[~~\Input~~]{\bot} 
				\langle State^4 \rangle ~ \text{main: preempted(a); x=x+1;}				
			}
		}{
			\begin{array}{l}
				\langle State^4 \rangle ~ \text{main: abort a \{}	\\
				\text{~~~nop; preempted(a); x=x+1;}					\\
				\text{\} when (x == 1);}							
			\end{array}
				\xrightarrow[~~\Input~~]{\bot} 
			\begin{array}{l}
				\langle State^4 \rangle ~ \text{main: abort a \{}	\\
				\text{~~~preempted(a); x=x+1;}						\\
				\text{\} when (x == 1);}							
			\end{array}
		}
\end{equation*}

\noindent
Step 8: Apply the (\ref{forec:aborted}), (\ref{forec:seq-left}) and 
(\ref{forec:preempted}) rules on thread \verb$main$. Preemption is taken.
\begin{equation*}
	\frac{
		\dfrac{
				\langle State^4 \rangle ~ \text{main: preempted(a)}
					\xrightarrow[~~\Input~~]{a} 
				\langle State^4 \rangle ~ \text{main: nop}
			}{
				\langle State^4 \rangle ~ \text{main: preempted(a); x=x+1;}					
					\xrightarrow[~~\Input~~]{a} 
				\langle State^4 \rangle ~ \text{main: nop; x=x+1;}				
			}
		}{
			\begin{array}{l}
				\langle State^4 \rangle ~ \text{main: abort a \{}	\\
				\text{~~~preempted(a); x=x+1;}						\\
				\text{\} when (x == 1);}							
			\end{array}
				\xrightarrow[~~\Input~~]{0} 
			\begin{array}{l}
				\langle State^4 \rangle ~ \text{main: nop;}
			\end{array}
		}
\end{equation*}

\noindent
Step 9: Apply the (\ref{forec:term3}) rule on thread \verb$main$.
The program terminates.
\begin{equation*}
	\frac{
			\text{main} \xrightarrow{~~0~~} \text{main}
		}{
			\langle State^4 \rangle \xrightarrow{~~~~~} \langle State^5 \rangle
		}
\end{equation*}


