\subsubsection{Program 4}
\begin{lstlisting}[style=snippet]
shared int a=2 combine all with plus;
shared int b=2 combine new with plus;
void main(void) {
  a++;
  par({a=b;},{b=a;});
  a++;
}
\end{lstlisting}
This program is structurally translated to:
\begin{lstlisting}[style=snippet]
int a=2;
int b=2;
void main(void) {
  copy;
  a++;
  par(t1:{copy; a=b;},t2:{copy; b=a;});
  a++;
}
\end{lstlisting}
The predicate $\Shared(main)$ is $\lbrace a \rbrace$.
The predicates $\Shared(t1)$, $\Shared(t2)$ and 
$\Shared(\Global)$ are all $\lbrace a, b \rbrace$.
Initially, the set of preemption statuses \Abort{}
is $\emptyset$. The program's environment \Environment{} and its 
derivatives are defined in Figure~\ref{figure:forec_program_4}.
\newline

\begin{figure}
	\centering
	$$\begin{array}{l l l}
		\Environment		&=& \left \lbrace
									\Global \to \lbrace a \to (2, \mathtt{pre}), b \to (2, \mathtt{pre}) \rbrace
								\right \rbrace	\\
		\Environment^1		&=& \left \lbrace
									\Global \to \lbrace a \to (2, \mathtt{pre}), b \to (2, \mathtt{pre}) \rbrace, 
									main \to \lbrace a \to (2, \mathtt{pre}) \rbrace
								\right \rbrace	\\
		\Environment^2		&=& \left \lbrace
									\Global \to \lbrace a \to (2, \mathtt{pre}), b \to (2, \mathtt{pre}) \rbrace, 
									main \to \lbrace a \to (3, \mathtt{mod}) \rbrace
								\right \rbrace	\\
		\Environment^3		&=& \left \lbrace
									\Global \to \lbrace a \to (2, \mathtt{pre}), b \to (2, \mathtt{pre}) \rbrace, 
									main \to \lbrace a \to (3, \mathtt{mod}) \rbrace,
									t1 \to \lbrace a \to (3, \mathtt{pre}), b \to (2, \mathtt{pre}) \rbrace
								\right \rbrace	\\
		\Environment^4		&=& \left \lbrace
									\Global \to \lbrace a \to (2, \mathtt{pre}), b \to (2, \mathtt{pre}) \rbrace, 
									main \to \lbrace a \to (3, \mathtt{mod}) \rbrace,
									t2 \to \lbrace a \to (3, \mathtt{pre}), b \to (2, \mathtt{pre}) \rbrace
								\right \rbrace	\\
		\Environment^5		&=& \lbrace
									\Global \to \lbrace a \to (2, \mathtt{pre}), b \to (2, \mathtt{pre}) \rbrace, 
									main \to \lbrace a \to (3, \mathtt{mod}) \rbrace,	\\
							& &	  ~ t1 \to \lbrace a \to (3, \mathtt{pre}), b \to (2, \mathtt{pre}) \rbrace,
									t2 \to \lbrace a \to (3, \mathtt{pre}), b \to (2, \mathtt{pre}) \rbrace
								\rbrace	\\
		\Environment^6		&=& \lbrace
									\Global \to \lbrace a \to (2, \mathtt{pre}), b \to (2, \mathtt{pre}) \rbrace, 
									main \to \lbrace a \to (3, \mathtt{mod}) \rbrace,	\\
							& &	  ~ t1 \to \lbrace a \to (2, \mathtt{mod}), b \to (2, \mathtt{pre}) \rbrace,
									t2 \to \lbrace a \to (3, \mathtt{pre}), b \to (2, \mathtt{pre}) \rbrace
								\rbrace	\\
		\Environment^7		&=& \lbrace
									\Global \to \lbrace a \to (2, \mathtt{pre}), b \to (2, \mathtt{pre}) \rbrace, 
									main \to \lbrace a \to (3, \mathtt{mod}) \rbrace,	\\
							& &	  ~ t1 \to \lbrace a \to (3, \mathtt{pre}), b \to (2, \mathtt{pre}) \rbrace,
									t2 \to \lbrace a \to (3, \mathtt{pre}), b \to (3, \mathtt{mod}) \rbrace
								\rbrace	\\
		\Environment^{8M}	&=& \lbrace
									\Global \to \lbrace a \to (2, \mathtt{pre}), b \to (2, \mathtt{pre}) \rbrace, 
									main \to \lbrace a \to (3, \mathtt{mod}) \rbrace,	\\
							& &	  ~ t1 \to \lbrace a \to (2, \mathtt{mod}), b \to (2, \mathtt{pre}) \rbrace,
									t2 \to \lbrace a \to (3, \mathtt{pre}), b \to (3, \mathtt{mod}) \rbrace
								\rbrace	\\
		\Environment^8		&=& \lbrace
									\Global \to \lbrace a \to (2, \mathtt{pre}), b \to (2, \mathtt{pre}) \rbrace, 
									main \to \lbrace a \to (5, \mathtt{cmb}), b \to (3, \mathtt{cmb}) \rbrace
								\rbrace	\\
		\Environment^9		&=& \lbrace
									\Global \to \lbrace a \to (2, \mathtt{pre}), b \to (2, \mathtt{pre}) \rbrace, 
									main \to \lbrace a \to (6, \mathtt{mod}), b \to (3, \mathtt{cmb}) \rbrace
								\rbrace	\\
		\Environment^{10}	&=& \lbrace
									\Global \to \lbrace a \to (6, \mathtt{pre}), b \to (3, \mathtt{pre}) \rbrace
								\rbrace	\\
	\end{array}$$
	
	\caption{Definition of the initial program state and its derivatives for Program 4.}
	\label{figure:forec_program_4}
\end{figure}

\noindent
Step 1: Apply the (\ref{forec:seq-right}) and (\ref{forec:copy}) rules. 
\begin{equation*}
	\frac{
			\langle \Environment, \Abort \rangle ~ \mathtt{main: copy}
				\xrightarrow[~~\Input~~]{0} 
			\langle \Environment^1, \Abort \rangle ~ \mathtt{main: nop}
		}{
			\begin{array}{l}
				\langle \Environment, \Abort \rangle ~ \mathtt{main: copy; a++; par(t1:\{copy;}	\\
				\mathtt{a=b;\},t2:\{copy; b=a;\}); a++}											\\
			\end{array}
				\xrightarrow[~~\Input~~]{\bot} 
			\begin{array}{l}
				\langle \Environment^1, \Abort \rangle ~ \mathtt{main: a++; par(t1:\{copy;}		\\
				\mathtt{a=b;\},t2:\{copy; b=a;\}); a++}											\\
			\end{array}
		}
\end{equation*}

\noindent
Step 2: Apply the (\ref{forec:seq-right}) and (\ref{forec:assign-shared}) rules. 
\begin{equation*}
	\frac{
		\dfrac{
				a \in \lbrace a, b \rbrace
			}{
				\langle \Environment^1, \Abort \rangle ~ \mathtt{main: a++}
					\xrightarrow[~~\Input~~]{0} 
				\langle \Environment^2, \Abort \rangle ~ \mathtt{main: nop}
			}
		}{
			\begin{array}{l}
				\langle \Environment^1, \Abort \rangle ~ \mathtt{main: a++; par(t1:\{copy; a=b;\},}	\\
				\mathtt{t2:\{copy; b=a;\}); a++}													\\
			\end{array}
				\xrightarrow[~~\Input~~]{\bot} 
			\begin{array}{l}
				\langle \Environment^2, \Abort \rangle ~ \mathtt{main: par(t1:\{copy; a=b;\},}		\\
				\mathtt{t2:\{copy; b=a;\}); a++}													\\
			\end{array}
		}
\end{equation*}

\noindent
Step 3: Apply the (\ref{forec:seq-left}) and (\ref{forec:par-1}) rules. 
Additionally, apply the (\ref{forec:seq-right}) and (\ref{forec:copy}) rules
to both threads.
\begin{equation*}
	\frac{
		\dfrac{
			\dfrac{
					\langle \Environment^2, \Abort \rangle ~ \mathtt{t1: copy}
						\xrightarrow[~~\Input~~]{0} 
					\langle \Environment^3, \Abort \rangle ~ \mathtt{t1: nop}
				}{
					\langle \Environment^2, \Abort \rangle ~ \mathtt{t1: copy; a=b}
						\xrightarrow[~~\Input~~]{\bot} 
					\langle \Environment^3, \Abort \rangle ~ \mathtt{t1: a=b}
				}
				\qquad
			\dfrac{
					\langle \Environment^2, \Abort \rangle ~ \mathtt{t2: copy}
						\xrightarrow[~~\Input~~]{0} 
					\langle \Environment^4, \Abort \rangle ~ \mathtt{t2: nop}
				}{
					\langle \Environment^2, \Abort \rangle ~ \mathtt{t2: copy; b=a}
						\xrightarrow[~~\Input~~]{\bot} 
					\langle \Environment^4, \Abort \rangle ~ \mathtt{t2: b=a}
				}
			}{
				\begin{array}{l}
					\langle \Environment^2, \Abort \rangle ~ \mathtt{main: par(t1:\{copy; a=b;\},}	\\
					\mathtt{t2:\{copy; b=a;\})}														\\
				\end{array}
					\xrightarrow[~~\Input~~]{\bot} 
				\begin{array}{l}
					\langle \Environment^5, \Abort \rangle ~ \mathtt{main: par(t1:\{a=b;\},}		\\
					\mathtt{t2:\{b=a;\})}															\\
				\end{array}
			}
		}{
			\begin{array}{l}
				\langle \Environment^2, \Abort \rangle ~ \mathtt{main: par(t1:\{copy; a=b;\},}	\\
				\mathtt{t2:\{copy; b=a;\}); a++}												\\
			\end{array}
				\xrightarrow[~~\Input~~]{\bot} 
			\begin{array}{l}
				\langle \Environment^5, \Abort \rangle ~ \mathtt{main: par(t1:\{a=b;\},}		\\
				\mathtt{t2:\{b=a;\}); a++}														\\
			\end{array}
		}
\end{equation*}

\noindent
Step 4: Apply the (\ref{forec:seq-right}) and (\ref{forec:par-5}) rules. 
Additionally, apply the (\ref{forec:seq-right}) and (\ref{forec:assign-shared}) rules
to both threads.
\begin{equation*}
	\frac{
		\dfrac{
			\dfrac{
					a \in \lbrace a, b \rbrace
				}{
					\langle \Environment^5, \Abort \rangle ~ \mathtt{t1: a=b}
						\xrightarrow[~~\Input~~]{0} 
					\langle \Environment^6, \Abort \rangle ~ \mathtt{t1: nop}
				}
				\qquad
			\dfrac{
					b \in \lbrace a, b \rbrace
				}{
					\langle \Environment^5, \Abort \rangle ~ \mathtt{t2: b=a}
						\xrightarrow[~~\Input~~]{0} 
					\langle \Environment^7, \Abort \rangle ~ \mathtt{t2: nop}
				}
			}{
				\langle \Environment^5, \Abort \rangle ~ \mathtt{main: par(t1:\{a=b;\},t2:\{b=a;\})}
					\xrightarrow[~~\Input~~]{\bot} 
				\langle \Environment^8, \Abort \rangle ~ \mathtt{main: copy}
			}
		}{
			\langle \Environment^5, \Abort \rangle ~ \mathtt{main: par(t1:\{a=b;\},t2:\{b=a;\}); a++}
				\xrightarrow[~~\Input~~]{\bot} 
			\langle \Environment^8, \Abort \rangle ~ \mathtt{main: copy; a++}
		}
\end{equation*}

\noindent
Step 5: Apply the (\ref{forec:seq-right}) and (\ref{forec:copy}) rules. 
\begin{equation*}
	\frac{
			\langle \Environment^8, \Abort \rangle ~ \mathtt{main: copy}
				\xrightarrow[~~\Input~~]{0} 
			\langle \Environment^8, \Abort \rangle ~ \mathtt{main: nop}
		}{
			\langle \Environment^8, \Abort \rangle ~ \mathtt{main: copy; a++}
				\xrightarrow[~~\Input~~]{\bot} 
			\langle \Environment^8, \Abort \rangle ~ \mathtt{main: a++}
		}
\end{equation*}

\noindent
Step 6: Apply the (\ref{forec:tick}) and (\ref{forec:assign-shared}) rules. 
The program terminates.
\begin{equation*}
	\frac{
		\dfrac{
				a \in \lbrace a, b \rbrace
			}{
				\langle \Environment^8, \Abort \rangle ~ \mathtt{main: a++}
					\xrightarrow[~~\Input~~]{0} 
				\langle \Environment^9, \Abort \rangle ~ \mathtt{main: nop}
			}
		}{
			\langle \Environment^8, \Abort \rangle ~ \mathtt{main: a++}
				\xrightarrow[~~\Input~~]{0} 
			\langle \Environment^{10}, \Abort \rangle ~ \mathtt{main: nop}
		}
\end{equation*}
