\subsubsection{Program 3}
\begin{lstlisting}[style=snippet]
int plus(int a, int b) {return a + b;}
shared:plus int x=0;
void main(void) {
  par( 
    t0:{x=x+1;} , 
    t1:{par( t2:{x=x+2;} , t3:{x=x+3;} );}
  );
}
\end{lstlisting}
The initial program state, $\langle State \rangle$,
and its derivatives are defined in Table~\ref{table:forec_program_3}.
\newline

\begin{table}[t]
	\centering
	\renewcommand{\arraystretch}{1.25}
	
	\begin{tabular}{| l | c | c | c | c | c |}
		\hline
		\textbf{State}						& \textbf{\textit{A}}			& \textbf{\textit{I}} 		& \textbf{\textit{P}} 	& \textbf{\textit{C}} 		& \textbf{\textit{E}}										\\ \hline
		$\langle State \rangle$				& $\lbrace main \rbrace$		& $\lbrace ~ \rbrace$		& $\lbrace ~ \rbrace$	& $\lbrace ~ \rbrace$		& $\lbrace \Global \to \lbrace x \to 0 \rbrace \rbrace$		\\ \cline{1-3}
		$\langle State^1 \rangle$			& $\lbrace t0,t1 \rbrace$		& $\lbrace main \rbrace$	& 						& 							& 															\\ \cline{1-1} \cline{6-6}
		$\langle State^2 \rangle$			&								&							&						&							& $\begin{Bmatrix} \begin{array}{r l} 
																																									\Global &\to \lbrace x \to 0 \rbrace,	\\ 
																																									t0 &\to \lbrace x \to 1 \rbrace 
																																								\end{array} \end{Bmatrix}$								\\ \cline{1-2}
		$\langle State^3 \rangle$			& $\lbrace t1 \rbrace$			& 							& 						& 							& 															\\ \cline{1-3}
		$\langle State^4 \rangle$			& $\lbrace t2,t3 \rbrace$		& $\lbrace main,t1 \rbrace$	& 						& 							& 															\\ \cline{1-1} \cline{6-6}
		$\langle State^5 \rangle$			&								&							&						&							& $\begin{Bmatrix} \begin{array}{r l} 
																																									\Global &\to \lbrace x \to 0 \rbrace,	\\ 
																																									t0 &\to \lbrace x \to 1 \rbrace, 		\\
																																									t2 &\to \lbrace x \to 2 \rbrace			
																																								\end{array} \end{Bmatrix}$								\\ \cline{1-2}
		$\langle State^6 \rangle$			& $\lbrace t3 \rbrace$			& 							& 						& 							& 															\\ \cline{1-1} \cline{6-6}
		$\langle State^7 \rangle$			&								&							&						&							& $\begin{Bmatrix} \begin{array}{r l} 
																																									\Global &\to \lbrace x \to 0 \rbrace,	\\ 
																																									t0 &\to \lbrace x \to 1 \rbrace, 		\\
																																									t2 &\to \lbrace x \to 2 \rbrace,			\\
																																									t3 &\to \lbrace x \to 3 \rbrace
																																								\end{array} \end{Bmatrix}$								\\ \cline{1-3} \cline{6-6}
		$\langle State^8 \rangle$			& $\lbrace t1 \rbrace$			& $\lbrace main \rbrace$	&						&							& $\begin{Bmatrix} \begin{array}{r l} 
																																									\Global &\to \lbrace x \to 0 \rbrace,	\\ 
																																									t0 &\to \lbrace x \to 1 \rbrace, 		\\
																																									t1 &\to \lbrace x \to 5 \rbrace
																																								\end{array} \end{Bmatrix}$								\\ \cline{1-3} \cline{6-6}
		$\langle State^9 \rangle$			& $\lbrace main \rbrace$		& $\lbrace ~ \rbrace$		&						&							& $\begin{Bmatrix} \begin{array}{r l} 
																																									\Global &\to \lbrace x \to 0 \rbrace,	\\ 
																																									main &\to \lbrace x \to 6 \rbrace 		
																																								\end{array} \end{Bmatrix}$								\\ \cline{1-2} \cline{6-6}
		$\langle State^{10} \rangle$		& $\lbrace ~ \rbrace$			&							&						&							& $\lbrace \Global \to \lbrace x \to 6 \rbrace \rbrace$		\\ \hline
	\end{tabular}
	
	\caption{Definition of the initial program state and its derivatives for Program 3.}
	\label{table:forec_program_3}
\end{table}

\noindent
Step 1: Apply the (\ref{forec:par}) rule on thread \verb$main$.
Threads \verb$t0$ and \verb$t1$ are forked.
\begin{equation*}
	\begin{array}{l}
		\langle State \rangle ~ \text{main: par( t0:\{x=x+1;\} ,}	\\
		\text{~~~~~~~~~~~~~~~~~~~~~~~~~t1:\{par( t2:\{x=x+2;\} , t3:\{x=x+3;\} );\} );}
	\end{array}
		\xrightarrow[~~\Input~~]{\bot} 
	\langle State^1 \rangle ~ \text{main: nop}
\end{equation*}

\noindent
Step 2: Apply the (\ref{forec:assign-shared}) rule on thread \verb$t0$.
\begin{equation*}
	\langle State^1 \rangle ~ \text{t0: x=x+1}
		\xrightarrow[~~\Input~~]{0} 
	\langle State^2 \rangle ~ \text{t0: nop}
\end{equation*}

\noindent
Step 3: Apply the (\ref{forec:term1}) rule on thread \verb$t0$. 
This terminates \verb$t0$.
\begin{equation*}
	\frac{
			\text{t0} \xrightarrow{~~0~~} \text{t0}
		}{
			\langle State^2 \rangle \xrightarrow{~~~~~} \langle State^3 \rangle
		}
\end{equation*}

\noindent
Step 4: Apply the (\ref{forec:par}) rule on thread \verb$t1$. 
Threads \verb$t2$ and \verb$t3$ are forked.
\begin{equation*}
	\langle State^3 \rangle ~ \text{t1: par( t2:\{x=x+2;\} , t3:\{x=x+3;\} );}
		\xrightarrow[~~\Input~~]{\bot} 
	\langle State^4 \rangle ~ \text{t1: nop}
\end{equation*}


\noindent
Step 5: Apply the (\ref{forec:assign-shared}) rule on thread \verb$t2$.
\begin{equation*}
	\langle State^4 \rangle ~ \text{t2: x=x+2}
		\xrightarrow[~~\Input~~]{0} 
	\langle State^5 \rangle ~ \text{t2: nop}
\end{equation*}

\noindent
Step 6: Apply the (\ref{forec:term1}) rule on thread \verb$t2$. 
This terminates \verb$t2$.
\begin{equation*}
	\frac{
			\text{t2} \xrightarrow{~~0~~} \text{t2}
		}{
			\langle State^5 \rangle \xrightarrow{~~~~~} \langle State^6 \rangle
		}
\end{equation*}

\noindent
Step 7: Apply the (\ref{forec:assign-shared}) rule on thread \verb$t3$.
\begin{equation*}
	\langle State^6 \rangle ~ \text{t3: x=x+3}
		\xrightarrow[~~\Input~~]{0} 
	\langle State^7 \rangle ~ \text{t3: nop}
\end{equation*}

\noindent
Step 8: Apply the (\ref{forec:term2}) rule on thread \verb$t3$.
This terminates \verb$t3$ and a join occurs.
\begin{equation*}
	\frac{
			\text{t3} \xrightarrow{~~0~~} \text{t3}
		}{
			\langle State^7 \rangle \xrightarrow{~~~~~} \langle State^8 \rangle
		}
\end{equation*}

\noindent
Step 9: Apply the (\ref{forec:nop}) rule on thread \verb$t1$.
\begin{equation*}
	\langle State^8 \rangle ~ \text{t1: nop}
		\xrightarrow[~~\Input~~]{0} 
	\langle State^8 \rangle ~ \text{t1:}
\end{equation*}

\noindent
Step 10: Apply the (\ref{forec:term2}) rule on thread \verb$t1$.
This terminates \verb$t1$ and a join occurs.
\begin{equation*}
	\frac{
			\text{t1} \xrightarrow{~~0~~} \text{t1}
		}{
			\langle State^8 \rangle \xrightarrow{~~~~~} \langle State^9 \rangle
		}
\end{equation*}

\noindent
Step 11: Apply the (\ref{forec:nop}) rule on thread \verb$main$.
\begin{equation*}
	\langle State^9 \rangle ~ \text{main: nop}
		\xrightarrow[~~\Input~~]{0} 
	\langle State^9 \rangle ~ \text{main:}
\end{equation*}

\noindent
Step 12: Apply the (\ref{forec:term3}) rule on thread \verb$main$.
The program terminates.
\begin{equation*}
	\frac{
			\text{main} \xrightarrow{~~0~~} \text{main}
		}{
			\langle State^9 \rangle \xrightarrow{~~~~~} \langle State^{10} \rangle
		}
\end{equation*}



