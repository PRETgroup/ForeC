\subsubsection{Program 2}
\begin{lstlisting}[style=snippet]
int x=1;
void main(void) {
  weak abort {  // Weak and delayed abort
    x=2; pause;
    x=3; pause;
  } when (x > 0);
}
\end{lstlisting}
This program is structurally translated to:
\begin{lstlisting}[style=snippet]
int x=1;
void main(void) {
  abort a {  // Weak and delayed abort
    addAbort(a,0);
    x=2; preempt(a); pause;
    x=3; preempt(a); pause;
  } when (x > 0);
}
\end{lstlisting}
The initial program state, $\langle State \rangle$,
and its derivatives are defined in Table~\ref{table:forec_program_2}.
\newline

\begin{table}[t]
	\centering
	\renewcommand{\arraystretch}{1.25}
	
	\begin{tabular}{| l | c | c | c | c | c |}
		\hline
		\textbf{State}						& \textbf{\textit{A}}			& \textbf{\textit{I}} 		& \textbf{\textit{P}} 	& \textbf{\textit{C}} 		& \textbf{\textit{E}}										\\ \hline
		$\langle State \rangle$				& $\lbrace main \rbrace$		& $\lbrace ~ \rbrace$		& $\lbrace ~ \rbrace$	& $\lbrace ~ \rbrace$		& $\lbrace \Global \to \lbrace x \to 1 \rbrace \rbrace$		\\ \cline{1-1} \cline{5-5}
		$\langle State^1 \rangle$			& 								& 							& 						& $\lbrace a \to 0 \rbrace$	& 															\\ \cline{1-1} \cline{6-6}
		$\langle State^2 \rangle$			& 								& 							& 						& 							& $\lbrace \Global \to \lbrace x \to 2 \rbrace \rbrace$		\\ \cline{1-1} \cline{5-5}
		$\langle State^3 \rangle$			& 								& 							& 						& $\lbrace a \to 1 \rbrace$	& 															\\ \cline{1-1} \cline{6-6}
		$\langle State^4 \rangle$			& 								& 							& 						& 							& $\lbrace \Global \to \lbrace x \to 3 \rbrace \rbrace$		\\ \cline{1-2}
		$\langle State^5 \rangle$			& $\lbrace ~ \rbrace$			& 							& 						& 							& 															\\ \hline
	\end{tabular}
	
	\caption{Definition of the initial program state and its derivatives for Program 2.}
	\label{table:forec_program_2}
\end{table}

\noindent
Step 1: Apply the (\ref{forec:not-aborted}), (\ref{forec:seq-right}), 
and (\ref{forec:del-abort}) rules. The delayed \verb$abort$ is initialised.
\begin{equation*}
	\frac{
		\dfrac{
				\langle State \rangle ~ \text{main: addAbort(a,0)}
					\xrightarrow[~~\Input~~]{0}
				 \langle State^1 \rangle ~ \text{main: nop}
			}{
				\begin{array}{l}
					\langle State \rangle ~ \text{main:}	\\
					\text{addAbort(a,0);}					\\
					\text{x=2; preempt(a); pause;}			\\
					\text{x=3; preempt(a); pause;}
				\end{array}
					\xrightarrow[~~\Input~~]{\bot} 
				\begin{array}{l}
					\langle State^1 \rangle ~ \text{main:}	\\
					\text{x=2; preempt(a); pause;}			\\
					\text{x=3; preempt(a); pause;}							
				\end{array}
			}
		}{
			\begin{array}{l}
				\langle State \rangle ~ \text{main: abort a \{}	\\
				\text{~~~addAbort(a,0);}						\\
				\text{~~~x=2; preempt(a); pause;}				\\
				\text{~~~x=3; preempt(a); pause;}				\\
				\text{\} when (x > 0);}
			\end{array}
				\xrightarrow[~~\Input~~]{\bot} 
			\begin{array}{l}
				\langle State^1 \rangle ~ \text{main: abort a \{}	\\
				\text{~~~x=2; preempt(a); pause;}					\\
				\text{~~~x=3; preempt(a); pause;}					\\
				\text{\} when (x > 0);}
			\end{array}
		}
\end{equation*}

\noindent
Step 2: Apply the (\ref{forec:not-aborted}), (\ref{forec:seq-right}), 
and (\ref{forec:assign-private}) rules. 
\begin{equation*}
	\frac{
		\dfrac{
				\langle State^1 \rangle ~ \text{main: x=2}
					\xrightarrow[~~\Input~~]{0}
				 \langle State^2 \rangle ~ \text{main: nop}
			}{
				\begin{array}{l}
					\langle State^1 \rangle ~ \text{main:}	\\
					\text{x=2; preempt(a); pause;}			\\
					\text{x=3; preempt(a); pause;}							
				\end{array}
					\xrightarrow[~~\Input~~]{\bot} 
				\begin{array}{l}
					\langle State^2 \rangle ~ \text{main:}	\\
					\text{preempt(a); pause;}				\\
					\text{x=3; preempt(a); pause;}							
				\end{array}
			}
		}{
			\begin{array}{l}
				\langle State^1 \rangle ~ \text{main: abort a \{}	\\
				\text{~~~x=2; preempt(a); pause;}					\\
				\text{~~~x=3; preempt(a); pause;}					\\
				\text{\} when (x > 0);}
			\end{array}
				\xrightarrow[~~\Input~~]{\bot} 
			\begin{array}{l}
				\langle State^2 \rangle ~ \text{main: abort a \{}	\\
				\text{~~~preempt(a); pause;}						\\
				\text{~~~x=3; preempt(a); pause;}					\\
				\text{\} when (x > 0);}
			\end{array}
		}
\end{equation*}

\noindent
Step 3: Apply the (\ref{forec:not-aborted}), (\ref{forec:seq-right}), 
and (\ref{forec:not-preempt}) rules. Preemption does not occur.
\begin{equation*}
	\frac{
		\dfrac{
				\langle State^2 \rangle ~ \text{main: preempt(a)}
					\xrightarrow[~~\Input~~]{0}
				 \langle State^2 \rangle ~ \text{main: nop}
			}{
				\begin{array}{l}
					\langle State^2 \rangle ~ \text{main:}	\\
					\text{preempt(a); pause;}				\\
					\text{x=3; preempt(a); pause;}							
				\end{array}
					\xrightarrow[~~\Input~~]{\bot} 
				\begin{array}{l}
					\langle State^2 \rangle ~ \text{main:}	\\
					\text{pause;}							\\
					\text{x=3; preempt(a); pause;}							
				\end{array}
			}
		}{
			\begin{array}{l}
				\langle State^2 \rangle ~ \text{main: abort a \{}	\\
				\text{~~~preempt(a); pause;}						\\
				\text{~~~x=3; preempt(a); pause;}					\\
				\text{\} when (x > 0);}
			\end{array}
				\xrightarrow[~~\Input~~]{\bot} 
			\begin{array}{l}
				\langle State^2 \rangle ~ \text{main: abort a \{}	\\
				\text{~~~pause;}									\\
				\text{~~~x=3; preempt(a); pause;}					\\
				\text{\} when (x > 0);}
			\end{array}
		}
\end{equation*}

\noindent
Step 4: Apply the (\ref{forec:not-aborted}), (\ref{forec:seq-left}), 
and (\ref{forec:pause}) rules. 
\begin{equation*}
	\frac{
		\dfrac{
				\langle State^2 \rangle ~ \text{main: pause}
					\xrightarrow[~~\Input~~]{1}
				 \langle State^2 \rangle ~ \text{main: nop}
			}{
				\begin{array}{l}
					\langle State^2 \rangle ~ \text{main:}	\\
					\text{pause;}							\\
					\text{x=3; preempt(a); pause;}							
				\end{array}
					\xrightarrow[~~\Input~~]{1} 
				\begin{array}{l}
					\langle State^2 \rangle ~ \text{main:}	\\
					\text{nop;}								\\
					\text{x=3; preempt(a); pause;}							
				\end{array}
			}
		}{
			\begin{array}{l}
				\langle State^2 \rangle ~ \text{main: abort a \{}	\\
				\text{~~~pause;}									\\
				\text{~~~x=3; preempt(a); pause;}					\\
				\text{\} when (x > 0);}
			\end{array}
				\xrightarrow[~~\Input~~]{1} 
			\begin{array}{l}
				\langle State^2 \rangle ~ \text{main: abort a \{}	\\
				\text{~~~nop;}										\\
				\text{~~~x=3; preempt(a); pause;}					\\
				\text{\} when (x > 0);}
			\end{array}
		}
\end{equation*}

\noindent
Step 5: Apply the (\ref{forec:global-tick}) rule.
A global tick occurs.
\begin{equation*}
	\frac{
			\text{main} \xrightarrow{~~1~~} \text{main}
		}{
			\langle State^2 \rangle \xrightarrow{~~~~~} \langle State^3 \rangle
		}
\end{equation*}

\noindent
Step 6: Apply the (\ref{forec:not-aborted}), (\ref{forec:seq-right}), 
and (\ref{forec:nop}) rules. 
\begin{equation*}
	\frac{
		\dfrac{
				\langle State^3 \rangle ~ \text{main: nop}
					\xrightarrow[~~\Input~~]{0}
				 \langle State^3 \rangle ~ \text{main: }
			}{
				\begin{array}{l}
					\langle State^3 \rangle ~ \text{main:}	\\
					\text{nop;}								\\
					\text{x=3; preempt(a); pause;}							
				\end{array}
					\xrightarrow[~~\Input~~]{\bot} 
				\begin{array}{l}
					\langle State^3 \rangle ~ \text{main:}	\\
					\text{x=3; preempt(a); pause;}							
				\end{array}
			}
		}{
			\begin{array}{l}
				\langle State^3 \rangle ~ \text{main: abort a \{}	\\
				\text{~~~nop;}										\\
				\text{~~~x=3; preempt(a); pause;}					\\
				\text{\} when (x > 0);}
			\end{array}
				\xrightarrow[~~\Input~~]{\bot} 
			\begin{array}{l}
				\langle State^3 \rangle ~ \text{main: abort a \{}	\\
				\text{~~~x=3; preempt(a); pause;}					\\
				\text{\} when (x > 0);}
			\end{array}
		}
\end{equation*}

\noindent
Step 7: Apply the (\ref{forec:not-aborted}), (\ref{forec:seq-right}), 
and (\ref{forec:assign-private}) rules. 
\begin{equation*}
	\frac{
		\dfrac{
				\langle State^3 \rangle ~ \text{main: x=3}
					\xrightarrow[~~\Input~~]{0}
				 \langle State^4 \rangle ~ \text{main: nop}
			}{
				\begin{array}{l}
					\langle State^3 \rangle ~ \text{main:}	\\
					\text{x=3; preempt(a); pause;}							
				\end{array}
					\xrightarrow[~~\Input~~]{\bot} 
				\begin{array}{l}
					\langle State^4 \rangle ~ \text{main:}	\\
					\text{preempt(a); pause;}							
				\end{array}
			}
		}{
			\begin{array}{l}
				\langle State^3 \rangle ~ \text{main: abort a \{}	\\
				\text{~~~x=3; preempt(a); pause;}					\\
				\text{\} when (x > 0);}
			\end{array}
				\xrightarrow[~~\Input~~]{\bot} 
			\begin{array}{l}
				\langle State^4 \rangle ~ \text{main: abort a \{}	\\
				\text{~~~preempt(a); pause;}						\\
				\text{\} when (x > 0);}
			\end{array}
		}
\end{equation*}

\noindent
Step 8: Apply the (\ref{forec:aborted}), (\ref{forec:seq-left}), 
and (\ref{forec:preempt}) rules. Preemption is taken.
\begin{equation*}
	\frac{
		\dfrac{
				\langle State^4 \rangle ~ \text{main: preempt(a)}
					\xrightarrow[~~\Input~~]{a}
				 \langle State^4 \rangle ~ \text{main: nop}
			}{
				\begin{array}{l}
					\langle State^4 \rangle ~ \text{main:}	\\
					\text{preempt(a); pause;}							
				\end{array}
					\xrightarrow[~~\Input~~]{a} 
				\begin{array}{l}
					\langle State^4 \rangle ~ \text{main:}	\\
					\text{nop; pause;}							
				\end{array}
			}
		}{
			\begin{array}{l}
				\langle State^4 \rangle ~ \text{main: abort a \{}	\\
				\text{~~~preempt(a); pause;}						\\
				\text{\} when (x > 0);}
			\end{array}
				\xrightarrow[~~\Input~~]{0} 
			\begin{array}{l}
				\langle State^4 \rangle ~ \text{main:}	\\
				\text{nop}
			\end{array}
		}
\end{equation*}

\noindent
Step 9: Apply the (\ref{forec:term3}) rule.
The program terminates.
\begin{equation*}
	\frac{
			\text{main} \xrightarrow{~~0~~} \text{main}
		}{
			\langle State^4 \rangle \xrightarrow{~~~~~} \langle State^5 \rangle
		}
\end{equation*}
