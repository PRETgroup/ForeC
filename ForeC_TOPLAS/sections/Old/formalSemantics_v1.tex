\section{The Formal Semantics of ForeC}
\label{sec:formalSemantics}

This section presents the formal semantics of ForeC as rewrite rules,
in the style of Structural Operation Semantics~\cite{semantics_sos}. 
The semantics is inspired by that of other synchronous programming
languages (Esterel~\cite{EsterelV7} and \pretc{}~\cite{pret_pretc} in particular).
We define the semantics on a set of primitive (kernel) ForeC statements
(see Table~\ref{table:forec_kernel}), from which the full set of ForeC 
statements is derived. The kernel statements are not used for compiling, 
but for simplifying the formal definition of ForeC. The kernel only 
considers a subset of the C language in order to focus on the semantics 
of parallel execution and shared variables. The translation of ForeC 
statements (Table~\ref{table:forec_semantics}) to the kernel statements 
(Table~\ref{table:forec_kernel}) is explained in 
Section~\ref{sec:formalSemantics_translations}.

\begin{table}[!h]
	\centering
	\renewcommand{\arraystretch}{1.25}
	
	\begin{tabular}{l l}
		\bf{Kernel Statement}													& \bf{Short Description}		\\ \hline
		\verb$nop$																& Empty statement				\\
		\verb$pause$															& Barrier synchronisation		\\
		\verb$shared:$ $cf$														& Shared type-qualifier			\\
		\var{}~\verb$=$~\expression{}											& Assignment					\\
		$f$\verb$;$~$g$															& Sequence						\\
		\verb$if$~(\expression{})~\{$f$\}~\verb$else$~\{$g$\}					& Conditional					\\
		\verb$while$~(\expression{})~\{$f$\}									& Loop							\\
		\verb$par$(\thread{}$_0$:\{$f_0$\},~$\dots$,~\thread{}$_n$:\{$f_n$\})	& Fork/join parallelism			\\
		\verb$abort$~$a$~\{$f$\}~\verb$when$~(\expression{})					& Preemption scope				\\
		\verb$addAbort$($a$, \imm{})											& Initialise a preemption		\\
		\verb$preempt$($a$)														& Take a preemption				\\
		\verb$preempted$(\thread{}$_0$,~$\dots$,~\thread{}$_n$)					& Propagate a preemption		\\
	\end{tabular}
	
	\caption{The ForeC kernel statements. Here, $cf$ is a combine function, \var{} is a variable, 
			 \expression{} and \imm{} are expressions, $f$ and $g$ are arbitrary compositions of kernel 
			 statements, $t_0$ to $t_n$ are thread identifiers, and $a$ is an abort identifier.}
	\label{table:forec_kernel}
\end{table}

We make the following assumptions about ForeC programs. 
(A1) All programs follow safety-critical coding practices,
as discussed in Section~\ref{sec:foreLanguage_programming}.
For example, all expressions are side-effect free, 
all memory allocations are static defined, and all loops are 
either bounded or always execute a \verb$pause$ in each iteration.
(A2) All programs are correctly typed. For example, all 
variables accessed by multiple threads are declared 
as shared variables. 
(A3) No recursive function calls or forking of threads.
Assumption A1 is to limit ourselves to a deterministic 
subset of the C language. Assumption A2 can be checked
and enforced statically. Assumption A3 prevents the unbounded 
execution of functions and threads, which can lead to unbounded 
memory use and execution time.

To simplify the presentation of the semantics, we 
assume that the following transformations have been performed on 
ForeC programs: (1) Inlining of functions at their call sites, 
(2) Renaming variables uniquely and hoisting their declarations 
up to the program's global scope, and (3) Replacing pointers 
with the actual variables they are referencing. 
Thus, the semantics can ignore: (1) Function calls, (2) Variable 
declarations, scoping and shadowing, and (3) pointers.
For example, the ForeC kernel program of 
Figure~\ref{fig:transform_forec_original} can be
transformed into the equivalent program of 
Figure~\ref{fig:transform_forec_refactored}.

\begin{figure}
	\subfloat[Original kernel program.] {
		\begin{minipage}[b]{0.46\textwidth}
			\lstinputlisting[style=full]{./code/transform_forec_original.forec}
			\label{fig:transform_forec_original}
		\end{minipage}
	}
	\hfill
	\subfloat[Transformed kernel program.] {
		\begin{minipage}[b]{0.45\textwidth}
			\lstinputlisting[style=full]{./code/transform_forec_refactored.forec}
			\label{fig:transform_forec_refactored}
		\end{minipage}
	}
	
	\caption{Example of transforming a ForeC kernel program.}
	\label{fig:transform_forec}
\end{figure}

This section continues by describing the notation of the 
rewrite rules. Then, predicates are defined for returning 
statically known properties about ForeC programs. 
Finally, the ForeC semantics are presented.


%-----------------------------------------------------------------------------

\subsection{Structural Translations}
\label{sec:formalSemantics_translations}

\begin{table}
	\centering
	\renewcommand{\arraystretch}{1.25}
	
	\begin{tabular}{| l | l |}
		\hline
		\bf{ForeC construct}														& \bf{Translation to kernel statement(s)}						\\ \hline
		\verb$input$ and \verb$output$												& Mapped to global variables.									\\ \hline
		\verb$pause$																& \verb$pause$													\\ \hline
		\verb$shared$ $\dots$ \verb$combine with$ $cf$								& \verb$shared:$ $cf$											\\ \hline
		\verb$par$( $st$ (, $st$)* )												& \verb$par$( \thread{}:\{$st$\} (, \thread{}:\{$st$\})* )		\\ \hline
		\verb$weak$?~\verb$abort$~\{$st$\}~\verb$when immediate$?~(\expression{})	& Mapped to \verb$abort$, \verb$addAbort$, \verb$preempt$, and 	\\
																					& \verb$preempted$ statements.									\\ \hline
	\end{tabular}
	
	\caption{The structural translations of the ForeC statements in 
			 Table~\ref{table:forec_semantics} to the kernel statements 
			 in Table~\ref{table:forec_kernel}.}
	\label{table:forec_structural_translations}
\end{table}

Table~\ref{table:forec_structural_translations} shows how 
the ForeC statements of Table~\ref{table:forec_semantics}, 
are translated to the 
kernel statements of Table~\ref{table:forec_kernel}.
The translations for \verb$input$, \verb$output$, and 
\verb$pause$ are straightforward. The \verb$shared$ type-qualifier
is translated into a more compact form. The translation
of a \verb$par$ statement prefixes each thread body ($st$) 
with a unique identifier (\thread{}) which serves
as an abstract reference (handle) to the thread body.
The \verb$abort$ semantics is more complex and we 
separate its behaviour into four kernel statements:
\begin{itemize}
	\item \verb$abort$~$a$~\{$f$\}~\verb$when$~(\expression{}):
		  Defines the scope of the \verb$abort$ statement
		  with a unique identifier $a$. Preemption is triggered 
		  when \expression{} evaluates to a non-zero value.
	\item \verb$addAbort$($a$, \imm{}):
		  Initialises the preemption status of \verb$abort$ $a$,
		  with $\imm{} =$~\verb$0$ for a delayed \verb$abort$ and 
		  $\imm{} =$~\verb$1$ for an immediate \verb$abort$.
	\item \verb$preempt$($a$):
		  Instructs the executing thread to preempt if  
		  \verb$abort$ $a$'s preemption has been triggered.
	\item \verb$preempted$(\thread{}$_0$,~$\dots$,~\thread{}$_n$):
		  Instructs the executing thread to preempt if its child
		  threads, \thread{}$_0$ to \thread{}$_n$, were also preempted.
\end{itemize}
All \verb$abort$s in ForeC are translated into a combination
of these four kernel statements. Figure~\ref{fig:translate_abort} 
illustrates the translation of a strong and delayed \verb$abort$. 
The \verb$e$, \verb$f$, \verb$g$, and \verb$h$ are arbitrary 
compositions of kernel statements.
The original \verb$abort$ (line~\ref{translate_abort_0}) is replaced 
by the kernel \verb$abort$ statement (line~\ref{translate_abort_1}). 
The kernel \verb$addAbort$ statement is inserted before the 
\verb$abort$ body to initialise the \verb$abort$'s preemption 
status when it is first executed (line~\ref{translate_abort_addAbort}). 
The kernel \verb$preempt$ statement is placed immediately after 
the \verb$addAbort$ and after every \verb$pause$ statement in 
the \verb$abort$ body (lines~\ref{translate_abort_preempt_0} and 
\ref{translate_abort_preempt_1}). These are the places where strong
preemption can occur. 
An \verb$abort$ could contain nested \verb$par$ 
statements and child threads. When the inner 
most child threads take a triggered preemption, 
we propagate the preemption out to the enclosing 
\verb$abort$. The propagation is facilitated by placing 
the kernel \verb$preempted$ statement after every 
\verb$par$ statement in the \verb$abort$ body
(line~\ref{translate_abort_preempted}). The
arguments of the \verb$preempted$ statement are
the child threads of the \verb$par$ statement.

The translation of a weak and delayed \verb$abort$ 
is similar to the strong and delayed \verb$abort$, 
as illustrated in Figure~\ref{fig:translate_weak_abort}. 
The only difference is that the \verb$preempt$ statements 
are only placed before every \verb$pause$ statement 
in the \verb$abort$ body (line~\ref{translate_weak_abort_preempt}). 
For the immediate \verb$abort$ variants, the \imm{} 
argument of the \verb$addAbort$ statement becomes
\verb$1$. E.g., the \verb$addAbort$ statements on 
line~\ref{translate_abort_addAbort} in Figure~\ref{fig:translate_abort} 
and Figure~\ref{fig:translate_weak_abort} become 
\verb$addAbort(a,1)$.

For nested \verb$abort$ statements, the \verb$abort$s 
can be translated in an arbitrary order. Figure~\ref{fig:translate_nested_abort}
illustrates the translation for four nested \verb$abort$s.
Only one \verb$preempted$ statement is needed after every 
\verb$par$ statement (line~\ref{translate_nested_abort_preempted}). 
Recall that the preemption of an outer \verb$abort$ takes 
precedence over its inner \verb$abort$s. We order the 
\verb$preempt$ statements before and after every \verb$pause$ 
statement to preserve the precedences. For strong \verb$abort$s,
the \verb$preempt$ statement of an outer \verb$abort$ is always 
before those of its inner \verb$abort$s. E.g., the \verb$preempt$
statement for \verb$abort$ \verb$a0$ (line~\ref{translate_nested_abort_a0}) 
is always before \verb$abort$ \verb$a2$'s (line~\ref{translate_nested_abort_a2}). 
For weak \verb$abort$s, the \verb$preempt$ statement of an outer 
\verb$abort$ is always after those of its inner \verb$abort$s. 
This allows the \verb$abort$ statements inside a weak \verb$abort$ 
to execute one last time before they are preempted. E.g., the 
\verb$preempt$ statement for \verb$abort$ \verb$a1$ 
(line~\ref{translate_nested_abort_a1}) is always after 
\verb$abort$ \verb$a3$'s (line~\ref{translate_nested_abort_a3}).

\begin{figure}
	\begin{minipage}[b]{0.45\textwidth}
		\subfloat[Strong and delayed abort.] {
			\begin{minipage}[b]{\textwidth}
				\lstinputlisting[style=full]{./code/translate_abort.forec}
				\label{fig:translate_abort}
			\end{minipage}
		}
		
		\subfloat[Weak and delayed abort.] {
			\begin{minipage}[b]{\textwidth}
				\lstinputlisting[style=full]{./code/translate_weak_abort.forec}
				\label{fig:translate_weak_abort}
			\end{minipage}
		}
	\end{minipage}
	\hfill
	\subfloat[Nested aborts.] {
		\begin{minipage}[b]{0.45\textwidth}
			\lstinputlisting[style=full]{./code/translate_nested_abort.forec}
			\label{fig:translate_nested_abort}
		\end{minipage}
	}
	
	\caption{Illustration of translating aborts.}
	\label{fig:translate_aborts}
\end{figure}


%-----------------------------------------------------------------------------

%\begin{figure}
	%\subfloat[Translated program.] {
		%\begin{minipage}[b]{0.45\textwidth}
			%\lstinputlisting[style=full]{./code/transform_forec_translated.forec}
			%\label{fig:transform_forec_translated}
		%\end{minipage}
	%}
	%\hfill
	%\subfloat[Inlined.] {
		%\begin{minipage}[b]{0.45\textwidth}
			%\lstinputlisting[style=full]{./code/transform_forec_inlining.forec}
			%\label{fig:transform_forec_inlining}
		%\end{minipage}
	%}
	
	%\subfloat[Renamed and hoisted.] {
		%\begin{minipage}[b]{0.45\textwidth}
			%\lstinputlisting[style=full]{./code/transform_forec_hoisting.forec}
			%\label{fig:transform_forec_hoisting}
		%\end{minipage}
	%}
	%\hfill
	%\subfloat[Deferenced.] {
		%\begin{minipage}[b]{0.45\textwidth}
			%\lstinputlisting[style=full]{./code/transform_forec_pointers.forec}
			%\label{fig:transform_forec_pointers}
		%\end{minipage}
	%}
	%
	%\caption{Illustration of the simplifying program transformations.}
	%\label{fig:transform_forec}
%\end{figure}

%\subsection{Simplifying Transformations}
%This section describes three simplifying transformations that are 
%used to simplify the structure of ForeC programs. The short ForeC 
%program in Figure~\ref{fig:transform_forec_original} is used to illustrate 
%each transformation and is written with the ForeC kernel statements. 
%The \verb$main$ function contains a 
%locally declared shared variable \verb$s$ (line~\ref{fig:transform_forec_original_s}). 
%The \verb$par$ statement forks two instances of function \verb$t0$ to 
%execute in parallel. The parameter of function \verb$t0$ 
%(line~\ref{fig:transform_forec_original_c}) is initialised with a 
%reference to the shared variable \verb$s$ (the argument).

%\subsubsection{Inlining Functions}
%The inlining of functions is a standard transformation performed by
%most compilers. Each function call is replaced by the function's body. 
%Statements are added at the start of the body to
%initialise all function parameters to their given arguments. 
%Figure~\ref{fig:transform_forec_inlining} gives an example of inlining
%functions. On lines~\ref{fig:transform_forec_inlining_c_0} and 
%\ref{fig:transform_forec_inlining_c_1}, the function parameter \verb$c$ is
%initialised to \verb$&s$ because that was the original argument for the 
%function calls.

%\subsubsection{Renaming and Hoisting Variables to the Global Scope}
%The goal of this transformation is to avoid the need to consider
%variable declarations and scoping in the semantics. This is achieved by hoisting all 
%variable declarations into the global scope. However, variables declared
%in different scopes, but with the same identifier, will clash when 
%hoisted up to the same global scope. We also need to ensure 
%that the variables accessible to each expression remains unchanged
%after the variables have been hoisted. To avoid name clashes, but also 
%preserve the scoping behaviour, we rename each variable and all its 
%scoped occurrences to the same unique identifier before hoisting its 
%declaration up. Figure~\ref{fig:transform_forec_hoisting} gives an
%example of hoisting the declarations in Figure~\ref{fig:transform_forec_inlining}
%up to the global scope. Note that the constant pointer initialisations are 
%hoisted up with the declarations.

%\subsubsection{Dereferencing Pointers}
%Given that all functions are inlined, all variable declarations are 
%in global scope, and only constant pointers are allowed, we can 
%statically dereference pointers to the variables they point to. 
%Figure~\ref{fig:transform_forec_pointers} gives an example of 
%converting pointers in Figure~\ref{fig:transform_forec_hoisting} 
%to the actual variable.


%-----------------------------------------------------------------------------

\subsection{Notation}
We use two forms of notation for the rewrite rules, in 
the style of Structural Operational Semantics~\cite{semantics_sos}.
The first style:
\begin{equation*}
	\langle State \rangle ~ \thread:\body 
		\xrightarrow[\Input]{~~k~~} 
	\langle State' \rangle ~ \thread:\body'
\end{equation*}
means that, in the program state $\langle State \rangle$ and with inputs \Input{}, 
the program fragment \body{} of thread \thread{} reacts 
and modifies the program state to $\langle State' \rangle$, emits the 
completion code $k$, and becomes the new program fragment 
$\body'$ of thread \thread{}. The second style:
\begin{equation*}
	\langle State \rangle
		\xrightarrow{~~~~~} 
	\langle State' \rangle
\end{equation*}
means that, the program state $\langle State \rangle$ changes to 
the new program state $\langle State' \rangle$ because of some 
transition triggered by a thread.

Let \Thread{} be the set of all 
threads in the program. Each thread 
is associated with a unique program fragment 
\body{} (the thread body). Let $\thread:\body$ 
denote that \body{} is the body of thread 
\thread{}. Thread \thread{} can be treated
as an abstract reference (handler) to its 
body \body{}. Then, $\langle State \rangle$ is defined as the 
$\langle \Active, \Inactive, \Paused, \Abort, \Environment \rangle$ tuple, where:
\begin{itemize}
	\item $\Active \subseteq \Thread$ is the set of (active) threads that can be chosen for execution.
	\item $\Inactive \subseteq \Thread$ is the set of (inactive) threads that cannot be chosen for execution
		  until their child threads have terminated.
	\item $\Paused \subseteq \Thread$ is the set of (paused) threads that have executed a \verb$pause$.
	\item \Abort{} is a mapping of \verb$abort$ identifiers to values (\Val{}) representing their preemption
		  status. Let $\mathcal{A}$ denote the set of \verb$abort$ identifiers, then
		  $\Abort: \mathcal{A} \to Val$. An \verb$abort$ with a non-zero value means that its
		  preemption has been triggered.
	\item \Environment{}: An environment that maps thread identifiers and a special 
		  identifier \Global{} to a store (\Store{}) of variables (\Var{}) and their 
		  valuations (\Val{}). Let $\Id = \Thread \cup \{\Global\}$ denote the set of 
		  thread identifiers and \Global{}. Then, $\Environment: \Id \to \Store$ and 
		  $\Store: \Var \to Val$. Both \Environment{} and \Store{} are partial functions. 
		  All the global variables are contained in $\Environment[\Global]$ 
		  and the copies of shared variables are contained in $\Environment[\thread]$
		  for each thread.
\end{itemize}
The completion code of a transition is encoded by $k$, where:
\begin{equation*}
	k = \begin{cases}
		0					& \text{If the transition terminates.}									\\
		1					& \text{If the transition pauses, but resumes in the next global tick.}	\\
		a \in \mathcal{A}	& \text{If the transition takes the preemption of abort $a$.}			\\
		\bot				& \text{If the transition continues.}
	\end{cases}
\end{equation*}


%-----------------------------------------------------------------------------

\subsection{Predicates}
To keep the ForeC semantics concise, we define predicates to return
statically known information about the program. The first set of
predicates describes the genealogy of the threads:
\begin{itemize}
	\item $\Parent(\thread)$:
		Returns the parent thread of thread \thread{}.
		However, if \thread{} is the \verb$main$ thread of the
		program, then $\top$ is returned.
	\item $\Children(\thread)$:
		Returns a set containing all the child threads that can be forked by the 
		thread \thread{}. An empty set is returned if thread 
		\thread{} has no \verb$par$ statement.
	\item $\Siblings(\thread)$:
		Returns a set containing all the sibling threads of thread \thread{}.
		An empty set is returned if \thread{} is the \verb$main$ thread
		or is the only argument in a \verb$par$ statement.
\end{itemize}
The second set of predicates describes shared variables and preemptions:
\begin{itemize}
	\item $\Shared(\store)$:
		Returns the set of all shared variables in the given \store{}.
	\item $\Combinator(\var)$:
		Returns the combine function specified for the shared variable \var{}.
	\item $\GetExp(a)$:
		Returns the expression \expression{} from the \verb$abort$ statement of \verb$abort$ $a$.
\end{itemize}
As an example, Figure~\ref{fig:forec_predicates} shows the predicates being used on
the program in Figure~\ref{fig:transform_forec_refactored}:
\begin{figure}[ht]
	\centering
	\begin{tabular}{|l l l|}
		\hline
												&																			&														\\
		$\Parent($\verb$main$$) = \top$			& $\Children($\verb$main$$) = \lbrace$\verb$t0$$,~$\verb$t1$$\rbrace$		& $\Siblings($\verb$main$$) = \lbrace ~ \rbrace$		\\
		$\Parent($\verb$t0$$) =~$\verb$main$	& $\Children($\verb$t0$$) = \lbrace ~ \rbrace$								& $\Siblings($\verb$t0$$) = \lbrace$\verb$t1$$\rbrace$	\\
		$\Parent($\verb$t1$$) =~$\verb$main$	& $\Children($\verb$t1$$) = \lbrace ~ \rbrace$								& $\Siblings($\verb$t1$$) = \lbrace$\verb$t0$$\rbrace$	\\
												&																			&														\\
		$\Shared(\Environment[\Global]) = \lbrace$\verb$s_main$$\rbrace$	& $\Combinator($\verb$s_main$$) =~$\verb$plus$	& $\GetExp($\verb$a_0$$) =~$\verb$s_main>10$			\\
												&																			&														\\
		\hline
	\end{tabular}
	
	\caption{Example of applying predicates to the program of Figure~\ref{fig:transform_forec_refactored}.}
	\label{fig:forec_predicates}
\end{figure}

%-----------------------------------------------------------------------------

\subsection{The Structural Operational Semantics}
We use the first form of the rewrite rules to 
describe the execution behaviour of a thread body. 
Then, we use the second form of the rewrite rules to 
describe the thread termination and global synchronisation
behaviours. Each rule is designed to be applied to 
an arbitrary active thread ($\thread \in \Active$). 
An active thread may require an arbitrary number of 
rewrites, potentially interleaved with the rewrites 
of another thread, before it pauses or terminates. 
The semantics of a ForeC program does not depend on 
the rewrite order and is deterministic 
(see Theorem~\ref{thm:deterministic}).

\subsubsection{The nop Statement}
The \verb$nop$ statement does nothing, emits
the completion code of $0$, and terminates 
immediately:
\begin{equation*}	
	\tag{nop}
	\label{forec:nop}
	\langle \thread \in \Active, \Inactive, \Paused, \Abort, \Environment \rangle ~ \thread:nop
		\xrightarrow[~~\Input~~]{0} 
	\langle \Active, \Inactive, \Paused, \Abort, \Environment \rangle ~ \thread: 
\end{equation*}

\subsubsection{The pause Statement}
The \verb$pause$ statement does nothing, emits the 
completion code of $1$, and terminates immediately:
\begin{equation*}
	\tag{pause}
	\label{forec:pause}
	\langle \thread \in \Active, \Inactive, \Paused, \Abort, \Environment \rangle ~ \thread:pause
		\xrightarrow[~~\Input~~]{1} 
	\langle \Active, \Inactive, \Paused, \Abort, \Environment \rangle ~ \thread:nop
\end{equation*}

\subsubsection{The Assignment Statement}
The assignment statement evaluates the expression
\expression{} into a value using the semantic
function $\Eval$ (described in Section~\ref{sec:forec_Eval}).
If \var{} is a shared variable (\ref{forec:assign-shared}), then the value is 
assigned to the thread's store, $\Environment[\thread][\var]$. 
This allows the thread to keep its modified copy of \var{} 
separate from the other threads.
Otherwise, if the variable \var{} is a private variable (\ref{forec:assign-private}), then the value 
is assigned to the actual variable, $\Environment[\Global][\var]$:
\begin{equation*}
	\tag{assign-shared}
	\label{forec:assign-shared}
	\frac{
			\var \in \Shared(\Environment[\Global])
		}{
			\langle \thread \in \Active, \Inactive, \Paused, \Abort, \Environment \rangle ~ \thread:\var = \expression
				\xrightarrow[~~\Input~~]{0} 
			\langle \Active, \Inactive, \Paused, \Abort, \Environment[\thread][\var \gets \Eval(\Environment, \thread, \expression)] \rangle ~ \thread:nop
		}
\end{equation*}

\begin{equation*}
	\tag{assign-private}
	\label{forec:assign-private}
	\frac{
			\var \notin \Shared(\Environment[\Global])
		}{
			\langle \thread \in \Active, \Inactive, \Paused, \Abort, \Environment \rangle ~ \thread:\var = \expression
				\xrightarrow[~~\Input~~]{0} 
			\langle \Active, \Inactive, \Paused, \Abort, \Environment[\Global][\var \gets \Eval(\Environment, \thread, \expression)] \rangle ~ \thread:nop
		}
\end{equation*}

\subsubsection{Sequencing}
For a sequence of two terms, the first term $f$
must be rewritten until it terminates before the 
second term can be rewritten (\ref{forec:seq-left}). 
The (\ref{forec:seq-left}) rule 
emits the completion code of the first term:
\begin{equation*}
	\tag{seq-left}
	\label{forec:seq-left}
	\frac{
			\langle \Active, \Inactive, \Paused, \Abort, \Environment \rangle ~ \thread:f
				\xrightarrow[~~\Input~~]{k \neq 0} 
			\langle \Active', \Inactive', \Paused', \Abort', \Environment' \rangle ~ \thread:f'
		}{
			\langle \thread \in \Active, \Inactive, \Paused, \Abort, \Environment \rangle ~ \thread:f;g
				\xrightarrow[~~\Input~~]{k} 
			\langle \Active', \Inactive', \Paused', \Abort', \Environment' \rangle ~ \thread:f';g
		}
\end{equation*}
Otherwise, if the first term terminates, then 
the (\ref{forec:seq-right}) rule is applied:
\begin{equation*}
	\tag{seq-right}
	\label{forec:seq-right}
	\frac{
			\langle \Active, \Inactive, \Paused, \Abort, \Environment \rangle ~ \thread:f
				\xrightarrow[~~\Input~~]{0} 
			\langle \Active', \Inactive', \Paused', \Abort', \Environment' \rangle ~ \thread:nop
		}{
			\langle \thread \in \Active, \Inactive, \Paused, \Abort, \Environment \rangle ~ \thread:f;g
				\xrightarrow[~~\Input~~]{\bot} 
			\langle \Active', \Inactive', \Paused', \Abort', \Environment' \rangle ~ \thread:g
		}
\end{equation*}

\subsubsection{The if-else Statement}
A conditional statement is rewritten into one of its branches, depending on 
the value of its expression:
\begin{equation*}
	\tag{then}
	\label{forec:then}
	\frac{
			\Eval(\Environment, \thread, \expression) \neq 0
		}{
			\langle \thread \in \Active, \Inactive, \Paused, \Abort, \Environment \rangle ~ \thread:\mathit{if}(\expression)~\lbrace f \rbrace~else~\lbrace g \rbrace
				\xrightarrow[~~\Input~~]{\bot} 
			\langle \Active, \Inactive, \Paused, \Abort, \Environment \rangle ~ \thread:f
		}
\end{equation*}

\begin{equation*}
	\tag{else}
	\label{forec:else}
	\frac{
			\Eval(\Environment, \thread, \expression) = 0
		}{
			\langle \thread \in \Active, \Inactive, \Paused, \Abort, \Environment \rangle ~ \thread:\mathit{if}(\expression)~\lbrace f \rbrace~else~\lbrace g \rbrace
				\xrightarrow[~~\Input~~]{\bot} 
			\langle \Active, \Inactive, \Paused, \Abort, \Environment \rangle ~ \thread:g
		}
\end{equation*}

\subsubsection{The while Loop}
For a loop statement, its body $f$ and condition expression \expression{} 
are unrolled once:
\begin{equation*}
	\tag{loop}
	\label{forec:loop}
			\langle \thread \in \Active, \Inactive, \Paused, \Abort, \Environment \rangle ~ \thread:while(\expression)~\lbrace f \rbrace
				\xrightarrow[~~\Input~~]{\bot} 
			\langle \Active, \Inactive, \Paused, \Abort, \Environment \rangle ~ \thread:\mathit{if}(\expression)~\lbrace f;while(\expression)~\lbrace f \rbrace \rbrace~else~\lbrace nop \rbrace
\end{equation*}

\subsubsection{The par Statement}
The \verb$par$ statement places its child threads 
($\thread_0,~\dots,~\thread_n$) into the set of 
active threads and makes the parent thread (\thread{}) 
inactive:
\begin{equation*}
	\tag{par}
	\label{forec:par}
	\langle \thread \in \Active, \Inactive, \Paused, \Abort, \Environment \rangle ~ \thread:par(\thread_0:\lbrace f_0 \rbrace,~\dots,~\thread_n:\lbrace f_n \rbrace)
		\xrightarrow[~~\Input~~]{\bot} 
	\langle \lbrace \thread_0,~\dots,~\thread_n \rbrace \cup \Active \setminus \lbrace \thread \rbrace, \lbrace \thread \rbrace \cup \Inactive, \Paused, \Abort, \Environment \rangle ~ \thread:nop
\end{equation*}

\subsubsection{Preemption}
The \verb$addAbort$ statement creates a new mapping between the 
\verb$abort$ identifier $a$ and its preemption status in \Abort{}. 
For an immediate \verb$abort$ (i.e., $\imm = 1$), the immediate 
semantics is captured by initialising the preemption status with 
an evaluation of the \verb$abort$ expression \expression{}:
\begin{equation*}
	\tag{imm-abort}
	\label{forec:imm-abort}
	\frac{
			\imm = 1
		}{
			\langle \thread \in \Active, \Inactive, \Paused, \Abort, \Environment \rangle ~ \thread:addAbort(a, \imm)
				\xrightarrow[~~\Input~~]{0} 
			\langle \Active, \Inactive, \Paused, \Abort[a \gets \Eval(\GetExp(a))], \Environment \rangle ~ \thread:nop
		}
\end{equation*}
Otherwise, the (\ref{forec:del-abort}) rule captures the delayed 
semantics by initialising the preemption status to $0$:
\begin{equation*}
	\tag{del-abort}
	\label{forec:del-abort}
	\frac{
			\imm \neq 1
		}{
			\langle \thread \in \Active, \Inactive, \Paused, \Abort, \Environment \rangle ~ \thread:addAbort(a, \imm)
				\xrightarrow[~~\Input~~]{0} 
			\langle \Active, \Inactive, \Paused, \Abort[a \gets 0], \Environment \rangle ~ \thread:nop
		}
\end{equation*}

The \verb$preempt$ statement checks if \verb$abort$ $a$'s
preemption has been triggered. If preemption has not been
triggered (i.e., $\Abort[a] = 0$), then the preemption is 
not taken by emitting $0$:
\begin{equation*}
	\tag{not-preempt}
	\label{forec:not-preempt}
	\frac{
			\Abort[a] = 0
		}{
			\langle \thread \in \Active, \Inactive, \Paused, \Abort, \Environment \rangle ~ \thread:preempt(a)
				\xrightarrow[~~\Input~~]{0} 
			\langle \Active, \Inactive, \Paused, \Abort, \Environment \rangle ~ \thread:nop
		}
\end{equation*}
Otherwise, the preemption is taken by emitting the \verb$abort$ identifier $a$:
\begin{equation*}
	\tag{preempt}
	\label{forec:preempt}
	\frac{
			\Abort[a] \neq 0
		}{
			\langle \thread \in \Active, \Inactive, \Paused, \Abort, \Environment \rangle ~ \thread:preempt(a)
				\xrightarrow[~~\Input~~]{a} 
			\langle \Active, \Inactive, \Paused, \Abort, \Environment \rangle ~ \thread:nop
		}
\end{equation*}

The \verb$preempted$ statement emits the \emph{overall} completion code 
of the threads in its argument. The semantic function $\Preempted$ 
(described in Section~\ref{sec:forec_Preempted}) determines the overall 
completion code:
\begin{equation*}
	\tag{preempted}
	\label{forec:preempted}
	\langle \thread \in \Active, \Inactive, \Paused, \Abort, \Environment \rangle ~ \thread:preempted(\thread_0,~\dots,~\thread_n)
		\xrightarrow[~~\Input~~]{k \in \lbrace 0 \rbrace \cup \mathcal{A}} 
	\langle \Active, \Inactive, \Paused, \Abort, \Environment \rangle ~ \thread:nop
\end{equation*}
with $k = \Preempted(\lbrace \thread_0,~\dots,~\thread_n \rbrace)$.

The \verb$abort$ statement terminates if its body 
emits a completion code of $0$ or $a$. A completion
code of $a$ means that the preemption of \verb$abort$ 
$a$ has been taken:
\begin{equation*}
	\tag{aborted}
	\label{forec:aborted}
	\frac{
			\langle \Active, \Inactive, \Paused, \Abort, \Environment \rangle ~ \thread:f
				\xrightarrow[~~\Input~~]{k \in \lbrace 0, a \rbrace} 
			\langle \Active', \Inactive', \Paused', \Abort', \Environment' \rangle ~ \thread:f'
		}{
			\langle \thread \in \Active, \Inactive, \Paused, \Abort, \Environment \rangle ~ \thread:abort~a~\{f\}~when~(\expression)
				\xrightarrow[~~\Input~~]{0} 
			\langle \Active', \Inactive', \Paused', \Abort', \Environment' \rangle ~ \thread:nop
		}
\end{equation*}
Otherwise, if the \verb$abort$ statement body does 
not emit a completion code of $0$ or $a$, then
the \verb$abort$ behaves as its body. The 
(\ref{forec:not-aborted}) rule may emit the identifier
of an enclosing \verb$abort$:
\begin{equation*}
	\tag{not-aborted}
	\label{forec:not-aborted}
	\frac{
			\langle \Active, \Inactive, \Paused, \Abort, \Environment \rangle ~ \thread:f
				\xrightarrow[~~\Input~~]{k \notin \lbrace 0, a \rbrace} 
			\langle \Active', \Inactive', \Paused', \Abort', \Environment' \rangle ~ \thread:f'
		}{
			\langle \thread \in \Active, \Inactive, \Paused, \Abort, \Environment \rangle ~ \thread:abort~a~\{f\}~when~(\expression)
				\xrightarrow[~~\Input~~]{k} 
			\langle \Active', \Inactive', \Paused', \Abort', \Environment' \rangle ~ \thread:abort~a~\{f'\}~when~(\expression)
		}
\end{equation*}
%Our method of preemption uses the natural unwinding
%of nested child threads back to their parent thread.
%Another method is to forcefully remove all threads 
%within the abort scope. However, this requires the 
%threads to check that all threads within the abort
%scope are no longer active. The copies of shared 
%variables will also need to be combined and assigned
%to the ancestor thread of the abort statement.

\subsubsection{Thread termination}
A thread terminates if its body emits either $0$ or an 
\verb$abort$ identifier as its completion code, 
$k \in \lbrace 0 \rbrace \cup \mathcal{A}$. The 
terminating thread is removed from the set of active 
threads. The (\ref{forec:term1}) rule applies if 
some of the thread's siblings have not yet terminated
(i.e., some siblings are in \Active{}, \Inactive{}, or \Paused{}):
\begin{equation*}	
	\tag{term1}
	\label{forec:term1}
	\frac{
			\thread \neq main
			\quad
			\thread \xrightarrow{k \in \lbrace 0 \rbrace \cup \mathcal{A}} \thread
			\quad
			\Siblings(\thread) \cap (\Active \cup \Inactive \cup \Paused) \neq \lbrace ~ \rbrace
		}{
			\langle \thread \in \Active, \Inactive, \Paused, \Abort, \Environment \rangle
				\xrightarrow{~~~~~} 
			\langle \Active \setminus \lbrace \thread \rbrace, \Inactive, \Paused, \Abort, \Environment \rangle
		}
\end{equation*}
If all of the thread's siblings have terminated, 
then a join occurs and the parent thread is resumed 
(\ref{forec:term2}). The modified copies of shared variables 
from the terminating thread and its siblings are combined 
and assigned to its parent thread. This is captured by the 
semantic function $\Combine$ 
(described in Section~\ref{sec:forec_Combine}):
\begin{equation*}	
	\tag{term2}
	\label{forec:term2}
	\frac{
			\thread \neq main
			\quad
			\thread \xrightarrow{k \in \lbrace 0 \rbrace \cup \mathcal{A}} \thread
			\quad
			\Siblings(\thread) \cap (\Active \cup \Inactive \cup \Paused) = \lbrace ~ \rbrace
		}{
			\langle \thread \in \Active, \Inactive, \Paused, \Abort, \Environment \rangle
				\xrightarrow{~~~~~} 
			\langle \lbrace \Parent(\thread) \rbrace \cup \Active \setminus \lbrace \thread \rbrace, \Inactive \setminus \lbrace \Parent(\thread) \rbrace, \Paused, \Abort, \Environment' \rangle 
		}
\end{equation*}
with $\Environment' = \Combine(\Environment, \lbrace \thread \rbrace \cup \Siblings(\thread), \Parent(\thread))$.

Otherwise, if the \verb$main$ thread terminates, then
the entire program terminates (\ref{forec:term3}). The
modified copies of shared variables are assigned to the
actual shared variables using the semantic function
$\Combine$ (described in Section~\ref{sec:forec_Combine}):
\begin{equation*}	
	\tag{term3}
	\label{forec:term3}
	\frac{
			\thread = main
			\quad
			\thread \xrightarrow{k \in \lbrace 0 \rbrace \cup \mathcal{A}} \thread
		}{
			\langle \thread \in \Active, \Inactive, \Paused, \Abort, \Environment \rangle
				\xrightarrow{~~~~~} 
			\langle \lbrace ~ \rbrace, \Inactive, \Paused, \Abort, \Combine(\Environment, \lbrace \thread \rbrace, \Global) \rangle 
		}
\end{equation*}

\subsubsection{Synchronisation}
A thread reaches the end of its local tick if
its body emits a completion code of $1$.
If a thread is not the last to pause, then the 
the thread is placed into the set of paused threads:
\begin{equation*}
	\tag{local-tick}
	\label{forec:local-tick}
	\frac{
			\Active \neq \lbrace \thread \rbrace
			\quad
			\thread \xrightarrow{~~1~~} \thread
		}{
			\langle \thread \in \Active, \Inactive, \Paused, \Abort, \Environment \rangle
				\xrightarrow{~~~~~} 
			\langle \Active \setminus \lbrace \thread \rbrace, \Inactive, \lbrace \thread \rbrace \cup \Paused, \Abort, \Environment \rangle
		}
\end{equation*}
Otherwise, if the thread is the last to pause, then a global 
tick occurs (\ref{forec:global-tick}). All the modified copies of 
shared variables are combined and assigned to the actual shared 
variables using the semantic function $\Combine$ (described in 
Section~\ref{sec:forec_Combine}). To allow the next global tick
to proceed, the paused threads are made active and the preemption
statuses are updated using the semantic function 
$\Preemptions$ (described in Section~\ref{sec:forec_Preemptions}):
\begin{equation*}
	\tag{global-tick}
	\label{forec:global-tick}
	\frac{
			\Active = \lbrace \thread \rbrace
			\quad
			\thread \xrightarrow{~~1~~} \thread
		}{
			\langle \thread \in \Active, \Inactive, \Paused, \Abort, \Environment \rangle
				\xrightarrow{~~~~~} 
			\langle \lbrace \thread \rbrace \cup \Paused, \Inactive, \lbrace ~ \rbrace, \Preemptions(\Environment', \Abort), \Environment' \rangle
		}
\end{equation*}
with $\Environment' = \Combine(\Environment, \Thread, \Global)$.

%-----------------------------------------------------------------------------

\subsection{Semantic Functions}
This section describes the various semantic functions 
used by the semantics of ForeC.

\subsubsection{\textsc{Eval}}
\label{sec:forec_Eval}
The semantic function $\Eval(\Environment, \thread, \expression)$ 
follows the evaluation rules of the C-language to 
evaluate the expression \expression{} and returns 
the result. During the evaluation, the value of a 
variable is retrieved by the semantic function $\GetVal$ 
(described in Section~\ref{sec:forec_GetVal}).

\begin{algorithm}[t]
	\begin{algorithmic}[1]
		\Function{$\GetVal$}{\Environment{}, \thread{}, \var{}}
			\If {$\var \in \Environment[\thread]$}	
				\State \Return $\Environment[\thread][\var]$							\Comment{Return the thread's value for \var{}.}\label{algo:GetVal_local}
			\Else
				\State $\thread_a = \GetAncestorWithCopy(\Environment, \thread, \var)$	\Comment{Get the ancestor with a copy of \var{}.}
				\If {$\thread_a \neq \top$}
					\State \Return $\Environment[\thread_a][\var]$						\Comment{Return the ancestor's value for \var{}.}\label{algo:GetVal_parent}
				\Else
					\State \Return $\Environment[\Global][\var]$						\Comment{Return the value of the actual \var{}.}\label{algo:GetVal_global}
				\EndIf
			\EndIf
		\EndFunction
	\end{algorithmic}
	
	\caption{Gets the value of a given variable.}
	\label{algo:GetVal}
\end{algorithm}

\subsubsection{\textsc{GetVal}}
\label{sec:forec_GetVal}
When evaluating an expression, the semantic function
$\GetVal(\Environment, \thread, \var)$ is used 
to retrieve the value of each variable. 
This function is described by Algorithm~\ref{algo:GetVal}. 
The inputs to the algorithm are: the program's environment 
\Environment{}, the thread \thread{} that the expression 
belongs to, and the variable \var{} that we want the value 
of. The output is a value \val{}. If \var{} is in thread 
\thread{}'s store, then its value in the store is returned 
(line~\ref{algo:GetVal_local}). Otherwise, we find the ancestor of
thread \thread{} that has a copy of \var{} by using the
semantic function $\GetAncestorWithCopy$ 
(described in Section~\ref{sec:forec_GetAncestorWithCopy}). 
If such an ancestor exists ($\thread_a \neq \top$), then
the value in the ancestor's store is returned (line~\ref{algo:GetVal_parent}). 
Otherwise, the value of the actual \var{} is returned
(line~\ref{algo:GetVal_global}).

\subsubsection{\textsc{GetAncestor}}
\label{sec:forec_GetAncestorWithCopy}
The semantic function $\GetAncestorWithCopy(\Environment, \thread, \var)$ 
finds thread \thread{}'s ancestor which has a copy 
of a given variable \var{} in its store. This function is described 
by Algorithm~\ref{algo:GetAncestorWithCopy}. The inputs
to the algorithm are: the program's environment 
\Environment{}, the thread \thread{} to start searching 
from, and the variable \var{} to find. The output is a
thread. The first ancestor we search through is thread 
\thread{}'s parent. If the parent has a copy of \var{}, 
then the parent is returned. Otherwise, the search continues 
through the thread's ancestry until a copy is found. On
line~\ref{algo:GetAncestorWithCopy_top1}, recall that 
$\Parent(\thread_a)$ returns $\top$ when $\thread_a$ is 
the program's \verb$main$ thread. If no ancestor is found, 
then $\top$ is returned (line~\ref{algo:GetAncestorWithCopy_top2}).

\begin{algorithm}[t]
	\begin{algorithmic}[1]
		\Function{$\GetAncestorWithCopy$}{\Environment{}, \thread{}, \var{}}
			\State $\thread_a = \Parent(\thread)$			\Comment{Get the parent.}
			\While {$\thread_a \neq \top$}					\Comment{While an ancestor exists.}
				\If {$\var \in \Environment[\thread_a]$}
					\State \Return $\thread_a$				\Comment{A copy of \var{} is in this ancestor's store.}
				\EndIf
				\State $\thread_a = \Parent(\thread_a)$		\Comment{continue searching through the ancestry.}\label{algo:GetAncestorWithCopy_top1}
			\EndWhile
			\State \Return $\top$							\Comment{No ancestor found.}\label{algo:GetAncestorWithCopy_top2}
		\EndFunction
	\end{algorithmic}
	
	\caption{Gets the ancestor that has a copy of a given variable.}
	\label{algo:GetAncestorWithCopy}
\end{algorithm}

\begin{algorithm}[t]
	\begin{algorithmic}[1]
		\Function{$\Combine$}{\Environment{}, \Thread{}, \id{}}
			\State $\Environment' = \Environment \setminus \Thread$								\Comment{Copy a subset of the environment.}\label{algo:Combine_copy_E}
			\ForAll {$\var \in \Shared(\Environment[\Global])$}									\Comment{For all shared variables.}
				\State $\Thread_c = \lbrace \thread \mid \forall \thread \in \Thread, \var \in \Environment[\thread] \rbrace$	\Comment{Get the threads that have a copy of \var{}.}\label{algo:Combine_find}
				\ForAll {$\thread_c \in \Thread_c,$ where $\Children(\thread_c) \cap \Thread_c \neq \lbrace ~ \rbrace$}
					\State $\Thread_c = \Thread_c \setminus \lbrace \thread_c \rbrace$			\Comment{Remove the parent threads.}\label{algo:Combine_remove}
				\EndFor
				\If {$\lvert \Thread_c \rvert > 0$}
					\State $\val = \Environment[\thread][\var],$ where $\thread \in \Thread_c$		\Comment{Use a copy to initialise \val{}.}\label{algo:Combine_init}
					\State $cf = \Combinator(\var)$												\Comment{Combine function for \var{}.}\label{algo:Combine_c}
					\ForAll {$\thread_c \in \Thread_c \setminus \lbrace \thread \rbrace$}
						\State $\val = cf(\val, ~ \Environment[\thread][\var])$					\Comment{Combine \val{} with the remaining copies.}\label{algo:Combine_combine}
					\EndFor
					\State $\Environment'[\id][\var \gets \val]$								\Comment{Assign the result to \id{}'s store.}\label{algo:Combine_assign}
				\EndIf
			\EndFor
			\State \Return $\Environment'$
		\EndFunction
	\end{algorithmic}
	
	\caption{Combines copies of shared variables for a given set of threads.}
	\label{algo:Combine}
\end{algorithm}

\subsubsection{\textsc{Combine}}
\label{sec:forec_Combine}
According to the (\ref{forec:assign-shared}) rule, a thread only 
makes a copy of a shared variable when it makes an assignment to the 
variable. The semantic function $\Combine(\Environment, \Thread, \id)$ 
combines the copies of shared variables for a set of threads \Thread{}. This function
is described by Algorithm~\ref{algo:Combine}. The inputs
to the algorithm are: the program's environment \Environment{},
the set of threads \Thread{} to combine, and the 
identifier \id{} of the store to assign the combined
values into. The output is an updated environment $\Environment{}'$.
The updated environment begins as a copy of \Environment{}, excluding
the stores belonging to the threads in \Thread{} (line~\ref{algo:Combine_copy_E}). 
For each shared variable \var{}, we find the threads that 
have a copy of \var{} (line~\ref{algo:Combine_find}). 
We remove the parent threads to prevent their copies from 
being combined with their children's copies (line~\ref{algo:Combine_remove}).
If at least one child thread remains, then we initialise 
a temporary value \val{} with a copy (line~\ref{algo:Combine_init}). 
The value is combined with the remaining copies (line~\ref{algo:Combine_combine}) 
by using the shared variable's combine function $cf$ 
(line~\ref{algo:Combine_c}). The combined value is assigned 
to the given \id{}'s store in $\Environment'$
(line~\ref{algo:Combine_assign}).

\subsubsection{\textsc{Preempted}}
\label{sec:forec_Preempted}
The semantic function $\Preempted(\Thread)$ determines the 
overall completion code for the set of threads \Thread{}. 
This function is only used in the (\ref{forec:preempted}) 
rule for the \verb$preempted$ statement. Recall that the 
structural translation of \verb$abort$s (Section~\ref{sec:formalSemantics_translations})
is the only use of \verb$preempted$ statements.
By construction, \verb$preempted$ statements are always
placed after \verb$par$ statements. The \verb$preempted$ 
statement takes, as arguments, the child threads of the 
preceding \verb$par$ statement, which are eventually passed 
into the semantic function $\Preempted$. All threads in 
\Thread{} would have terminated and their last completion code 
would be $0$ or $\abort \in \mathcal{A}$. Using the threads' 
last completion codes, Algorithm~\ref{algo:Preempted} describes
how the overall completion code is determined. The input 
to the algorithm is the set of child threads \Thread{} and
the output is a completion code. For each child thread, its
last completion code is retrieved (line~\ref{algo:Preempted_k}).
If the completion code is an \verb$abort$ identifier, then 
the thread was preempted by an enclosing \verb$abort$ statement.
Note that the \verb$abort$ statement would have enclosed the other 
child threads and preempted them too. Hence, the \verb$abort$ 
identifier can be returned (line~\ref{algo:Preempted_a}) without 
having to check the completion code of the remaining threads. 
Otherwise, if no child thread was preempted, $0$ is returned as 
the completion code (line~\ref{algo:Preempted_0}).

\begin{algorithm}[t]
	\begin{algorithmic}[1]
		\Function{$\Preempted$}{\Thread{}}
			\ForAll {$\thread \in \Thread$}
				\State $k =$ thread \thread's last completion code	\label{algo:Preempted_k}
				\If {$k \in \mathcal{A}$}
					\State \Return $k$								\label{algo:Preempted_a}
				\EndIf
			\EndFor
			\State \Return $0$										\label{algo:Preempted_0}
		\EndFunction
	\end{algorithmic}
	
	\caption{Returns the overall completion code for a set of threads.}
	\label{algo:Preempted}
\end{algorithm}

\subsubsection{\textsc{Preemptions}}
\label{sec:forec_Preemptions}
The semantic function $\Preemptions(\Environment, \Abort)$
updates the preemption statuses of all \verb$abort$s. This function
is described by Algorithm~\ref{algo:Preemptions}. The inputs 
to the algorithm are: the program's environment \Environment{}, 
and the set of \verb$abort$ identifiers \Abort{}. The output is 
\Abort{} with updated preemption statuses. For each \verb$abort$, 
line~\ref{algo:Preemptions_val} evaluates its expression 
using values of the actual variables. The result 
is assigned to the \verb$abort$ identifier as its new preemption 
status (line~\ref{algo:Preemptions_update}).

\begin{algorithm}[t]
	\begin{algorithmic}[1]
		\Function{$\Preemptions$}{\Environment{}, \Abort{}}
			\ForAll {$\abort \in \Abort$}
				\State $\val = \Eval(\Environment, \Global, \GetExp(a))$	\label{algo:Preemptions_val}
				\State $\Abort[\abort \gets \val]$							\label{algo:Preemptions_update}
			\EndFor
			\State \Return \Abort
		\EndFunction
	\end{algorithmic}
	
	\caption{Updates the preemption status of all abort identifiers.}
	\label{algo:Preemptions}
\end{algorithm}
